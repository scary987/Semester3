\documentclass[]{article}
\usepackage{amsmath}
%opening
\title{Übung 11}
\author{}
\date{}
\begin{document}

\maketitle



\section{}
\begin{gather*}
	\Omega = \{(x,y) | -1 \leq x,y \leq 1\} \\
	\text{Wahrscheinlichkeit für einen bestimmten Punkt im Kreis:} f((x,y))= \frac{1}{2}*\frac{1}{2} = \frac{1}{4} \\
		P (A) = \frac{|\text{Punkte im Kreis}|}{|\text{Punkte im Viereck}|} \\
	 = \frac{\text{Flächeninhalt vom Kreis}}{\text{Flächeninhalt vom Viereck}}\\
	 =\frac{\pi * r^2}{2* 2} \\
	 \frac{\pi * 1^2}{4}\\
	 = \frac{\pi}{4} = 0.78539816339..	
\end{gather*}

\end{document}
	A = \{ (x,y)| (x,y) \in \Omega \land \sqrt{(x^2+y^2)} \leq 1 \}\\