\documentclass[]{article}

%opening
\title{}
\author{}
\usepackage{graphicx}
\usepackage{amsmath}
\usepackage{amsfonts}
\usepackage{amssymb}
\usepackage{lmodern}
\usepackage{pgfplots}
\usepackage{comment}
\usepackage{tikz}
\usetikzlibrary{datavisualization.formats.functions}
\usetikzlibrary{arrows}
\begin{document}

\maketitle

\section{Aufgabe 8}
\subsection{a}
$ p(6) = 1/6$

$P(X) = \binom{10}2 * \frac{1}{6}^2 * \frac{5}{6}^8 $
\subsection{b}
Mann kann die Wahrscheinlichkeit so ausdruecken. Aus Einfachheit halber, habe ich einfach $ pn \in \mathbb{N} und (1-p)n = n-np $ angenommen.
Generell gilt: fuer X=x   mit der Wahrscheinlichkeit p und der Anzahl der Stichproben k und x < k, nat\"urlich.
Auch Hypergeometrische Verteilung mit n=n , r= np , s = (1-p)n und k = k

$P(X=x) = \binom{pn}{x} * \binom{(1-p)n}{k-x}  * \frac{1}{\binom{n}{k}} $
$$P(X=x) = \frac{(pn)!* ((1-p)n)! * k! * (n-k)!  }{x! * (pn -x)! * (k-x)! * ((1-p)n)! * n} $$


also 
$P(X = 4) = 0.0785 $
\subsection{c}
Wie bei b) r = 4, s = 6.
Da die Population sehr gross ist, h\"atte man es auch sicherlich mit der Binomialverteilung appromixieren koennen. 
Ergebniss berechnet mit dem Taschenrechner auf: https://keisan.casio.com/exec/system/1180573201
$ P (X = 4) = 0.088014 $

\subsection{d}

Poisson-Verteilung

$ \lambda = 4 $\\
$ P( X \geq 8 ) = \sum^{\infty}_{ k=8 } *\frac{\lambda^x}{x!}*e^{ -\lambda }  $\\
$ \overline{P \geq  }  = P( X \leq 7) =  \sum^7_0 \frac{\lambda^x}{x!}*e^{-\lambda}   $ 
$\overline{P}(X \geq 8)  =9,4886638420715*10^{-1}  $\\
Nach dem Rechner auf : 
https://matheguru.com/stochastik/poisson-verteilung.html
$ P(X \geq 8) = 0.05113361579285 $

\subsection{e}
$ P(X = 1) = \sum^{\infty}_{k= 0} (\frac{1}{2})^{2k+1} = \frac{1}{2} * \sum^{\infty}_{k=0} (\frac{1}{4})^k = \frac{2}{3}  $ Geometrische Reihe

\section{9} 

\subsection{a} 
\begin{tikzpicture}
\begin{axis}[
axis x line=center,
axis y line=center,
xtick={0, pi/2, pi},
ytick={-1, -1/2,0,1/2, 1},
xticklabels={,$\pi/2$ ,$\pi$},
xlabel={$x$ },
ylabel={$y$},
xlabel style={below right},
ylabel style={above left},
xmin=,
xmax=4,
ymin=-1,
ymax=1]
\addplot [mark=none,domain=0:4] {sin (deg(x)) /2};
\end{axis}
\end{tikzpicture}


Bemerkung: Kann sehr wohl eine Verteilung sein, da sie \"uberall positiv ist. 

Berechnung :
$$ \int 2 sin_x dx = 2 *-cos_x dx$$ \linebreak
$$ cos(\pi) = -1 $$ \linebreak
$$ cos(0) = 1 $$ \linebreak
$$ \int_{0}^{\pi} \frac {sin (x)}{2} dx = [\frac{-cos (x)}{2}  ]_{x =0 }^{x=\pi}  = \frac{1+1 }{2} = 1$$

Das ist eine Wahrscheinlichkeitsverteilung!
\subsection{b} 



\begin{tikzpicture}

\begin{axis}[
axis x line=center,
axis y line=center,
xtick={0, pi/2, pi, 3 *pi/2, 2* pi, 5 * pi /2 , 3 *pi},
ytick={-1, -1/2,0,1/2, 1},
xticklabels={,,$\pi$,,$2\pi$,, $3\pi$},
xlabel={$x$},
ylabel={$y$},
xlabel style={below right},
ylabel style={above left},
xmin=0,
xmax=10,
ymin=-1,
ymax=1]
\addplot [mark=none,domain=0:3 *pi] {sin (deg(x)) /2};
\end{axis}

\end{tikzpicture}


Bemerkung: Ist definitive keine Verteilung , da das Interval $[\pi , 2\pi ] $ im negativ ist und nach Definition $$
\forall x,y P( X \subset [i,j] ) \geq = 0 
 $$ und $$ P (X) = \int_i^j f(x) dx  $$, was definitiv kleiner als 0  ist, f\"ur das Intervall $[\pi, 2\pi] $
\linebreak
\newpage
\subsection{c}

\begin{comment}
\begin{tikzpicture}

\begin{axis}

[
axis x line=center,
axis y line=center,
xtick={0 ... 10},
%xticklabels={,,$\pi$,,$2\pi$,, $3\pi$},
ytick={-1, -1/2,0,1/2, 1},
xlabel={$x$},
ylabel={$y$},
xlabel style={below right},
ylabel style={above left},
xmin=0,
xmax=11,
ymin=-1,
ymax=1] 
\addplot [mark=none,domain=0:10] {x * exp(-x)};
\end{axis}

\end{tikzpicture}
\end{comment}



\begin{tikzpicture}

\begin{axis}[
axis x line=center,
axis y line=center,
xtick={0, 1 ,2,3,4,5,6,7,8,9, 10},
ytick={-1, -1/2,0,1/2, 1},
%xticklabels={,,$\pi$,,$2\pi$,, $3\pi$},
xlabel={$x$},
ylabel={$y$},
xlabel style={below right},
ylabel style={above left},
xmin=0,
xmax=10,
ymin=-1,
ymax=1]
\addplot [mark=none,domain=0:3 *pi] {x * exp(-x)};
\end{axis}

\end{tikzpicture}
Kann sein, dass es eine Wahrscheinlichkeitsverteilung ist, allerdings, muss man das noch mit dem Grenzwert berechnen. 
$$\int x*\mathrm{e}^{-x} dx = -\left(x+1\right)\mathrm{e}^{-x} +C = F(x) $$
$$ \lim_{ x \rightarrow \infty }  -\left(x+1\right)\mathrm{e}^{-x} = 0 $$
 $$ F(0) = -1  $$
 $$ \int_{0}^{\infty} x*\mathrm{e}^{-x} dx = [F(x)]_{x=0}^{x=\infty} = \lim_{ x \rightarrow \infty } F(x) - F(0)  $$ $$
 = 0 + 1 = 1
 $$


\subsection{d}

\begin{tikzpicture}
\begin{axis}[
axis x line=center,
axis y line=center,
xtick={0, 1 ,2,3,4,5,6,7,8,9, 10},
ytick={-1, -1/2,0,1/2, 1},
%xticklabels={,,$\pi$,,$2\pi$,, $3\pi$},
xlabel={$x$},
ylabel={$y$},
xlabel style={below right},
ylabel style={above left},
xmin=0,
xmax=10,
ymin=-1,
ymax=1]
\addplot [mark=none,domain=0:3 *pi] {x * x * exp(-x)};
\end{axis}

\end{tikzpicture}
\newline
Dass Gleiche was f\"ur Aufgabe c gilt.
\newline
Berechnung:
$ f(x) = x^2 *e^{-x} $
$$\int f(x) dx =  -\left(x^2+2x+2\right)\mathrm{e}^{-x} = F(x)
$$
$$\lim_{ x \rightarrow \infty }  F (x) = 0  $$
$$ F (0)  = 2$$
 $$ \int_{0}^{\infty} x^2*\mathrm{e}^{-x} dx =
 [F(x)]_{x=0}^{x=\infty} = \lim_{ x \rightarrow \infty } F(x) - F(0) = 2 -- 0  = 2 \neq 1
 $$
 Das ist keine Wahrscheinlichkeitsverteilung.
 
 
 \section{11}
 
 
 \subsection{i)}
 
 Nach Definition 
 
 $$
 F_X(t) = P(X \leq t) =\frac {  | \{ k: x_k \leq t \}  |} {n} \land n=4
 $$
  
 \begin{tikzpicture}
  \begin{axis}[
 %	legend pos=north west, 
 	axis x line=center,
 	axis y line=center,
 	xtick={ 1,2,3,4},
 	ytick={0, 0.25,1/2,0.75, 1},
 	%yticklabels={,,,,},
 	%xticklabels={,,$\pi$,,$2\pi$,, $3\pi$},
 	xlabel={$x$},
 	ylabel={ },
 	xlabel style={below right},
 	ylabel style={above left},
 	xmin=0,
 	xmax=4,
 	ymin=0,
 	ymax=1]
 	\addplot [blue, mark=none,thick,domain=0:1] {0.25};
 	\addplot [blue, mark=none,thick,domain=1:2] {0.5};
 	\addplot [blue, mark=none,thick,domain=2:3] {0.75};
	\addplot [blue, mark=none,thick,domain=3:4] {1}; 
 %	\addlegendentry{ f}
 	% \addplot [mark=none,domain=0:1] {y=1};
 	% \addplot [mark=none,domain=0:1] {y=4};
 %	\addplot [red, mark=none, thick, domain=1:4] { 0.25 *x};
 %	\addlegendentry{ F}
 \end{axis}
\draw[dashed] (1.72,0) -- (1.72,2.84);
\draw[dashed] (3.54,0) -- (3.54,4.26);
\draw[dashed] (5.2,0) -- ( 5.2, 5.68 );
\draw[dashed] (6.88,0) -- (6.88,5.68);
\end{tikzpicture}
 
 \subsection{ii)} Die Gleichverteilung muesste so aussehen: 
 
 \begin{tikzpicture}

 \begin{axis}[
 legend pos=north west, 
 axis x line=center,
 axis y line=center,
 xtick={ 1 ,4},
 xticklabels={ },
 ytick={0,0.25 , 1},
yticklabels={, $ \frac{1}{b-a}$, 1},
 xticklabels={a,b},
 xlabel={$x$},
 ylabel={ },
 xlabel style={below right},
 ylabel style={above left},
 xmin=0,
 xmax=4,
 ymin=0,
 ymax=1]
 \addplot [blue, mark=none,thick,domain=1:4] {0.25};

 \addlegendentry{ f}
% \addplot [mark=none,domain=0:1] {y=1};
% \addplot [mark=none,domain=0:1] {y=4};
 \addplot [red, mark=none, thick, domain=1:4] { 0.25 *x};
 \addlegendentry{ F}
 \end{axis}
   \draw[dashed] (1.72,0) -- (1.72,1.42);
   \draw[dashed] (6.88,0) -- (6.88,1.42);
 \end{tikzpicture}

 Also f\"ur das Intervall auf [x,y], ist die Gleichverteilung offentsichtlich f\"ur  $ \forall a,b : a \geq x , b \leq y  \land a < b $, so gilt: 
 $$ 
 P (X = [a,b] ) = \frac{b-a}{y-x}
 $$
  Also wäre für  den Fall, die Wahrscheinlichkeit $$ P = \begin{cases}
  \frac{1}{b+a}  & x \in [a,b]  \\
   0 & sonst 
  \end{cases} 
  $$
  $$
  \int f(x) = F(x)   $$ $$ 
  F(x) =  \begin{cases}  \frac{x-a}{b-a} & x \in [a,b]\\ 
  							 0 &  x < a \\
  							 1& x> b \\
  		\end{cases} 
  $$
  Beweis der 4 Vorraussetzungen nach 2.15 
  
  
  
  \begin{itemize}
  	\item[i)] 	F ist monoton wachsend: Offentsichtlich
  	\item[ii)] F is rechtsseitig stetig: Offentsichtlich, da
  	stetig, $\frac{x-a}{b-a} $für $ x \in [a,b]  $ nur zwischen [0,1] und betragen kann und stetig w\"achst.
  	\item[iii)] Per Definition.
  	\item[iiii)] Per Definition.
  \end{itemize}

\end{document}
