% !TEX TS-program = pdflatex
% !TEX encoding = UTF-8 Unicode

% This is a simple template for a LaTeX document using the "article" class.
% See "book", "report", "letter" for other types of document.

\documentclass[11pt]{article} % use larger type; default would be 10pt

\usepackage[utf8]{inputenc} % set input encoding (not needed with XeLaTeX)
\usepackage{color}
\usepackage{amssymb}

%%% Examples of Article customizations
% These packages are optional, depending whether you want the features they provide.
% See the LaTeX Companion or other references for full information.

%%% PAGE DIMENSIONS
\usepackage{geometry} % to change the page dimensions
\geometry{a4paper} % or letterpaper (US) or a5paper or....
% \geometry{margin=2in} % for example, change the margins to 2 inches all round
% \geometry{landscape} % set up the page for landscape
%   read geometry.pdf for detailed page layout information

\usepackage{graphicx} % support the \includegraphics command and options

% \usepackage[parfill]{parskip} % Activate to begin paragraphs with an empty line rather than an indent
\usepackage{euler}
%%% PACKAGES
\usepackage{booktabs} % for much better looking tables
\usepackage{array} % for better arrays (eg matrices) in maths
\usepackage{paralist} % very flexible & customisable lists (eg. enumerate/itemize, etc.)
\usepackage{verbatim} % adds environment for commenting out blocks of text & for better verbatim
\usepackage{subfig} % make it possible to include more than one captioned figure/table in a single float
% These packages are all incorporated in the memoir class to one degree or another...

%%% HEADERS & FOOTERS
\usepackage{fancyhdr} % This should be set AFTER setting up the page geometry
\pagestyle{fancy} % options: empty , plain , fancy
\renewcommand{\headrulewidth}{0pt} % customise the layout...
\lhead{}\chead{}\rhead{}
\lfoot{}\cfoot{\thepage}\rfoot{}

%%% SECTION TITLE APPEARANCE
\usepackage{sectsty}
%\allsectionsfont{\sffamily\mdseries\upshape} % (See the fntguide.pdf for font help)
% (This matches ConTeXt defaults)

%%% ToC (table of contents) APPEARANCE
\usepackage[nottoc,notlof,notlot]{tocbibind} % Put the bibliography in the ToC
\usepackage[titles,subfigure]{tocloft} % Alter the style of the Table of Contents
\renewcommand{\cftsecfont}{\rmfamily\mdseries\upshape}
\renewcommand{\cftsecpagefont}{\rmfamily\mdseries\upshape} % No bold!

%%% END Article customizations

%%% The "real" document content comes below...

\title{\color{blue} Serie 5 }
\author{Tobias Reincke 218203884 \\}
%\date{} % Activate to display a given date or no date (if empty),
         % otherwise the current date is printed
\begin{document}
	
\maketitle
\section*{\textbf Aufgabe 16:}

\subsection*{a)}
$\frac{ F(Y >s+t) \land F(Y>s)}{F(Y>s)} = \frac{F(Y>s+t)}{F(Y>s)}$\\
= $\left\{\\
\begin{array}{cc}
  \frac{e^{-\mu (s+t)y}}{e^{-\mu s y}}   & s+t > 0 \land s  > 0 \\
  0 & s+t \leq 0 \land s \leq 0 
\end{array}\right. $\\
= $\left\{
 \begin{array}{cc}
  e^{-\mu t y}   & s+t > 0 \land s  >0 \\
  0 & s+t \leq 0 \land s \leq 0
\end{array} \right. $
\\=$F(Y>t)$
\section*{\textbf Aufgabe 17}
\subsection*{a)}
Sei $\mathbb{R}$   
$$ E(X) =
\sum_{x \in \mathbb{R} } x P(X = x) \\
= x \frac{ \lambda^x e^{-\lambda}}{x!}
= \sum_{0}^{\infty} x e^{-\lambda} \lambda^x \frac{1}{x!}
= \sum_{1}^{\infty} x e^{-\lambda} \lambda^x \frac{1}{x!}

$$
\subsection*{b) Exponentialverteilung}
Die Verteilung ist immer null für Werte kleiner 0, daher das Intergral über 0 bis $\infty$
Normaler Erwartungswert für stetige Funktionen mit Dichte $f_X(x) =x \lambda ^{-\lambda x}  $
$$ 
\int_{0}^{\infty}  x f_X(x) dx \\
= \int_{0}^{\infty} x \lambda  e^{-\lambda x} dx \\
$$
Es gilt: (partielle Integration)\\
$\int_a^b u(x)v'(x) dx = [ u(x) v(x) ]_a^b - \int_a^b u' (x) v(x) dx  \land u(x) = x \land v'(x) = \lambda e^{-\lambda x} $\\
Außerdem: $e^{\lambda x} = e ^ {x ^ \lambda} $\\
Setze ein :
$$
[-x  e^{-x \lambda} ]_0^\infty + \int_{0}^{\infty} 1 e^{-\lambda x} dx \\
=(0 - 0) + [ - \frac{1}{\lambda} e^{-\lambda x} ]_0^\infty \\
= 0 + ( 0 -  -\frac{e^0}{\lambda}  )
= 0 + ( 0 + \frac{1}{\lambda}) \\
= \frac{1}{\lambda}
$$
\subsection*{c)}
\subsection*{d) differentielle Gleichverteilung}
$$
E(X) = \sum_i x_i * P(X=x_i) 
$$
n sei die Mächtigkeit der Ergebnismenge($\Omega = \{1, ... , n\} $) bzw. die Menge aller $x_i$
$$
=\sum_i x_i * \frac{1}{n} \\
= \frac{1}{n} * \sum_i x_i  \\
= \frac{1}{n} * \frac{n * (n+1)}{2} \\
=\frac{n+1}{2}
$$

\end{document}
