% !TEX TS-program = pdflatex
% !TEX encoding = UTF-8 Unicode

% This is a simple template for a LaTeX document using the "article" class.
% See "book", "report", "letter" for other types of document.

\documentclass[11pt]{article} % use larger type; default would be 10pt

\usepackage[utf8]{inputenc} % set input encoding (not needed with XeLaTeX)
\usepackage{color}
\usepackage{amssymb}

%%% Examples of Article customizations
% These packages are optional, depending whether you want the features they provide.
% See the LaTeX Companion or other references for full information.

%%% PAGE DIMENSIONS
\usepackage{geometry} % to change the page dimensions
\geometry{a4paper} % or letterpaper (US) or a5paper or....
% \geometry{margin=2in} % for example, change the margins to 2 inches all round
% \geometry{landscape} % set up the page for landscape
%   read geometry.pdf for detailed page layout information

\usepackage{graphicx} % support the \includegraphics command and options

% \usepackage[parfill]{parskip} % Activate to begin paragraphs with an empty line rather than an indent
\usepackage{euler}
%%% PACKAGES
\usepackage{booktabs} % for much better looking tables
\usepackage{array} % for better arrays (eg matrices) in maths
\usepackage{paralist} % very flexible & customisable lists (eg. enumerate/itemize, etc.)
\usepackage{verbatim} % adds environment for commenting out blocks of text & for better verbatim
\usepackage{subfig} % make it possible to include more than one captioned figure/table in a single float
% These packages are all incorporated in the memoir class to one degree or another...

%%% HEADERS & FOOTERS
\usepackage{fancyhdr} % This should be set AFTER setting up the page geometry
\pagestyle{fancy} % options: empty , plain , fancy
\renewcommand{\headrulewidth}{0pt} % customise the layout...
\lhead{}\chead{}\rhead{}
\lfoot{}\cfoot{\thepage}\rfoot{}

%%% SECTION TITLE APPEARANCE
\usepackage{sectsty}
%\allsectionsfont{\sffamily\mdseries\upshape} % (See the fntguide.pdf for font help)
% (This matches ConTeXt defaults)
\usepackage{amsmath}
%%% ToC (table of contents) APPEARANCE
\usepackage[nottoc,notlof,notlot]{tocbibind} % Put the bibliography in the ToC
\usepackage[titles,subfigure]{tocloft} % Alter the style of the Table of Contents
\renewcommand{\cftsecfont}{\rmfamily\mdseries\upshape}
\renewcommand{\cftsecpagefont}{\rmfamily\mdseries\upshape} % No bold!

%%% END Article customizations

%%% The "real" document content comes below...

\title{\color{blue} Serie 5 }
\author{Tobias Reincke\\ Matrikelnummer: 218203884 \\}
%\date{} % Activate to display a given date or no date (if empty),
         % otherwise the current date is printed
\begin{document}
	
\maketitle
\section*{\textbf Aufgabe 15:}
\subsection*{a)} $\Omega = \{(w1,w2,w3)| w1,w2,w3 \in \{1,..,6\}\}$ \\ $\mathbb{P} := \mathbb{U}_{\Omega}$\\ 
 $W_1,W_2,W_3 \in$ $\mathbb{A}$ mit  $W_i$ = x bedeutet, dass im i-ten Wurf wurde x gewürfelt.  
\subsection*{b)}
Die insgesamte Anzahl an Kombinationen ist: $6^3$

Die Mächtigkeit von A beträgt für 6 * jede valide Kombination von w1 und w2 (da w3 beliebig gewählt werden kann. Die Anzahl von validen Kombinationen lässt sich durch folgende Summe darstellen. ( 1 2 ; 1 3 , .. , 2 3, .., 5 6 ) \\ \ \\
{$$\frac{|A|}{6}=\sum_{i=0}^{6} 6-i = \sum_{i=1}^{5} i = \frac{5*(5+1)}{2}=\frac{30}{2}=15
\\
\rightarrow P(A)=\frac{|A|}{6^3} = \frac{90}{6^3}$$ \\
Dasselbe analog für $B$ und $P(B)$ indem man w2 und w3 tauscht.
Zum Durchschnitt von $A$ und $B$:\\
\{(1,2,2),(1,2,3),(1,3,2), ...\}\\

Habe es hier einfach ausgerechnet: \\ $w1<w2<w3 \ \& \ w1<w3<w2$ sind equivalent in Größe, da einfach w2 und w3 vertauschen zum jeweils anderen Ergebnis führt..
$$|A\cap B| = |\{(w1,w2,w3)|w1<w2< w3 \land w1 <w3<w2  \land w1<w2=w3\}|\\ $$
 $$=2  \sum_{k=1}^{6} (k-1) * (6-k) + \sum_{k=1}^{6} 6-k = 2  \sum_{k=1}^6 k*(5-k) + \sum_{k=1}^5 k  \\  $$ \\ $$=2 \sum_{k=1}^5 {(k*5) - k*k} \ \ \ \     + 15 = 2*(5 * (\frac{5* (5+1) }{2}) - \frac{5(5+1)(2 * 5 +1)}{6}) +15 =2 (15 * 5 - 5 * 11) +15$$
 $$=2  (75 - 55) + 15= 2 *20+15 = 45  $$   \textit{Wird auch durch nachzählen erreicht} \\ 
 
$$P(A|B) =\frac{P(A \cap B )}{P(B)} =  \frac{\frac{45}{6^3}}{\frac{90}{6^3}} = \frac{1}{2} \neq P(A) $$\\
$\rightarrow P(A)$und  $  P(B)$  sind nicht von einander unabhängig.  Das Ergebnis bedeutet, dass sich A und B in der Hälfte der Ergebnisse schneiden, was auch logisch erklingt, da genauso viele Kombinationen wie in A zu A $\cap$ B gehören wie auch nicht.



\subsection*{c)}
x=W1  \\ 6 * 1/6 mal da jedes w2 und w3 funktioniert \\
P(X=x) =  $\frac{1}{6}$ *  $\frac{6}{6}$ * $\frac{6}{6}$ = $\frac{1}{6}$ \\
Y=W2 \\ 6 * 1/6 mal da jedes w1 und w3 funktioniert \\
P(Y=y) = $\frac{6}{6}$ * $\frac{1}{6}$ * $\frac{6}{6}$ = $\frac{1}{6}$ \\
\textit{ Irgendwie sehr offentsichtlicht.} \\
X $\cap$ Y = {(w1,w2,w3 | w1 = x , w2 = y)} \\
P (X=x, Y = y) = $\frac{1}{6}$ * $\frac{1}{6}$ * $\frac{6}{6}$ = $\frac{1}{36}$ = P (X $\cap$ Y) \\
P(X $|$ Y)= $\frac{P(X \cap Y)}{X} = \frac{\frac{1}{6}}{\frac{1}{36}}$    = $\frac{1}{6} = P (X) \rightarrow$ X und Y sind unabhängig.\\
\textit{Wer hätte es gedacht. }
\subsection*{d)}
S=w1 + w2 + w3 $\land$ (w1,w2,w3) $\in$ $\Omega \rightarrow S \in \{3, ... , 18\} $
$\\ \\
\begin{array}{ccccccccccccccccc} 
	\textbf{S =} & 3  & 4 & 5 & 6 & 7 & 8 & 9 & 10 & 11 & 12 & 13 & 14 & 15 & 16 & 17 & 18 \\
	
	\#K_S & 1   & 3 & 6 & 10  & 15  & 21 & 25 & 27 & 27  & 25  & 21 & 15  & 10 &6  & 3 &1  \\

\end{array}$\\

\#$K_S$= Anzahl Kombinationen \\
P(S=x)= $\frac{\#K_S(x)}{6^3}$\\

$\begin{matrix}
M =  & 1 & 2 & 3 & 4 & 5 & 6 \\
\#K_M =  & 1 & 7 & 19 & 37 & 61 & 91
\end{matrix}$ \\
\#$K_M$ = Anzahl Kombinationen \\
P(M=x)= $\frac{\#K_M(x)}{6^3}$
P (S=3 $\cap$ M=1 )= P(\{1,1,1\}) = $\frac{1}{216}$ $\neq$ P(M=1)P(S=3)

Ergebnismengen werden durch Wählen der Ergebnismenge veringert, sodass nicht mehr alle Ergebnismengen beim anderen Ergebnis mit reinspielen und deshalb die Wahrscheinlichkeit sich unproportional zu anderen Menge ändert, aufgrund der Würfelergebnisgleichverteilung.

\section*{\textbf Aufgabe 16:}

\subsection*{a)}
$\frac{ F(Y >s+t) \land F(Y>s)}{F(Y>s)} = \frac{F(Y>s+t)}{F(Y>s)}$\\
= $\left\{\\
\begin{array}{cc}
  \frac{e^{-\mu (s+t)y}}{e^{-\mu s y}}   & s+t > 0 \land s  > 0 \\
  0 & s+t \leq 0 \land s \leq 0 
\end{array}\right. $\\
= $\left\{
 \begin{array}{cc}
  e^{-\mu t y}   & s+t > 0 \land s  >0 \\
  0 & s+t \leq 0 \land s \leq 0
\end{array} \right. $
\\=$F(Y>t)$
\subsection*{b)}
{min $X_i$ $>t$ } $\rightarrow$ min($X_i$, ..., $X_n$) $>t$ $\rightarrow$ $\overline{F}_{X} = \mathbb{P}(min(X_1,...,X_n) >t5)=\\ \mathbb{P}(X_1>x,...,X_n >x)\\ = \mathbb{P}(X_1 >x)\mathbb{P} (X_2 >t) .. \mathbb{P}(X_n>n)$\\ = $\overline{F}_1 \cdot \overline{F}_1 \cdot ... \cdot  \overline{F}_n $ 

\subsection{c)} $Y = min \{X_1, ..., X_n \} = F_Y(x)=\mathbb{P}(Y \leq x) = 1 - P(Y >x)\\\textit{Identität Aufgabe b} \\= 1- \coprod_{i \in  \{1,...,n\}} \mathbb{P}(X_i) $ \\
$\mathbb{P}(X_i>x)=1 - \mathbb{P}(X_i \leq x) = 1 - (1 - \mathrm{e}^{-\lambda x}) = \mathrm{e}^{-\lambda x} $\\
$\rightarrow 1- \coprod_{i \in  \{1,...,n\}} \mathbb{P}(X_i) = 1- \coprod_{i \in  \{1,...,n\}} \mathrm{e}^{-\lambda x}  $ \\
$\rightarrow F_Y(x) = 1- \mathrm{e}^{-\lambda x n} \textbf{1}_{(0, \inf) (x)  }  $
\section*{\textbf Aufgabe 17}
Die Varianzen von Wikipedia.
\subsection*{a) Poissonverteilung}
Sei $\mathbb{R}$   
$$ E(X) =
\sum_{x \in \mathbb{R} } x P(X = x) \\
= x \frac{ \lambda^x e^{-\lambda}}{x!} \\
= \sum_{0}^{\infty} x e^{-\lambda} \lambda^x \frac{1}{x!} \\
= \sum_{1}^{\infty} x e^{-\lambda} \lambda^x \frac{1}{x!} \\
$$
 \textit{Ersetze x durch y:=x-1 }
$$
= \sum_{0}^{\infty} (y+1) e^{-\lambda} \lambda^{y+1} \frac{1}{y+1!} \\
= \sum_{0}^{\infty} (y+1) e^{-\lambda} \lambda^{y} \lambda \frac{1}{y+1!} \\
= \sum_{0}^{\infty} (y+1) e^{-\lambda} \lambda^{y} \lambda \frac{1}{(y+1) y!} \\ $$
$$
= \lambda \sum_{0}^{\infty}  e^{-\lambda} \lambda^{y}  \frac{1}{y!} = \lambda \sum_{0}^{\infty} P(Y=y)  \\
= \lambda	
$$
\noindent\makebox[\linewidth]{\rule{\textwidth}{0.4pt}}
$$ Var(X)= E(X^2) -E(X)^2 
\rightarrow  E(X^2) = Var (X) - E(X)^2 = \lambda + \lambda^2 $$

\subsection*{b) Exponentialverteilung}
Die Verteilung ist immer null für Werte kleiner 0, daher das Intergral über 0 bis $\infty$
Normaler Erwartungswert für stetige Funktionen mit Dichte $f_X(x) =x \lambda ^{-\lambda x}  $
$$ 
\int_{0}^{\infty}  x f_X(x) dx \\
= \int_{0}^{\infty} x \lambda  e^{-\lambda x} dx \\
$$
Es gilt: (partielle Integration)\\
$\int_a^b u(x)v'(x) dx = [ u(x) v(x) ]_a^b - \int_a^b u' (x) v(x) dx  \land u(x) = x \land v'(x) = \lambda e^{-\lambda x} $\\
Außerdem: $e^{\lambda x} = e ^ {x ^ \lambda} $\\
Setze ein :
$$
[-x  e^{-x \lambda} ]_0^\infty + \int_{0}^{\infty} 1 e^{-\lambda x} dx \\
=(0 - 0) + [ - \frac{1}{\lambda} e^{-\lambda x} ]_0^\infty \\
= 0 + ( 0 -  -\frac{e^0}{\lambda}  )
= 0 + ( 0 + \frac{1}{\lambda}) \\
= \frac{1}{\lambda}
$$\\
\noindent\makebox[\linewidth]{\rule{\textwidth}{0.4pt}}
$$Var(X) = E(X^2) -E(X) \rightarrow E(X^2) = Var(X) +E(X)  = \frac{1}{\lambda^2} + \frac{1}{\lambda}  $$



\subsection*{c) diskrete  Gleichverteilung}
$$
E(X) = \sum_i x_i * P(X=x_i) 
$$
n sei die Mächtigkeit der Ergebnismenge($\Omega = \{1, ... , n\} $) bzw. die Menge aller $x_i$
$$
=\sum_i x_i * \frac{1}{n} \\
= \frac{1}{n} * \sum_i x_i  \\
= \frac{1}{n} * \frac{n * (n+1)}{2} \\
=\frac{n+1}{2}
$$ \\
\noindent\makebox[\linewidth]{\rule{\textwidth}{0.4pt}}
$$ E(X^2) = \frac{1}{n} P(X^2 =x^2 ) = \frac{1}{n} \sum_1^n x_i^2 = 
\frac{n (n+1) (2n+1)}{6n} = \frac{(n+1)(2n+1)}{6}
$$

\subsection{d) stetige Gleichverteilung}
$$
E(X) = \int_{-\infty}^\infty x f(x) = \int_a^b x \frac{1}{b-a}\\$$
Da f(x) außerhalb [a,b] 0 ergbt.
$$
=\frac{1}{b-a} [\frac{1}{2} x^2]_a^b = \frac{b^2-a^2}{2 (b-a)} = \frac{(b-a) (b+a)}{2(b-a)} = \frac{b+a}{2}
$$
\noindent\makebox[\linewidth]{\rule{\textwidth}{0.4pt}}
\\$$E(X^2)=\int_{-\infty}^\infty x^2 f(x) = \frac{1}{b-a} \int_{a}^{b}x^2 = \frac{1}{b-a} [\frac{1}{3} x^3]^b_a = \frac{b^3 - a^3}{3(b-a)} = \frac{1}{3}(a^2 + ab + b^2) 
$$
\end{document}
