% !TEX TS-program = pdflatex
% !TEX encoding = UTF-8 Unicode

% This is a simple template for a LaTeX document using the "article" class.
% See "book", "report", "letter" for other types of document.

\documentclass[11pt]{article} % use larger type; default would be 10pt

\usepackage[utf8]{inputenc} % set input encoding (not needed with XeLaTeX)
\usepackage{color}

%%% Examples of Article customizations
% These packages are optional, depending whether you want the features they provide.
% See the LaTeX Companion or other references for full information.

%%% PAGE DIMENSIONS
\usepackage{geometry} % to change the page dimensions
\geometry{a4paper} % or letterpaper (US) or a5paper or....
% \geometry{margin=2in} % for example, change the margins to 2 inches all round
% \geometry{landscape} % set up the page for landscape
%   read geometry.pdf for detailed page layout information

\usepackage{graphicx} % support the \includegraphics command and options

% \usepackage[parfill]{parskip} % Activate to begin paragraphs with an empty line rather than an indent

%%% PACKAGES
\usepackage{booktabs} % for much better looking tables
\usepackage{array} % for better arrays (eg matrices) in maths
\usepackage{paralist} % very flexible & customisable lists (eg. enumerate/itemize, etc.)
\usepackage{verbatim} % adds environment for commenting out blocks of text & for better verbatim
\usepackage{subfig} % make it possible to include more than one captioned figure/table in a single float
% These packages are all incorporated in the memoir class to one degree or another...

%%% HEADERS & FOOTERS
\usepackage{fancyhdr} % This should be set AFTER setting up the page geometry
\pagestyle{fancy} % options: empty , plain , fancy
\renewcommand{\headrulewidth}{0pt} % customise the layout...
\lhead{}\chead{}\rhead{}
\lfoot{}\cfoot{\thepage}\rfoot{}

%%% SECTION TITLE APPEARANCE
\usepackage{sectsty}
%\allsectionsfont{\sffamily\mdseries\upshape} % (See the fntguide.pdf for font help)
% (This matches ConTeXt defaults)

%%% ToC (table of contents) APPEARANCE
\usepackage[nottoc,notlof,notlot]{tocbibind} % Put the bibliography in the ToC
\usepackage[titles,subfigure]{tocloft} % Alter the style of the Table of Contents
\renewcommand{\cftsecfont}{\rmfamily\mdseries\upshape}
\renewcommand{\cftsecpagefont}{\rmfamily\mdseries\upshape} % No bold!

%%% END Article customizations

%%% The "real" document content comes below...

\title{\color{blue}DBS Übung 4}
\author{Tobias Reincke 218203884 \\
	Andreas Kübrich ... \\
              ....                    ....}
%\date{} % Activate to display a given date or no date (if empty),
         % otherwise the current date is printed 





























\begin{document}
\maketitle
\section*{ \textbf{Aufgabe 1: Dekomposition}}

\subsection*{(a)} R= (ABCD,{ABC})
\subsection*{(b)} Universalschlüssel: A \\
Universalschema: (ABCD,\{Ad\})\\                                                                                                                                        
Transitive Abhängigkeiten: A $\rightarrow$ B $\rightarrow$ C (A zu C ), A $\rightarrow$ B $\rightarrow$ D (A zu D)\\ 
Schritt 1:\{(ABCD,\{A\})\} zu \{(ABD, \{A\}),(BC,\{B\}) \} $|$ A zu C \\
Schritt 2: \{ABD,\{A\}\} zu  \{(AB,\{A\}),(BD,\{B\})\} $|$ A zu D \\
Vereinigung: \{(AB,\{A\}),(BD,\{B\},(BC,\{B\}) \}

\section*{\textbf{Aufgabe 2: }}
C ist bei allen gleich, kann kein Schlüssel sein.
Schlüsselkombinationen können sein: 
ABCD, ABD, AB, AC, AD,  BCD, A sind alle eindeutig. \\
Damit ist A die minimale identifizierende Attriutmenge.


\section*{\textbf{Aufgabe 3:}}
create  database db; 
use db;
create table Mannschaft ( Mannschaftsname varchar(50) not null, primary key mannschaftsname);\\
create table Spieler(Name varchar(50) not null,  Rückennummer integer(2), primary key (Name) );
create table Vertrag(Gehalt double not null , Name varchar(50) not null, Mannschaftsname varchar(50) not null, primary key(Name), foreign key (Name) references Spieler (Name), foreign key (Mannschaftsname) references Mannschaft(Mannschaftsname) );

create table Torwart(Name varchar (50) not null , Elfmeter integer (4) not null, foreign key (Name) references Spieler(Name),  primary key (Name) );

\textit{That should be about it.}

\section*{\textbf{Aufgabe 4:}}
\subsection*{(a)}
* ist das Wildcardsymbol und wählt alle aus.\\
SELECT [DISTINCT] ( * $|$ attribute 1 , ..., attribute N )
\\
FROM table\_reference
\\
\textit{ab hier alle optional}\\
  WHERE [where\_condition]  \textit{\small Bedingung}
\\
GROUP BY [attribute] \textit{\small Gruppierung} \\ 
ORDER BY attribute [ASC $|$ DESC  ] \textit{\small Sortierung nach Attribut}\\
\subsection*{(b)}

\end{document}
