\documentclass[]{article}
\usepackage{amsmath}
\usepackage{amssymb}

\usepackage{graphicx}
%opening
\title{Mathematik für Informatiker - Serie 4 }
\author{Tobias Reincke \\ Matrikelnummer: 218203884}

\begin{document}

\maketitle
\section{Aufgabe 1}
\begin{gather*}
\textit{Assoziativität} \\
a \circ b := a +b -a*b \\ \\
(a \circ b) \circ c = a \circ ( b \circ c ) \\
( a+b - a*b ) + c  - ( a+b - a*b )*c = a + (b + c -b*c  )  -  a* (b + c -b*c  )\\
( a+b - a*b ) + c -  a *c - b*c + a*b*c =  a + (b + c -b*c  )  - a*b -a*c + a*b*c\\
a + b + c - a*b -  a *c - b*c + a*b*c  = -  a +b+c -a*b-b*c -a*c + a*b*c\\
\text{Bewiesen!}\\
\text { Neutrales Element e: }\\
a \circ e = a \\
\rightarrow a + e -a*e = a  \\ 
\textit{ Beide Seiten   $-a $}\\
\rightarrow e = a* e \\
\rightarrow e = 0\\ \\
\text{ Inverse Elemente }\\
a \circ a_{invers} = a + a_{invers} - a *a_ {invers} = 0 \\
\rightarrow a  + a_{invers} = a*a_{invers}\\
\rightarrow \text{Menge der inversierbaren Elemente =} \{ x \| \exists y : x + y =x*y \}\\
\\
\text{Beispiele sind (2,2),(0,0) (-1, $\frac{1}{2}$ ) }\\
 \text{Es ist keine Gruppe, da nicht alle Elemente invertierbar sind , z.b Beispiel 1.}\\
  \text{  Das wäre Lösung der Gleichung 1*x = 1+x, was keine Lösung hat. }
\end{gather*}
\section{Aufgabe 2}
\begin{tabular}{|c|c|c|c|c|}
	\hline 
	*  & [1] & [5] & [7] & [11] \\ 
	\hline 
	[1] & 1 & 11 & 7 & 11 \\ 
	\hline 
	[5] & 5 & 1 & 11 & 7 \\ 
	\hline 
	[7]& 7 & 11 & 1 & 5 \\ 
	\hline 
	[11] & 11 & 7 & 5 & 1  \\ 
	\hline 
\end{tabular} 
$\rightarrow$ [5] ist mit sich selbst 1, d.h Es ist gleichzeitig das inverse zu sich selbst.
\section{Aufgabe 3}
\begin{gather*}
\text {Assoziativität gegegeben}\\
z.z: \exists \text{ Neutrales Element, Inverses Element }\\
		\text{Das neutrale Element ist für jede Matrix gegeben: Die Einheitsmatrix.}\\
	 \left( \begin{matrix}
		1 & 0 \\ 
		0 & 1
		\end{matrix} \right) \\
	\text{	Das Inverse einer Matrix M  ist durch folgende Formel gegeben: }\\
	\frac{1}{|M|}*Adjunktenmatrix(M)\\
	\text{ In diesem Falle :}\\
	\frac{1}{a*c} * \left(\ \begin{matrix}
	c & -b \\ 
	0 & a
	\end{matrix} \right) = \left(\begin{array}{cc}
	\frac{1}{a} & \frac{-b}{ac} \\ 
	0 & \frac{1}{c}
	\end{array} \right) \\
		\text{Da a,c $\neq$ 0 funktioniert das. :)}
\end{gather*}
\newpage
\section{Aufgabe 4}
Da 7 prim ist, sind die primen Restklassen von p alles Zahlen kleiner 7, da sie alle keine Teiler teilen.
$P_7 =\{[1]_7[2]_7[3]_7[4]_7[5]_7[6]_7\}$ \\
Wegen Modulogesetzen gilt, dass $\forall$[x]:  [x]*[1]=[x], [1] ist ein neutrales Element dieser Gruppe. \\ 
Hier die Tabelle: siehe Figure 1 \\ Daraus lässt sich schließen, dass [1] das einzige neutrale Element dieser Gruppe ist. Es muss dementsprechend Element jeder Untergruppe sein,\\ sonst würde der Untergruppe das neutrale Element fehlen, und sie wäre dementsprechend keine Gruppe.\\
Für eine Teilmenge mit der Mächtigkeit 3 benötigt man aber noch 2 weitere Elemente. [1]*[1] ist mit sich selbst invers. Also muss man diese so wählen, dass diese gegenseitig invers sind, da [1] zu keinem anderem Element invers ist. Dafür kommen die Paare ([3],[5]) und ([2][4]) in Frage. ($5*3 \mod 7 = 1 \land 4*2 \mod 7 = 1 $). [5] und [3] als Paar fliegen raus, weil [5]*[5] = [4], also nicht Element der Gruppe ist.  
Also wähle ich meine Untergruppe:\\ $U_7 = (\{ [1],[2],[4] \}, *) $ \\
\textit{[6] ist auch invers, aber mit sich selbst, also nur ein Element. }

\begin{figure}

	\includegraphics[width=0.7\linewidth]{verknüpfungstabelle}
	\caption{Mit einen kleinen c++ Programm ausgegeben.}
	\label{fig:verknupfungstabelle}
\end{figure}

\newpage

\section{Aufgabe 5}
\subsection*{a}
\begin{gather*}
a,b \in \mathbb{Z } \\
f_1(a \circ b) = f_1(a) \circ f(b) \\
f_1 (a+b) = 1 \circ 1 \\
1 = 1+1 \\
\rightarrow f_1 \text{ist nicht  homomorph.}
\end{gather*}
\subsection{b}
\begin{gather*}
a,b \in \mathbb{R}\\
f_2(a \circ b) = f_2(a) \circ f_2(b)\\
f_2(a+b) = 2*a \circ 2*b \\
2*(a+b) = 2*a + 2*b\\
\rightarrow f_2\text{ ist homomorph.}
\end{gather*}
\subsection{c}
\begin{gather*}
 a,b \in \mathbb{R}\\
 f_3(a \circ b) = f_3(a) \circ f_3(b)\\
	 \text{Seien die Nachkommastellen von a,b größer gleich 1,}\\\text{ dann ist a+b abegerundet 1 größer als a abgerundet 
 	+ b abgerundet.}\\
 \rightarrow f_3 \text{ ist nicht homomorph.}
 \end{gather*}





\end{document}
