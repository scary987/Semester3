% !TEX TS-program = pdflatex
% !TEX encoding = UTF-8 Unicode

% This is a simple template for a LaTeX document using the "article" class.
% See "book", "report", "letter" for other types of document.

\documentclass[11pt]{article} % use larger type; default would be 10pt

\usepackage[utf8]{inputenc} % set input encoding (not needed with XeLaTeX)

%%% Examples of Article customizations
% These packages are optional, depending whether you want the features they provide.
% See the LaTeX Companion or other references for full information.

%%% PAGE DIMENSIONS
\usepackage{geometry} % to change the page dimensions
\geometry{a4paper} % or letterpaper (US) or a5paper or....
 \geometry{margin=1in} % for example, change the margins to 2 inches all round
% \geometry{landscape} % set up the page for landscape
%   read geometry.pdf for detailed page layout information

\usepackage{graphicx} % support the \includegraphics command and options

% \usepackage[parfill]{parskip} % Activate to begin paragraphs with an empty line rather than an indent

%%% PACKAGES
\usepackage{booktabs} % for much better looking tables
\usepackage{array} % for better arrays (eg matrices) in maths
\usepackage{paralist} % very flexible & customisable lists (eg. enumerate/itemize, etc.)
\usepackage{verbatim} % adds environment for commenting out blocks of text & for better verbatim
\usepackage{subfig} % make it possible to include more than one captioned figure/table in a single float
% These packages are all incorporated in the memoir class to one degree or another...

%%% HEADERS & FOOTERS
\usepackage{fancyhdr} % This should be set AFTER setting up the page geometry
\pagestyle{plain} % options: empty , plain , fancy
\renewcommand{\headrulewidth}{0pt} % customise the layout...
\lhead{}\chead{}\rhead{}
\lfoot{}\cfoot{\thepage}\rfoot{}

%%% SECTION TITLE APPEARANCE
\usepackage{sectsty}
\allsectionsfont{\sffamily\mdseries\upshape} % (See the fntguide.pdf for font help)
% (This matches ConTeXt defaults)

%%% ToC (table of contents) APPEARANCE
\usepackage[nottoc,notlof,notlot]{tocbibind} % Put the bibliography in the ToC
\usepackage[titles,subfigure]{tocloft} % Alter the style of the Table of Contents
\renewcommand{\cftsecfont}{\rmfamily\mdseries\upshape}
\renewcommand{\cftsecpagefont}{\rmfamily\mdseries\upshape} % No bold!

%%% END Article customizations

%%% The "real" document content comes below...

\title{A02 Imperative Programmierung}
%\date{} % Activate to display a given date or no date (if empty),
         % otherwise the current date is printed 

\begin{document}
\maketitle

\section{Authoren}
Schümann, Hauke. Matrikelnr: 219203901\\
Hoang Anh, Pham. Matrikelnr: 219204706\\
Muhannad Albakkar. Matrikelnr: 218202908

\section{Aufgabe 1}

In C dürfen nur Buchstaben (groß oder klein), Ziffern und Unterspriche benutzt werden.\\
Das erste Zeichen darf keine Ziffer sein. Es gibt Einschränkung für die Länge, mehr dazu in Aufgabe 2.\\
$foobar$ ist ein erlaubter Name, da er die eben genannten Einschränkungen nicht verletzt.\\
$foo\&bar$ ist kein erlaubter Name, da ein '$\&$' vorkomment. '$\&$' ist weder Buchstabe, noch Ziffer, noch Untersprich. Es ist stattdessen sogar in unärer Operator.

\section{Aufgabe 2}

Im ANSI-Standard kann ein interner Bezeichner (Variablenname in Funktion) mit 31 Zeichen (6 Zeichen für externe Bezeichner) unterschieden werden.\\

\section{Aufgabe 5}

Das Programm gibt die Reste der Divisionen von $x$ durch die Element der Reihe $(w^{k+1})_{k=n\rightarrow0}$ mit $n$ als die kleinste natürliche Zahl, für die gilt $w^{n+2}\ge x$\\
\\
Algorithmus:\\
  1. Eingabe von zwei natürliche Zahlen in die Variablen $w$ und $x$.\\
  2. Sei die Variable $y=1$.\\
  3. $y=y*w$.\\
  4. Wenn $y > x$, dann weiter mit 5, sonst weiter mit 3.\\
  5. $y=y/w$.\\
  6. $z=x/y$.\\
  7. Zeige den Wert von $z$.\\
  8. $x=x-y*z$.\\
  9. $y=y/w$.\\
 10. Wenn $y=0$, dann weiter mit 11, sonst weiter mit 6.\\
 11. Ende.\\

\end{document}
