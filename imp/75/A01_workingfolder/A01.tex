% !TEX TS-program = pdflatex
% !TEX encoding = UTF-8 Unicode

% This is a simple template for a LaTeX document using the "article" class.
% See "book", "report", "letter" for other types of document.

\documentclass[11pt]{article} % use larger type; default would be 10pt

\usepackage[utf8]{inputenc} % set input encoding (not needed with XeLaTeX)

%%% Examples of Article customizations
% These packages are optional, depending whether you want the features they provide.
% See the LaTeX Companion or other references for full information.

%%% PAGE DIMENSIONS
\usepackage{geometry} % to change the page dimensions
\geometry{a4paper} % or letterpaper (US) or a5paper or....
\geometry{margin=1in} % for example, change the margins to 2 inches all round
% \geometry{landscape} % set up the page for landscape
%   read geometry.pdf for detailed page layout information

\usepackage{graphicx} % support the \includegraphics command and options

% \usepackage[parfill]{parskip} % Activate to begin paragraphs with an empty line rather than an indent

%%% PACKAGES
\usepackage{booktabs} % for much better looking tables
\usepackage{array} % for better arrays (eg matrices) in maths
\usepackage{paralist} % very flexible & customisable lists (eg. enumerate/itemize, etc.)
\usepackage{verbatim} % adds environment for commenting out blocks of text & for better verbatim
\usepackage{subfig} % make it possible to include more than one captioned figure/table in a single float
\usepackage{fancybox}
\usepackage{struktex}
\input{syntax.tex} 
% These packages are all incorporated in the memoir class to one degree or another...

%%% HEADERS & FOOTERS
\usepackage{fancyhdr} % This should be set AFTER setting up the page geometry
\pagestyle{fancy} % options: empty , plain , fancy
\renewcommand{\headrulewidth}{0pt} % customise the layout...
\lhead{}\chead{}\rhead{}
\lfoot{}\cfoot{\thepage}\rfoot{}

%%% SECTION TITLE APPEARANCE
\usepackage{sectsty}
\allsectionsfont{\sffamily\mdseries\upshape} % (See the fntguide.pdf for font help)
% (This matches ConTeXt defaults)

%%% ToC (table of contents) APPEARANCE
\usepackage[nottoc,notlof,notlot]{tocbibind} % Put the bibliography in the ToC
\usepackage[titles,subfigure]{tocloft} % Alter the style of the Table of Contents
\renewcommand{\cftsecfont}{\rmfamily\mdseries\upshape}
\renewcommand{\cftsecpagefont}{\rmfamily\mdseries\upshape} % No bold!

%%% END Article customizations

%%% The "real" document content comes below...

\title{Imperative Programmierung, Aufgaben A01}
\author{Hauke Schümann, Martikelnr.: 219203901}
%\date{} % Activate to display a given date or no date (if empty),
         % otherwise the current date is printed 

\begin{document}
\maketitle

\section{Aufgabe 1: Turing Maschine}

\subsection{Teilaufgabe (a)}
\begin{large}
\begin{tabular}{|l|c|r|}
\hline
Zustand & Band & Zeigerposition\\\hline
$S_1$ & 11000 & 1\\\hline
$S_2$ & 01000 & 2\\\hline
$S_2$ & 01000 & 3\\\hline
$S_3$ & 01000 & 4\\\hline
$S_4$ & 01010 & 3\\\hline
$S_5$ & 01010 & 2\\\hline
$S_5$ & 01010 & 1\\\hline
$S_1$ & 11010 & 2\\\hline
$S_2$ & 10010 & 3\\\hline
$S_3$ & 10010 & 4\\\hline
$S_3$ & 10010 & 5\\\hline
$S_4$ & 10011 & 4\\\hline
$S_4$ & 10011 & 3\\\hline
$S_5$ & 10011 & 2\\\hline
$S_1$ & 11011 & 3\\\hline
$S_6$ & 11011 & 3\\\hline
\end{tabular}
\end{large}
\\\\
Man endet mit dem Band $11011$.

\subsection{Teilaufgabe (b)}

Die beschriebene Machine ist eine die im ersten Schritt terminiert.\\
\\
$\sum = \{s_1\}, \\
 A = \{0,1\}, \\
\sigma = s_1, \\
F = \{s_1\}, \\
\delta = \emptyset$

\newpage

\section{Aufgabe 2: Syntax}

Wir beginnen beim Startsymbol \textit{Zuweisung} und bestimmen alle mögliche Ableitungen. \\

\begin{large}
\begin{tabular}{|l|l|r|}
\hline
&Ausdruck&Nächste Regel \\\hline
1.&Zuweisung&Regel 1\\
2.&Variable ":=" Ausdruck&Regel 4\\
2.&"a :=" Ausdruck&Regel 5\\
3.&"a:=" Ausdruck "+" Ausdruck&Regel 5 oder Regel 4\\\hline

Fall 1.&&Anwendung von Regel 5\\
4.&"a := a +" Ausdruck&Regel 4\\
5.&"a := a +" Ausdruck "+" Ausdruck&Regel 6\\
6.&"a := a + b +" Ausdruck&Regel 7\\
7.&"a := a + b + 1"&\\\hline

Fall 2.&&Anwendung von Regel 4\\
4.&"a :=" Ausdruck "+" Ausdruck "+" Ausdruck&Regel 5\\
5.&"a := a +" Ausdruck "+" Ausdruck&Regel 6\\
6.&"a := a + b +" Ausdruck&Regel 7\\
7.&"a := a + b + 1"&\\\hline
\end{tabular}
\end{large}
\\\\
Es gibt zwei mögliche Ableitung. \\

\section{Aufgabe 3: Nur ein Weg nach Rom.}

Wir nennen den in der Aufgabe geforderte Grammatik $G_1 =(T_1, N_1, P_1, S_1)$ mit \\
\\
$T_1 = \{"a", "b", ":=", "+", "1", "0"\}\\
N_1 = \{Zuweisung, Variable, Ausdruck, Konstante, Wert\}\\
P_1 =	\{\\
	Zuweisung =_1 Variable ":=" Ausdruck,\\
	Ausdruck =_2 Wert,\\
	Ausdruck =_3 Wert "+" Ausdruck,\\
	Wert =_4 Konstante, \\
	Wert =_5 Variable, \\
	Konstante =_6 "0",\\
	Konstante =_7 "1",\\
	Variable =_8 "a",\\
	Variable =_9 "b"\\
\}$

\newpage

\section{Aufgabe 4}

\subsection{Syntax Diagramme}

Zuweisung\\
\begin{Syntaxdiagramm}%
	\SynVarR{Variable}%
	\TerSymR{":="}%
	\SynVarR{Ausdruck}%
\end{Syntaxdiagramm}
\\\\
Ausdruck\\
\begin{Syntaxdiagramm}%
	\begin{Alternativen}
	{\SynVarR{Variable}}%
	{\SynVarR{Ausdruck}\TerSymR{"+"}\SynVarR{Ausdruck}}%
	\SynVarR{Konstante}\\%
	\end{Alternativen}
\end{Syntaxdiagramm}
\\\\
Variable\\
\begin{Syntaxdiagramm}%
	\begin{Alternativen}
	{\TerSymR{"b"}}%
	{\TerSymR{"a"}}%
	\end{Alternativen}
\end{Syntaxdiagramm}
\\\\
Konstante\\
\begin{Syntaxdiagramm}%
	\begin{Alternativen}
	{\TerSymR{"1"}}%
	{\TerSymR{"0"}}%
	\end{Alternativen}
\end{Syntaxdiagramm}

\subsection{EBNF}

$Zuweisung$ = $Variable$ ":=" $Ausdruck$.\\
$Ausdruck$ = ($Variable|Konstante$)["+" $Ausdruck$].\\
$Variable$ = ("a"$|$"b").\\
$Konstante$ = ("1"$|$"0").\\

\newpage

\section{Aufgabe 5}

\subsection{EBNF}

$Adresse$ = $Text$ ";" ($Text$ " " $Zahl$|"Postfach "$Zahl$) ";" $Text$ ";" $Zahl$ ";" ["Tel "$Zahl$ ";"]["Fax "$Zahl$ ";"]["Email "$Email$].\\
$Email$ = $Text$"@"$Text$"."$Text$.\\
$Text$ = $Buchstabe$\{$Buchstabe$\}\\
$Zahl$ = $Ziffer$\{$Ziffer$\}.\\
$Ziffer$ = (1$|$2$|$3$|$4$|$5$|$6$|$7$|$8$|$9$|$0)\\
$Buchstabe$ = ('a'$|$'b'$|$...$|$'A'$|$...$|$'ä'$|$...$|$'  '$|$...$|Ziffer$)
\\\\
Es gibt eine immer noch eine große Menge unzulässiger Kombinationen die man hiermit erschaffen könnte aber ich möchte hier nicht 30 Stunden sitzen.\\
Zum Beispiel ist "2a" eine zulässige Hausnummer, der Einfachheit halber ist das jedoch nicht repräsentiert.

\subsection{Syntaxdiagramme}

$Adresse$\\
\begin{Syntaxdiagramm}%
	\SynVarR{Name}
	\SynVarR{Anschrift}
	\SynVarR{Stadt}
	\SynVarR{PLZ}
	\SynVarR{Optionales}
\end{Syntaxdiagramm}
\\
$Name$\\
\begin{Syntaxdiagramm}%
	\SynVarR{Text}
	\TerSymR{";"}
\end{Syntaxdiagramm}
\\
$Anschrift$\\
\begin{Syntaxdiagramm}%
	\Alternative
		{\SynVarR{Text}
		\TerSymR{" "}
		\SynVarR{Zahl}
		}
		{\TerSymR{"Postfach "}
		\SynVarR{Zahl}
		}
	\TerSymR{";"}
\end{Syntaxdiagramm}
\\
$Stadt$\\
\begin{Syntaxdiagramm}%
	\SynVarR{Text}
	\TerSymR{";"}
\end{Syntaxdiagramm}
\\
$PLZ$\\
\begin{Syntaxdiagramm}%
	\SynVarR{Zahl}
	\TerSymR{";"}
\end{Syntaxdiagramm}
\\
$Optionales$\\
\begin{Syntaxdiagramm}%
	\Alternative
	{\SynVarR{Zahl}\TerSymR{";"}}
	{\PfeilR}
	\Alternative
	{\SynVarR{Zahl}\TerSymR{";"}}
	{\PfeilR}
	\Alternative
	{\SynVarR{Email}\TerSymR{";"}}
	{\PfeilR}
	\TerSymR{";"}
\end{Syntaxdiagramm}
\\
$Email$\\
\begin{Syntaxdiagramm}%
	\TerSymR{"Email "}
	\SynVarR{Text}
	\TerSymR{"@"}
	\SynVarR{Text}
	\TerSymR{"."}
	\SynVarR{Text}
\end{Syntaxdiagramm}

\newpage

\section{Aufgabe 6}

\subsection{Algorithmusprüfung}
Prüfen ob die Formal ein Algorithmus ist:

Operative Finitheit.
Nicht gegeben, Formel läuft endlos weiter.

Effektivität.
Es sind keine unrealisierbaren Elementaroperationen gefragt.

Vollständigkeit.
Formel is vollständig. Es kann stets angegeben werden welche Operation durchgeführt werden soll.

Finitheit.
Die Formel kann endlich dargestellt werden.

Dynamische Finitheit.
Ergebnis kann irrational sein. Daher nicht gegeben.

\subsection{Struktogramm}

\sProofOn
\begin{struktogramm}(95,100)
	\assign{parameter $\leftarrow$ input}
	\assign{iterations $\leftarrow$ input}
	\assign{loopindex = 0}
	\assign{sum = 0}
	\while[8]{loopindex $<$ iterations}
		\assign{secindex = 0}
		\assign{
		npot = 1\\
		onepot = 1\\
		nfac = 1\\
		}
		\while[7]{secindex $<$ loopindex}
			\assign{secloopindex$++$}
			\assign{onepot $*=$ -1}
			\assign{npot $*=$ parameter $*$ parameter}
			\assign{nfac $*=$ loopindex $*$ 2}
			\assign{nfac $*=$ loopindex $*$ 2 $+$ 1}
		\whileend
		\assign{npot $*=$ parameter}
		\assign{sum $+=$ onepot $*$ npot $/$ nfac}
		\assign{loopindex$++$}
	\whileend
	\exit[0]{output $\leftarrow$ sum}
\end{struktogramm}
\sProofOff
\\
Kann man das optimieren? Klar aber warum sollte ich? \\

\end{document}
