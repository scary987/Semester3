% !TeX spellcheck = de_DE
\documentclass[]{article}

\usepackage[utf8]{inputenc}
\usepackage[T1]{fontenc}
\usepackage[scaled]{beramono}
\usepackage{amssymb}
\usepackage{amsmath}
\usepackage{graphicx}
\usepackage[top=2cm, bottom=2cm, left=2cm, right=4cm]{geometry}
\usepackage{csquotes}
\usepackage{listings}
\usepackage{hyperref}
\usepackage{booktabs}

\lstset{
	tabsize=2,
	language=Java,
	showstringspaces=false,
	basicstyle=\footnotesize\ttfamily,
}

%no indentation
\setlength\parindent{0pt}

\pagestyle{empty}

\begin{document}	
	
	\section{Vorbereitung}
	
	\subsection{Anmelden}
	
	Melden Sie sich am Laborrechner mit Ihrem ITMZ-Account an.
	
	\subsection{Registrieren im Stud.IP}
	
	\begin{enumerate}
		\item Rufen Sie in einem Browser Ihrer Wahl (außer Internet Explorer) \href{https://studip.uni-rostock.de}{studip.uni-rostock.de} auf.
		\item Melden Sie sich mit Ihrem ITMZ-Account an.
		\item Tragen Sie sich in die Veranstaltungen zur imperativen Programmierung ein:
		\begin{itemize}
		\item Modulnummer 23001 (Vorlesung) 
		\item Modulnummer 230011 (Übung für Informatiker - \texttt{handout01.pdf} herunterladen und öffnen)
		\end{itemize}
		\item Öffnen Sie in der Übungsveranstaltung den Reiter \enquote{Lernmodule}.
		\item Absolvieren Sie den Test \enquote{01 Fragen zu Programmierfertigkeiten}.
	\end{enumerate}
	
	\subsection{Dateierweiterungen in Windows anzeigen}
	
	\begin{enumerate}
		\item Öffnen Sie den Windows Explorer.
		\item Öffnen Sie in der Menüleiste den Reiter \enquote{Ansicht}.
		\item Setzen Sie das Häkchen bei \enquote{Dateinamenerweiterungen}
		\item Bestätigen Sie mit OK.
	\end{enumerate}
	
	\section{Hello World}
	
	\subsection{Arbeitsverzeichnis anlegen}
		
		\begin{enumerate}
				\item Öffnen Sie das Laufwerk \texttt{R:} im Windows Explorer.
				\item Legen Sie einen Ordner namens \texttt{ip2019} und darin einen Ordner \texttt{helloworld} an.
		\end{enumerate}
	
	\subsection{Ein Programm schreiben}
	
	\begin{enumerate}
		\item Notepad++ starten - das ist unser Editor.
		\item Folgenden Quellcode eintippen:
		\begin{lstlisting}[gobble=6]
			#include <stdio.h>
			
			main() {
				printf("Hello World!\n");
			}
		\end{lstlisting}
		\item Speichern Sie die Datei unter \verb|R:\ip2019\helloworld\helloworld.c|
	\end{enumerate}
	
	\subsection{Ein Programm kompilieren und ausführen}
	
	\begin{enumerate}
		\item Starten Sie die Anwendung \enquote{MinGW Shell}.
		\item Ins Arbeitsverzeichnis wechseln: \verb|cd ip2019/helloworld|
		\item Das Programm kompilieren: \verb|gcc helloworld.c -o helloworld.exe|
		\item Das Programm ausführen: \verb|helloworld.exe|
	\end{enumerate}
	
	\section{Programmieren in C}
	
	\begin{enumerate}
		\item Modifizieren Sie Ihr Programm so, dass \enquote{Hello World} vertikal ausgegeben wird.
		\item Versuchen Sie, Fehler in Ihren Quelltest einzubauen (z.B. vergessenes Semikolon/Klammer, Groß- und Kleinschreibung, \dots). Lässt sich das Programm noch kompilieren? Verstehen Sie die Fehlermeldungen, die \texttt{gcc} liefert?
		\item Kompilieren Sie Ihr Programm mit dem zusätzlichen Kommandozeilenparametern \verb|-Wall| und \verb|-pedantic|, wodurch der Compiler den Quellcode strenger prüft. Verstehen Sie die Compilermeldungen? Können Sie den Quelltest so modifizieren, dass er warnungsfrei kompiliert werden kann?
	\end{enumerate}
	
	%Fehler einbauen
	%Hello World vertikal
	
\end{document}