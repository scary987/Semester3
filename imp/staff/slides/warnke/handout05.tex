% !TeX spellcheck = de_DE
\documentclass[]{article}

\usepackage[utf8]{inputenc}
\usepackage[T1]{fontenc}
\usepackage[scaled]{beramono}
\usepackage{amssymb}
\usepackage{amsmath}
\usepackage{graphicx}
\usepackage[top=2cm, bottom=2cm, left=2cm, right=4cm]{geometry}
\usepackage{csquotes}
\usepackage{listings}
\usepackage{hyperref}
\usepackage{booktabs}

\lstset{
	tabsize=2,
	language=Java,
	showstringspaces=false,
	basicstyle=\footnotesize\ttfamily,
}

%no indentation
\setlength\parindent{0pt}

\pagestyle{empty}

\lstset{breaklines=true,frame=single}

\begin{document}
	
	\section{Fehlersuche}
	
	Welches der folgenden Programme kompiliert und arbeitet fehlerfrei? Wo liegen Fehler?
	\vspace{3ex}
	
	\noindent
	\begin{minipage}{0.5\linewidth} %1
		\begin{lstlisting}[gobble=6]
			#include <studio.h>
			
			int input;
			
			main() {
			
				scanf("%d", &input); /* Nutzereingabe */
				output = input + 1;
				printf("Der Nachfolger von %d ist %d.\n", input, output);
			}
		\end{lstlisting}
	\end{minipage}
	\quad
	\begin{minipage}{0.5\linewidth} %2
		\begin{lstlisting}[gobble=6]
			#include <stdio.h>
			
			int input;
			
			main() {
			
				scanf("d", &input); /* Nutzereingabe */
				
				printf("Der Nachfolger von %d ist %d.\n", input, input+1);
			}
		\end{lstlisting}
	\end{minipage}
	%
	\begin{minipage}{0.5\linewidth} %3
		\begin{lstlisting}[gobble=6]
			#include <stdio.h>
			
			int input;
			
			main() {
			
				scanf("%d", &input); /* Nutzereingabe */
				int output = ++input;
				printf("Der Nachfolger von %d ist %d.\n", input, output);
			}
		\end{lstlisting}
	\end{minipage}
	\quad
	\begin{minipage}{0.5\linewidth} %4
		\begin{lstlisting}[gobble=6]
			#include <stdio.h>
			
			int input;
			
			main() {
			
				scanf("%d", input); /* Nutzereingabe */
				
				printf("Der Nachfolger von %d ist %d.\n, input, input+1);
			}
		\end{lstlisting}
	\end{minipage}
	%
	\begin{minipage}{0.5\linewidth} %5
		\begin{lstlisting}[gobble=6]
			#include <stdio.h>
			
			int input;
			
			main() {
			
			scanf("%d", &input); /* Nutzereingabe */
			int output++ = input;
			printf("Der Nachfolger von %d ist %d.\n", input, output);
			}
		\end{lstlisting}
	\end{minipage}
	\quad
	\begin{minipage}{0.5\linewidth} %6
		\begin{lstlisting}[gobble=6]
			#include <stdio.h>
			
			int input;
			
			main() {
			
				scanf("%d", &input); /* Nutzereingabe */
				int output = 1 + input;
				print("Der Nachfolger von %d ist %d.\n"; input, output);
			}
		\end{lstlisting}
	\end{minipage}
	%
	\begin{minipage}{0.5\linewidth} %7
		\begin{lstlisting}[gobble=6]
			#include <stdio.h>
			
			int input;
			
			main() {
			
				scanf("%d", &input); /* Nutzereingabe */
				int output = input + 1;
				printf("Der Nachfolger von %d ist %d.\n", input, output);
			}
		\end{lstlisting}
	\end{minipage}
	\quad
	\begin{minipage}{0.5\linewidth} %8
		\begin{lstlisting}[gobble=6]
			#include <stdio.h>
			
			int input;
			
			main() 
			
				scanf("%d", &input); /* Nutzereingabe */
				int output = 1 + input;
				printf("Der Nachfolger von %d ist %d.\n", input, output);
			}
		\end{lstlisting}
	\end{minipage}
	%
	\begin{minipage}{0.5\linewidth} %9
		\begin{lstlisting}[gobble=6]
			#include <stdio.h>
			
			int input;
			
			main() {
			
				scanf("%d", &input) /* Nutzereingabe */
				int output = input + 1;
				printf("Der Nachfolger von %d ist %d.\n", input, output);
			}
		\end{lstlisting}
	\end{minipage}
	\quad
	\begin{minipage}{0.5\linewidth} %10
		\begin{lstlisting}[gobble=6]
			#include <stdio.h>
			
			int input;
			
			main() {
				
				scanf("%d", &input); /* Nutzereingabe */
				int output = input;
				printf("Der Nachfolger von %d ist %d.\n", --input, ++input);
			}	
		\end{lstlisting}
	\end{minipage}
	
	\section{Feedback}
	
	Bearbeiten Sie den Fragebogen unter \href{https://evasys.uni-rostock.de/evasys/online.php?p=ip2017}{https://evasys.uni-rostock.de/evasys/online.php?p=ip2017}.
	
	\section{Zustandsdiagramme}
	
	Ein Programm liest eine Datei ein, die nur Buchstaben und Whitespaces enthält.
	Es soll die Worte zählen, die mit einem Großbuchstaben beginnen und sonst nur Kleinbuchstaben enthalten (und zwar mindestens einen).
	Zeichnen Sie ein Zustandsdiagramm, das das Verhalten des Programms beschreibt!
	
	\vspace{50ex}
	
	\section{Ein- und Ausgabe}
	
	Schreiben Sie ein Programm, das seine Standardeingabe in die Standardausgabe kopiert und dabei jede Folge von Leerzeichen und Tabulatorzeichen durch ein einzelnes Leerzeichen ersetzt.
	
	\section{Funktionen}
	
	\subsection{BMI}
	
	Schreiben Sie eine Funktion, die aus gegebenen Gewicht und Körpergröße den Body Mass Index (BMI) berechnet.
	
	\subsection{Primzahlen}
	
	\begin{enumerate}
		\item Schreiben Sie eine Funktion \lstinline|isPrime|, die für eine übergebene Zahl bestimmt, ob es sich um eine Primzahl handelt.
		\item Schreiben Sie mit Hilfe Ihrer Funktion \lstinline|isPrime| ein Programm, das alle Primzahlen bis zu einer vom Nutzer eingegebenen Grenze ermittelt und ausgibt.
		\item Überlegen Sie, ob Ihre Implementierung von \lstinline|isPrime| optimiert werden kann.
	\end{enumerate}
	
\end{document}