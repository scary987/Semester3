% !TeX spellcheck = de_DE
\documentclass[]{article}

\usepackage[utf8]{inputenc}
\usepackage[T1]{fontenc}
\usepackage[scaled]{beramono}
\usepackage{amssymb}
\usepackage{amsmath}
\usepackage{graphicx}
\usepackage[top=2cm, bottom=2cm, left=2cm, right=4cm]{geometry}
\usepackage{csquotes}
\usepackage{listings}
\usepackage{hyperref}
\usepackage{booktabs}

\lstset{
	tabsize=2,
	language=Java,
	showstringspaces=false,
	basicstyle=\footnotesize\ttfamily,
}

%no indentation
\setlength\parindent{0pt}

\pagestyle{empty}

\lstset{frame=single,breaklines=true}

\begin{document}

	\section{Fehlersuche}
	
	Welche der folgenden Programme kompilieren und arbeiten fehlerfrei? Wo liegen Fehler?
	\vspace{3ex}
	
	\noindent
	\begin{minipage}{0.5\linewidth} %1 output nicht deklariert
		\begin{lstlisting}[gobble=6]
			#include <stdio.h>
			
			int input;
			
			int main() {
			
				scanf("%d", &input); /* Nutzereingabe */
				output = input + 1;
				printf("Der Nachfolger von %d ist %d.\n", input, output);
				return 0;
			}
		\end{lstlisting}
	\end{minipage}
	\quad
	\begin{minipage}{0.5\linewidth} %2 % in scanf fehlt
		\begin{lstlisting}[gobble=6]
			#include <stdio.h>
			
			int input;
			
			int main() {
			
				scanf("d", &input); /* Nutzereingabe */
				
				printf("Der Nachfolger von %d ist %d.\n", input, input+1);
				return 0;
			}
		\end{lstlisting}
	\end{minipage}
	%
	\begin{minipage}{0.5\linewidth} %3 input wird vor der ausgabe inkrementiert
		\begin{lstlisting}[gobble=6]
			#include <stdio.h>
			
			int input, output;
			
			int main() {
			
				scanf("%d", &input); /* Nutzereingabe */
				output = ++input;
				printf("Der Nachfolger von %d ist %d.\n", input, output);
				return 0;
			}
		\end{lstlisting}
	\end{minipage}
	\quad
	\begin{minipage}{0.5\linewidth} %4 " fehlt in printf
		\begin{lstlisting}[gobble=6]
			#include <stdio.h>
			
			int input;
			
			int main() {
			
				scanf("%d", input); /* Nutzereingabe */
				
				printf("Der Nachfolger von %d ist %d.\n, input, input+1);
				return 0;
			}
		\end{lstlisting}
	\end{minipage}
	%
	\begin{minipage}{0.5\linewidth} %5 output++ ist kein lvalue
		\begin{lstlisting}[gobble=6]
			#include <stdio.h>
			
			int input, output;
			
			int main() {
			
				scanf("%d", &input); /* Nutzereingabe */
				output++ = input;
				printf("Der Nachfolger von %d ist %d.\n", input, output);
				return 0;
			}
		\end{lstlisting}
	\end{minipage}
	\quad
	\begin{minipage}{0.5\linewidth} %6 ; in printf
		\begin{lstlisting}[gobble=6]
			#include <stdio.h>
			
			int input, output;
			
			int main() {
			
				scanf("%d", &input); /* Nutzereingabe */
				output = 1 + input;
				printf("Der Nachfolger von %d ist %d.\n"; input, output);
				return 0;
			}
		\end{lstlisting}
	\end{minipage}
	%
	\begin{minipage}{0.5\linewidth} %7 richtig
		\begin{lstlisting}[gobble=6]
			#include <stdio.h>
			
			int input, output;
			
			int main() {
			
				scanf("%d", &input); /* Nutzereingabe */
				output = input + 1;
				printf("Der Nachfolger von %d ist %d.\n", input, output);
				return 0;
			}
		\end{lstlisting}
	\end{minipage}
	\quad
	\begin{minipage}{0.5\linewidth} %8 { fehlt
		\begin{lstlisting}[gobble=6]
			#include <stdio.h>
			
			int input, output;
			
			int main() 
			
				scanf("%d", &input); /* Nutzereingabe */
				output = 1 + input;
				printf("Der Nachfolger von %d ist %d.\n", input, output);
				return 0;
			}
		\end{lstlisting}
	\end{minipage}
	%
	\begin{minipage}{0.5\linewidth} %9 ; fehlt nach scanf
		\begin{lstlisting}[gobble=6]
			#include <stdio.h>
			
			int input, output;
			
			int main() {
			
				scanf("%d", &input) /* Nutzereingabe */
				output = input + 1;
				printf("Der Nachfolger von %d ist %d.\n", input, output);
				return 0;
			}
		\end{lstlisting}
	\end{minipage}
	\quad
	\begin{minipage}{0.5\linewidth} %10 richtig (Argumente von printf werden von rechts nach links ausgewertet)
		\begin{lstlisting}[gobble=6]
			#include <stdio.h>
			
			int input, output;
			
			int main() {
				
				scanf("%d", &input); /* Nutzereingabe */
				output = input;
				printf("Der Nachfolger von %d ist %d.\n", --input, ++input);
				return 0;
			}	
		\end{lstlisting}
	\end{minipage}
	
	\section{Standardeingabe \& Standardausgabe}
	
	Schreiben Sie ein Programm, das Zeichen von der Standardausgabe einliest und auf die Standardausgabe zurückschreibt.
	Dabei sollen alle Kleinbuchstaben durch Großbuchstaben ersetzt werden.
	Testen Sie Ihr Programm mit Dateien als Ein- und Ausgabe (als Eingabe können Sie z.B. den Programmquellcode verwenden).
	
	\textit{Hinweis: Zeichen werden als Zahlen kodiert: siehe Vorlesungsfolie 41 und/oder Internetrecherche nach \enquote{ASCII}.}

	\section{Umrechnung ins Oktalsystem}
	
		Schreiben Sie ein Programm, das eine positive ganze Zahl in das oktale Zahlensystem übersetzt.
		Dabei dürfen Sie die Oktalstellen in umgedrehter Reihenfolge ausgeben (d.h., niederwertigste
		Stelle zuerst), so dass Sie etwa für die Zahl $13_{10}$ die Ausgabe $51_8$ erhalten (in normaler Reihung
		wäre die Ausgabe $15_8$

	
\end{document}