% !TeX spellcheck = de_DE
\documentclass[a4paper]{article}

\usepackage[utf8]{inputenc}
\usepackage[T1]{fontenc}
\usepackage[scaled]{beramono}
\usepackage{amssymb}
\usepackage{amsmath}
\usepackage{graphicx}
\usepackage[top=2cm, bottom=2cm, left=2cm, right=4cm]{geometry}
\usepackage{csquotes}
\usepackage{listings}
\usepackage{hyperref}
\usepackage{booktabs}

\lstset{
	tabsize=2,
	language=Java,
	showstringspaces=false,
	basicstyle=\footnotesize\ttfamily,
}

%no indentation
\setlength\parindent{0pt}

\pagestyle{empty}

\begin{document}

	\section{Strukturen und dynamischer Speicher}
	
	\subsection{Erstellen neuer Werte}
	
	Eins der beiden Programme stürzt ab, wenn man es ausführt.
	Welches und warum?
	
	\begin{minipage}{.48\textwidth}
	\begin{lstlisting}[gobble=4]
		#include <stdlib.h>
		#include <stdio.h>
		
		int* number() {
			int n;
			n = 4;
			return &n;
		}
		
		int main() {
			int *n = number();
			printf("%d\n", *n);
			return 0;
		}
	\end{lstlisting}
	\end{minipage}\hspace{0.04\textwidth}
	\begin{minipage}{.48\textwidth}
	\begin{lstlisting}[gobble=4]
		#include <stdlib.h>
		#include <stdio.h>
		
		int* number() {
			int *n = malloc(sizeof(int));
			*n = 4;
			return n;
		}
		
		int main() {
			int *n = number();
			printf("%d\n", *n);
			return 0;
		}
	\end{lstlisting}
	\end{minipage}
	
	\clearpage
	
	\subsection{Brüche}
	
	\begin{enumerate}
		\item Definieren Sie eine Struktur zum Speichern von Brüchen.
			Brüche bestehen aus einem ganzzahligen Zähler und einem ganzzahligen Nenner.
		\item Schreiben Sie eine Prozedur \lstinline|print|, die eine Referenz auf einen Bruch übergeben bekommt und den Bruch ausgibt.
		\item Schreiben Sie eine Prozedur \lstinline|reduce|, die eine Referenz auf einen Bruch übergeben bekommt und den Bruch kürzt.
		\item Schreiben Sie eine Funktion \lstinline|add|, die zwei Referenzen auf Brüche übergeben bekommt und eine Referenz auf die Summe beider Brüche zurückgibt.
			Achten Sie darauf, die vorhandenen Brüche nicht zu überschreiben.
	
			Nun sollte die \lstinline|main|-Funktion links die Ausgabe rechts liefern.
			
			\begin{minipage}[t][][b]{0.65\textwidth}
			\begin{lstlisting}[gobble=8]
				int main() {		
				  Rational r1 = {1, 2};
				  Rational r2 = {1, 3};
				  Rational r3 = {2, 3};
				
				  RationalRef sum = add(add(&r1, &r2), &r3);
				  
				  printf("sum is: "); print(sum); printf("\n");
				  
				  reduce(sum);
				  
				  printf("reduced sum is: "); print(sum); printf("\n");
				  
				  return 0;
				}
			\end{lstlisting}
			\end{minipage}\qquad
			\begin{minipage}[t][][b]{0.25\textwidth}
			\begin{lstlisting}[gobble=8]
				sum is: 27/18
				reduced sum is: 3/2
			\end{lstlisting}
			\end{minipage}
	
		\item Schreiben Sie eine Funktion \lstinline|compare|, die zwei Referenzen auf Brüche übergeben bekommt und
			\begin{itemize}
				\item 0 zurückgibt, wenn beide Brüche gleich sind,
				\item eine Zahl < 0 zurückgibt, wenn der erste Bruch kleiner ist als der zweite, und
				\item eine Zahl > 0 zurückgibt, wenn der erste Bruch größer ist als der zweite.
			\end{itemize}
		\item Nutzen Sie die Bibliotheksfunktion \lstinline|qsort| und Ihre Funktion \lstinline|compare| um ein Feld von Brüchen zu sortieren (wie in Vorlesung auf den Folien 185ff. gezeigt).

			Nun sollte die \lstinline|main|-Funktion links die Ausgabe rechts liefern.
			
			\begin{minipage}[t][][b]{0.65\textwidth}
			\begin{lstlisting}[gobble=8]
				int main() {		
					int i;
					
					Rational rs[5] = {{1, 2}, {5, 2}, {1, 3}, {1, 4}, {3, 4}};
					
					qsort(rs, 5, sizeof(Rational), compare);
					
					for(i = 0; i < 5; i++) {
						print(rs + i); printf("\n");
					}
					
					return 0;
				}
			\end{lstlisting}
			\end{minipage}\qquad
			\begin{minipage}[t][][b]{0.25\textwidth}
			\begin{lstlisting}[gobble=8]
				1/4
				1/3
				1/2
				3/4
				5/2
			\end{lstlisting}
			\end{minipage}		
	
	\end{enumerate}

	\clearpage

	\section{Verkettete Listen}
	
	Verwenden Sie mindestens folgende Dateien:
	\begin{description}
		\item[\texttt{list.h}] enthält die Schnittstelle der Liste
		\item[\texttt{list.c}] enthält die Implementierung der einfach verketteten Liste
		\item[\texttt{main.c}] enthält die \texttt{main}-Funktion mit Testcode
	\end{description}
	
	\subsection{Typdefinition}
	
	Definieren Sie mit Hilfe von \texttt{typedef} einen Typen \texttt{Elem}, der auf eine Struktur mit einem \texttt{int}-Wert und einer Referenz auf ein nächstes \texttt{Elem} abgebildet ist.
	Definieren Sie außerdem einen Typen \texttt{List}, der eine Referenz auf ein \texttt{Elem} ist.
	
	\subsection{Elemente einfügen}
	
	Definieren Sie eine Funktion \texttt{insert}, die eine Liste und ein \texttt{int} übergeben bekommt und den \texttt{int}-Wert als neues Element vorn in die Liste einfügt.
	
	\subsection{Ausgabe von Listen}
	
	Definieren Sie eine Funktion \texttt{print}, die eine Liste übergeben bekommt und alle Elemente in der Liste ausgibt.
	Beispielsweise soll das Programm links die Ausgabe rechts liefern:
	
	\noindent\begin{minipage}[t][][b]{0.7\linewidth}
	\begin{lstlisting}[gobble=4]
		int main() {
			List l;
			
			l = insert(insert(insert(insert(NULL, 4), 6), 3), 9);
			print(l);
			
			return 0;
		}
	\end{lstlisting}
	\end{minipage}\qquad
	\begin{minipage}[t][][b]{0.2\linewidth}
	\begin{lstlisting}[gobble=4]
		9 3 6 4
	\end{lstlisting}
	\end{minipage}
	
	\subsection{\texttt{map} auf verketteten Listen}
	
	Definieren Sie eine Funktion \texttt{map}, die eine Liste und eine Funktion übergeben bekommt und eine neue Liste zurückgibt, die dadurch entsteht, dass die übergebene Funktion auf jedes Element der übergebenen Liste angewendet wird.
	Achten Sie darauf, dass die übergebene Liste intakt gelassen wird.
	Beispielsweise soll das Programm links die Ausgabe rechts liefern:
		
	\noindent\begin{minipage}[t][][b]{0.7\linewidth}
	\begin{lstlisting}[gobble=4]
		int times_two(int n) {
			return 2*n;
		}
		
		int main() {
			List l, m;
			
			l = insert(insert(insert(insert(NULL, 4), 6), 3), 9);
			print(l);
			
			m = map(l, times_two);
			print(l);
			print(m);
			
			return 0;
		}
	\end{lstlisting}
	\end{minipage}\qquad
	\begin{minipage}[t][][b]{0.2\linewidth}
	\begin{lstlisting}[gobble=4]
		9 3 6 4 
		9 3 6 4 
		18 6 12 8 
	\end{lstlisting}
	\end{minipage}
	
	
\end{document}