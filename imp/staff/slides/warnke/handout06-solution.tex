% !TeX spellcheck = de_DE
\documentclass[]{article}

\usepackage[utf8]{inputenc}
\usepackage[T1]{fontenc}
\usepackage[scaled]{beramono}
\usepackage{amssymb}
\usepackage{amsmath}
\usepackage{graphicx}
\usepackage[top=2cm, bottom=2cm, left=2cm, right=4cm]{geometry}
\usepackage{csquotes}
\usepackage{listings}
\usepackage{hyperref}
\usepackage{booktabs}

\lstset{
	tabsize=2,
	language=Java,
	showstringspaces=false,
	basicstyle=\footnotesize\ttfamily,
}

%no indentation
\setlength\parindent{0pt}

\pagestyle{empty}

\begin{document}

	\section{Funktionen}
	
	\subsection{Funktionsaufrufe}
	
	\begin{enumerate}
		\item \texttt{main()}
		\item \texttt{print\_with\_square(3)}
		\item \texttt{square(3)}
		\item \texttt{print\_with\_square(5)}
		\item \texttt{square(5)}
	\end{enumerate}

	\subsection{Alphabet}
	
	\begin{lstlisting}[gobble=4]
		int offset(int c) {
		  if (c >= 'A' && c <= 'Z') {
		    return c - 'A';
		  } else if (c >= 'a' && c <= 'z') {
		    return c - 'a';
		  } else {
		    return -1;
		  }
		}
	\end{lstlisting}
	
	\subsection{Primzahlen}

	\begin{lstlisting}[gobble=4]
		#include <stdio.h>
		
		int isPrime(int n) {
			int f, isprime = 1;
			for(f = 2; isprime && f*f <= n; f++) {
				if(n % f == 0) isprime = 0;
			}
			return isprime;
		}
				
		int main() {
			int limit, number;
			
			printf("Search limit: ");
			scanf("%d", &limit);
			
			for(number = 2; number <= limit; number++) {
				if(isPrime(number)) {
					printf("Found prime: %d\n", number);
				}
			}
			return 0;
		}	
	\end{lstlisting}

	\clearpage
	
	\section{Felder}
	
	\subsection{Summe}
	
	\begin{lstlisting}[gobble=4]
		double array_sum(double nums[], int n) {
			int i;
			double sum;
			
			for(i = 0; i < n; i++)
				sum += nums[i];
			
			return sum;
		}
	\end{lstlisting}
	
	\subsection{Ausgabe}
	
	\begin{lstlisting}[gobble=4]
		void print_array(int a[3][3]) {
			int i, j;
			
			for(i = 0; i < 3; i++) {
				for(j = 0; j < 3; j++) {
					printf("%5d ", a[i][j]);
				}
				printf("\n");
			}
		}
	\end{lstlisting}
	
	\subsection{Textstatistik}
	
	\begin{lstlisting}[gobble=4]
		int main() {		
			int i, c, n;
			int count[26];
			
			for(i = 0; i < 26; ++i) {
				count[i] = 0;
			}
			
			while ((c = getchar()) != EOF) {
				n = offset(c);
				if (n != -1) {
					count[n]++;
				}
			}
			
			for(i = 0; i < 26; ++i) {
				printf("%c: %d\n", i + 65, count[i]);
			}
		}
	\end{lstlisting}	
	
\end{document}