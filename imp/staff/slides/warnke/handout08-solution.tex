% !TeX spellcheck = de_DE
\documentclass[]{article}

\usepackage[utf8]{inputenc}
\usepackage[T1]{fontenc}
\usepackage[scaled]{beramono}
\usepackage{amssymb}
\usepackage{amsmath}
\usepackage{graphicx}
\usepackage[top=2cm, bottom=1cm, left=2cm, right=4cm]{geometry}
\usepackage{csquotes}
\usepackage{listings}
\usepackage{hyperref}
\usepackage{booktabs}

\lstset{
	tabsize=2,
	language=Java,
	showstringspaces=false,
	basicstyle=\footnotesize\ttfamily,
}

%no indentation
\setlength\parindent{0pt}

\pagestyle{empty}

\begin{document}

	\section{Iteration und Rekursion}
	
	\subsection{Schleifenarten und Rekursion}
	
	\begin{lstlisting}[gobble=4]
		void count_down(int i) {
		  if (i > 0) {
		    printf("%d\n", i);
		    count_down(i - 3);
		  }
		}
		
		int main() {
		  int i = 30;
		  
		  for (i = 30; i > 0; i -= 3) {
		    printf("%d\n", i);
		  } 
		  
		  printf("\n");
		  
		  i = 30;
		  while (i > 0) {
		    printf("%d\n", i);
		    i -= 3;
		  }  
		  
		  printf("\n");
		 
		  i = 30;
		  do {
		    printf("%d\n", i);
		    i -= 3;
		  } while (i > 0);
		  
		  printf("\n");
		  
		  count_down(30);
		  
		  return 0;
		}
	\end{lstlisting}
	
	\clearpage
	
	\subsection{Summe in einem Feld}
	
	\begin{lstlisting}[gobble=4]
		int array_sum_iter(int *nums, int l) {
		  int i, sum = 0;
		  
		  for (i = 0; i < l; i++) {
		    sum += nums[i];
		  }
		  
		  return sum;
		}
		
		int array_sum_rec(int *nums, int l) {
		  if (l <= 0) {
		    return 0;
		  } else {
		    int tail_sum = array_sum_rec(nums + 1, l - 1);
		    return tail_sum + nums[0];
		  }
		}
	\end{lstlisting}

	\subsection{Maximum in einem Feld}

	\begin{lstlisting}[gobble=4]
		int array_max_iter(int *nums, int l) {
		  int i , max = -1;
		  
		  for (i = 0; i < l; i++) {
		    if (nums[i] > max) {
		      max = nums[i];
		    }
		  }
		  
		  return max;
		}
		
		int array_max_rec(int *nums, int l) {
		  if (l <= 0) {
		    return -1;
		  } else {
		    int tail_max = array_max_rec(nums + 1, l - 1);
		    if (nums[0] > tail_max) {
		      return nums[0];
		    } else {
		      return tail_max;
		    }
		  }
		}		
	\end{lstlisting}
	
	\clearpage

	\section{Referenzen}
	
	\subsection{Swap für Referenzen}
	
	\begin{lstlisting}[gobble=4]
		void swap(double **x, double **y) {
		  double *tmp = *x;
		  *x = *y;
		  *y = tmp;
		}
	\end{lstlisting}
	
	\subsection{Call by reference}
	
	\begin{lstlisting}[gobble=4]
		void minmax(int *nums, int length, int *min, int *max) {
		
			int curmin = nums[0]; /* assumes non-empty list! */
			int curmax = nums[0]; /* assumes non-empty list! */
			int i;
			
			for(i=0; i<length; i++) {
				if(nums[i] < curmin) curmin = nums[i];
				if(nums[i] > curmax) curmax = nums[i];
			}
			*min = curmin;
			*max = curmax;
		}		
	\end{lstlisting}	
	
	\subsection{Vergleich von Zeichenketten}
	
	\begin{lstlisting}[gobble=4]
		int equals(char *one, char *two) {    
			if(*one != *two) return 0;
			if(*one == '\0') return 1;
			return equals(one+1, two+1);
		}
	\end{lstlisting}

\end{document}