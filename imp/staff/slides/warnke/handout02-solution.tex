% !TeX spellcheck = de_DE
\documentclass[]{article}

\usepackage[utf8]{inputenc}
\usepackage[T1]{fontenc}
\usepackage[scaled]{beramono}
\usepackage{amssymb}
\usepackage{amsmath}
\usepackage{graphicx}
\usepackage[top=2cm, bottom=2cm, left=2cm, right=4cm]{geometry}
\usepackage{csquotes}
\usepackage{listings}
\usepackage{hyperref}
\usepackage{booktabs}

\lstset{
	tabsize=2,
	language=Java,
	showstringspaces=false,
	basicstyle=\footnotesize\ttfamily,
}

%no indentation
\setlength\parindent{0pt}

\pagestyle{empty}
\usepackage{hyperref}

\begin{document}	

	\section{Ein- und Ausgabe}
	
	\begin{itemize}	
	
	\item Modifizieren Sie das Programm so, dass zwei Zahlen eingelesen werden und deren Summe ausgegeben wird.
	
	\begin{lstlisting}[gobble=4]
		#include <stdio.h>
		
		int main() {
			
			int a;
			int b;
			
			printf("Bitte geben Sie die erste Zahl ein:  ");
			scanf("%d", &a);
			printf("Bitte geben Sie die zweite Zahl ein: ");
			scanf("%d", &b);			
			
			printf("Die Summe von %d und %d ist %d.\n", a, b, a + b);
			
			return 0;
		}
		
	\end{lstlisting}

	
	\end{itemize}
	
	\section{Algorithmen}
	
	\begin{description}
		\item[Quersumme]
		
		Eingabe: eine natürliche Zahl $a$
		
		Ausgabe: $s$, die Quersumme von $a$
		
		\begin{enumerate}
			\item Speichere $a$ in Variable $x$
			\item Setze Variable $s$ auf 0
			\item Bestimme eine Zahl $b$ so, dass $a = 10 \cdot y + b$ und $0 \leq b < 10$ \\(Anders gesagt: $b = a \bmod 10$)
			\item Speichere den Wert von $y$ in $x$
			\item Speichere die Summe von $b$ und $s$ in $s$
			\item Wenn $y \neq 0$, dann weiter mit 3, sonst weiter mit 7
			\item Gib den Wert von $s$ zurück
		\end{enumerate}
		
		\item[Teilbarkeit durch 3]
		
		Eingabe: eine natürliche Zahl $n$
		
		Ausgabe: \emph{wahr}, wenn $n$ durch $3$ teilbar ist, sonst \emph{falsch}
		
		\begin{enumerate}
			\item Speichere $n$ in Variable $x$
			\item Speichere die Quersumme von $x$ in Variable $s$ (zum Bestimmen der Quersumme kann der obige Algorithmus benutzt werden)
			\item Wenn $s < 10$, weiter mit 6, sonst weiter mit 4
			\item Speichere den Wert von $s$ in $x$
			\item Weiter mit 2
			\item Wenn $s = 0$ oder $s = 3$ oder $s = 6$ oder $s = 9$, gib \emph{wahr} zurück, sonst gib \emph{falsch} zurück
		\end{enumerate}
		
	\end{description}

	
\end{document}