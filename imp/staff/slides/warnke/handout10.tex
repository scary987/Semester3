% !TeX spellcheck = de_DE
\documentclass[a4paper]{article}

\usepackage[utf8]{inputenc}
\usepackage[T1]{fontenc}
\usepackage[scaled]{beramono}
\usepackage{amssymb}
\usepackage{amsmath}
\usepackage{graphicx}
\usepackage[top=2cm, bottom=1cm, left=2cm, right=4cm]{geometry}
\usepackage{csquotes}
\usepackage{listings}
\usepackage{hyperref}
\usepackage{booktabs}

\lstset{
	tabsize=2,
	language=Java,
	showstringspaces=false,
	basicstyle=\footnotesize\ttfamily,
}

%no indentation
\setlength\parindent{0pt}

\pagestyle{empty}

\begin{document}

	\section{Referenzen, Strukturen und dynamischer Speicher}
	
	\subsection{Swap für Referenzen}
	
	Schreiben Sie eine Funktion \texttt{swap}, die die Werte von zwei Referenzen austauscht.
	Wenn also \texttt{a} auf \texttt{x} zeigt und \texttt{b} auf \texttt{y}, soll nach Anwendung von \texttt{swap} \texttt{a} auf \texttt{y} zeigen und \texttt{b} auf \texttt{x}.
	Beispielsweise soll das Programm links die Ausgabe rechts liefern:

	\noindent\begin{minipage}[t][][b]{0.6\linewidth}
	\begin{lstlisting}[gobble=4]
		int main() {    
			double x = 3.0, y = 5.0;
			double *a = &x, *b = &y;
			printf("*a = %0.1f, *b = %0.1f\n", *a, *b);
			swap(&a, &b);
			printf("*a = %0.1f, *b = %0.1f\n", *a, *b);
			return 0;
		}
	\end{lstlisting}
	\end{minipage}\qquad
	\begin{minipage}[t][][b]{0.3\linewidth}
	\begin{lstlisting}[gobble=4]
		*a = 3.0, *b = 5.0
		*a = 5.0, *b = 3.0
	\end{lstlisting}
	\end{minipage}
	
	\subsection{Zeichenketten mit dynamischer Länge}

	Modifizieren Sie das Listing so, dass Autor und Titel beliebiger Längen vom Nutzer eingelesen werden.
	Fragen Sie den Nutzer dazu erst, wie lang die Zeichenketten sind.
	
	\begin{lstlisting}[gobble=4]
		typedef struct _Book {
			char author[20];
			char title[20];
		} Book, *BookRef;
		
		main () {
			BookRef a;
			a = (BookRef) malloc(sizeof(Book));
			/* read in values for author and title */
			free(a);
		}
	\end{lstlisting}
	
	\clearpage
	
	\section{Verkettete Listen}
	
	\subsection{Typdefinition}
	
	Definieren Sie mit Hilfe von \texttt{typedef} einen Typen \texttt{Elem}, der auf eine Struktur mit einem \texttt{int}-Wert und einer Referenz auf ein nächstes \texttt{Elem} abgebildet ist.
	Definieren Sie außerdem einen Typen \texttt{List}, der eine Referenz auf ein \texttt{Elem ist}.
	
	\subsection{Elemente einfügen}
	
	Definieren Sie eine Funktion \texttt{insert}, die eine Liste und ein \texttt{int} übergeben bekommt und den \texttt{int}-Wert als neues Element vorn in die Liste einfügt.
	
	\subsection{Ausgabe von Listen}
	
	Definieren Sie eine Funktion \texttt{print}, die eine Liste übergeben bekommt und alle Elemente in der Liste ausgibt.
	Beispielsweise soll das Programm links die Ausgabe rechts liefern:
	
	\noindent\begin{minipage}[t][][b]{0.7\linewidth}
	\begin{lstlisting}[gobble=4]
		int main() {
			List l;
			
			l = insert(insert(insert(insert(NULL, 4), 6), 3), 9);
			print(l);
			
			return 0;
		}
	\end{lstlisting}
	\end{minipage}\qquad
	\begin{minipage}[t][][b]{0.2\linewidth}
	\begin{lstlisting}[gobble=4]
		9 3 6 4
	\end{lstlisting}
	\end{minipage}
	
	\subsection{\texttt{map} auf verketteten Listen}
	
	Definieren Sie eine Funktion \texttt{map}, die eine Liste und eine Funktion übergeben bekommt und eine neue Liste zurückgibt, die dadurch entsteht, dass die übergebene Funktion auf jedes Element der übergebenen Liste angewendet wird.
	Achten Sie darauf, dass die übergebene Liste intakt gelassen wird.
	Beispielsweise soll das Programm links die Ausgabe rechts liefern:
		
	\noindent\begin{minipage}[t][][b]{0.7\linewidth}
	\begin{lstlisting}[gobble=4]
		int times_two(int n) {
			return 2*n;
		}
		
		int main() {
			List l, m;
			
			l = insert(insert(insert(insert(NULL, 4), 6), 3), 9);
			print(l);
			
			m = map(l, times_two);
			print(l);
			print(m);
			
			return 0;
		}
	\end{lstlisting}
	\end{minipage}\qquad
	\begin{minipage}[t][][b]{0.2\linewidth}
	\begin{lstlisting}[gobble=4]
		9 3 6 4 
		9 3 6 4 
		18 6 12 8 
	\end{lstlisting}
	\end{minipage}
	
\end{document}