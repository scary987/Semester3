% !TeX spellcheck = de_DE
\documentclass[]{article}

\usepackage[utf8]{inputenc}
\usepackage[T1]{fontenc}
\usepackage[scaled]{beramono}
\usepackage{amssymb}
\usepackage{amsmath}
\usepackage{graphicx}
\usepackage[top=2cm, bottom=2cm, left=2cm, right=4cm]{geometry}
\usepackage{csquotes}
\usepackage{listings}
\usepackage{hyperref}
\usepackage{booktabs}

\lstset{
	tabsize=2,
	language=Java,
	showstringspaces=false,
	basicstyle=\footnotesize\ttfamily,
}

%no indentation
\setlength\parindent{0pt}

\pagestyle{empty}
\usepackage{hyperref}

\begin{document}	

	Erinnerung: Arbeiten Sie auf dem Laufwerk \texttt{R:}.
	
	\section{Subversion mit RapidSVN oder TortoiseSVN}
	
	\begin{enumerate}
		\item Laden Sie sich die bebilderte Anleitung zu RapidSVN oder TortoiseSVN aus dem Stud.IP herunter.
		Auf UniComp finden Sie RapidSVN, auf den Rechnern im Raum 310 TortoiseSVN.
		\item Probieren Sie die in der Anleitung gezeigten Operationen aus. 
		Überprüfen Sie nach jeder Änderung mit einem Browser, ob die Änderung auf dem Server sichtbar ist.
		\begin{itemize}
			\item Checken Sie den Playgroundordner (\url{https://svn.informatik.uni-rostock.de/lehre/ip2019/playground}) aus.
			\item Erstellen Sie darin einen Ordner xy123 (Ihr eigenes ITMZ-Kürzel) und commiten Sie ihn.
			\item Erstellen und commiten Sie eine Datei in Ihrem Ordner.
			\item Führen Sie im playground ein Update aus und schauen Sie, ob sie die Ordner Ihrer Kommilitonen sehen.
			\item Sehen Sie sich das Log des Playground-Ordners an.
			\item Geben Sie die URL des playground in einem Webbrowser ein und erkunden Sie den Inhalt auf diese Weise.
		\end{itemize}
		\item Wiederholen Sie die Schritte im Ordner Ihrer Hausaufgabengruppe (\url{https://svn.informatik.uni-rostock.de/lehre/ip2019/groups/xy} für Gruppe Nummer xy)
	\end{enumerate}

	\section{Ein- und Ausgabe}
	
	\begin{itemize}
	\item Speichern Sie den folgenden Programmquellcode als \texttt{inout.c}, kompilieren Sie ihn (\texttt{-Wall} \texttt{-pedantic} nicht vergessen) und führen Sie das entstehende Programm aus.
	
	\begin{lstlisting}[gobble=4]
		#include <stdio.h>
		
		main() {
			
			int a;
			
			printf("Bitte geben Sie eine Zahl ein: ");
			scanf("%d", &a);
			
			printf("Das Quadrat von %d ist %d.\n", a, a*a);
		}
		
	\end{lstlisting}
	
	\item Modifizieren Sie das Programm so, dass zwei Zahlen eingelesen werden und deren Summe ausgegeben wird.
	
	\item Commiten Sie die fertige Quellcodedatei in Ihren Ordner im playground.
	
	\end{itemize}
	
	\section{Algorithmen}

	Ihre Beschreibungen können zum Beispiel so aussehen wie in der Vorlesung auf Folie 18.
	
	\begin{enumerate}
		\item Beschreiben Sie einen Algorithmus, der die Quersumme einer Zahl bestimmt.
		\item Beschreiben Sie einen Algorithmus, der mit Hilfe der Quersumme entscheidet, ob eine Zahl durch 3 teilbar ist.
			Dazu muss die Quersumme eventuell mehrmals gebildet werden.
	\end{enumerate}

	
\end{document}