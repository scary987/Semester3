% !TeX spellcheck = de_DE
%% 
%% This is file 'beamer_sample.tex'
%% 	
%% by Mathias Winkel
%%
%% Problems, bugs and comments to 
%% mathias.winkel2@uni-rostock.de
%%
\RequirePackage{fix-cm} % because of font size substituion warnings
\documentclass[10pt]{beamer} % die 10pt sollten festgelegt bleiben, da dies die Groesse der Mathematikschrift etc. beeinflusst

\usepackage[ngerman]{babel}  % deutsche Bezeichnungen und Trennung etc
\usepackage[utf8]{inputenc}
\usepackage[T1]{fontenc}
\usepackage{csquotes}
\usepackage{hyperref}        % interne Hyperlinks
\usepackage[scaled]{beramono}
\usepackage{microtype}
\usepackage{listings}
\usepackage{ragged2e}
\usepackage{marvosym}
\usepackage{xcolor}
\usepackage{pgffor}

%******************************************
% For giving image sources, see http://tex.stackexchange.com/a/48485
\usepackage[absolute,overlay]{textpos}
\setbeamercolor{framesource}{fg=gray}
\setbeamerfont{framesource}{size=\tiny}

\setbeamercolor{plain}{fg=black,bg=white}

\newcommand{\source}[1]{\begin{textblock*}{4cm}(8.2cm,8.6cm)
		\begin{beamercolorbox}[ht=0.5cm,right]{framesource}
			\usebeamerfont{framesource}\usebeamercolor[fg]{framesource} Quelle: {#1}
		\end{beamercolorbox}
	\end{textblock*}}
%******************************************

\lstdefinestyle{customc}{
	tabsize=2,
	belowcaptionskip=1\baselineskip,
	breaklines=true,
	xleftmargin=\parindent,
	language=C,
	showstringspaces=false,
	basicstyle=\scriptsize\ttfamily,
	keywordstyle=\bfseries\color{green!40!black},
	commentstyle=\itshape\color{purple!40!black},
	identifierstyle=\color{blue},
	stringstyle=\color{orange},
}

\lstset{escapechar=@,style=customc}

\definecolor{links}{HTML}{2A1B81}
\hypersetup{colorlinks,linkcolor=,urlcolor=structure.fg}

\newcommand{\framefrompdf}[2]{
	\begin{frame}[plain]
		\hspace*{-8.5pt}
		\begin{beamercolorbox}[wd=\paperwidth,ht=\paperheight]{plain}
			\makebox[\paperwidth]{\hspace{8.5pt}\includegraphics[page=#2,width=\paperwidth]{#1}}
		\end{beamercolorbox}
	\end{frame}
}

\newcommand{\framesfrompdf}[3]{
	\foreach \n in {#2,...,#3}{\framefrompdf{#1}{\n}}
}

\newcommand{\framesfrompdfclipped}[8]{
	\foreach \n in {#2,...,#3}{
		\begin{frame}[plain]
			\vspace*{-18pt}\hspace*{-8.5pt}
			\makebox[\linewidth]{\includegraphics[page=\n,width=#8,clip,trim = #4 #5 #6 #7]{#1}}
		\end{frame}
	}
}

\newcommand{\picturefrompdf}[7]{
	\begin{center}
		%trim option's parameter order: left bottom right top
		\includegraphics[page=#2,width=#7,clip,trim = #3 #4 #5 #6]{#1} 
	\end{center}
}

\newcommand{\rto}{$\Rightarrow$ }

\newcommand{\emp}[1]{{\color{orange}{#1}}}

\newcommand{\img}[2]{
	\centering
	\includegraphics[#2]{figures/#1}
}

\newcommand{\questions}{
\begin{frame}
	
	\centering
	\Large
	Fragen?

\end{frame}
}

\newenvironment{task}[1]{
	\begin{block}{\textbf{Aufgabe} \hfill #1}
}{
	\end{block}
}

\AtBeginSection[]
{
	\begin{frame}{Plan für heute}
		\tableofcontents[currentsection]
	\end{frame}
}

\usepackage[uni,footuni,headlogo]{./unirostock/beamerthemeRostock}
% erster Parameter: Farbschema
%        moegliche Werte (nomen est omen): uni, inf, msf, ief, mnf, mef, juf, wsf, auf, thf, phf
%        Standardwert: uni
% zweiter Parameter: Fusszeile
%        moegliche Werte: footuni (Standard) - Fusszeile nach Handbuch des CD
%                         foottitle          - Autor und Title in der Fusszeile
%                         footheadings       - lebende Ueberschriften in der Fusszeile
%                         footuniheadings    - Autor und Uni sowie lebende Ueberschriften in der Fusszeile
% dritter Parameter: Kopfzeile
%        moegliche Werte: headlogo (Standard)- Inhalt von \mylogo in der Kopfzeile
%                         headtitle          - Vortragstitel
%                         headframetitle     - Folientitel
%                         headframesubtitle  - Folientitel und -untertitel
%        bitte nicht mehrere Varianten gleichzeitig angeben :-)

\setbeamercovered{invisible}
\addtobeamertemplate{block begin}{}{\justifying}

%%%%%%%%%%%% Festlegung der Titelseite %%%%%%%%%%%%%%%%%%%%%%%%%%%%%%%%%
\title[]{Imperative Programmierung}

\subtitle{Übung}

\author{\textsc{Tom Warnke}}

\institute{Universität Rostock, Institut für Informatik}

\titlegraphic{}

% ein alternatives Titelbild kann mittels 
%\titleimage{Dateiname.xyz}
% angegeben werden (auf vernuenftiges Seitenformat und Kontrastwerte achten, Skalierung und Abschneiden der oberen rechten Ecke passieren automatisch)
\titleimage{}

%%%%%%%%%%%% Festlegungen fuer Kopf- und Fusszeile %%%%%%%%%%%%%%%%%%%%
% Institutsname f\"ur die Fusszeile (nur wenn bei Paketeinbindung 'footuni' angegeben ist)
\footinstitute{Fakultät für Informatik und Elektrotechnik, Institut für Informatik}
% eigenes Logo oben rechts hinzufuegen (bitte auf vernuenftiges Format achten - ein zu hohes Logo verschiebt das Layout)
\renewcommand{\mylogo}{}

%%%%%%%%%%%%%%%%%%%%%%%%%%%%%%%%%%%%%%%%%%%%%%%%%%%%%%%%%%%%%%%%%%%%%%%

\date{}

%%%%%%%%%%%%%%%%%%%%%%%%%%%%%%%%%%%%%%%%%%%%%%%%%%%%%%%%%%%%%%%%%%%%%%%
\begin{document}

\maketitle

\begin{frame}{Anfang}

	\begin{itemize}
		\item Fragen?
		\item SVN-Probleme?
		\item Hausaufgaben angefangen?
	\end{itemize}

\end{frame}

\begin{frame}{Plan für heute}

	\tableofcontents

\end{frame}

\section{Rückblick}

\begin{frame}[fragile]{Rückblick}{Funktionen}

	\begin{lstlisting}[gobble=4]
		#include <stdio.h>
		
		int capitalize(int c) {
			if (c >= 'a' && c <= 'z')   /*	if (c >= 97 && c <= 122)	*/
				return c - ('a' - 'A');   /*		return c - 32;					*/
			else												/*	else 											*/
				return c;									/*		return c;								*/
		}
		
		int main() {
			
			int c;
			
			while ((c = getchar()) != EOF) {
				putchar(capitalize(c));
			}
			
			return 0;
		}
	\end{lstlisting}

\end{frame}

\section{Nachbereitung Hausaufgabe}

\begin{frame}[fragile]{Lösungsvorschlag}{Aufgabe 2.1}

	\begin{lstlisting}[gobble=4]
		#include <stdio.h>
		#include <stdlib.h>
		#include <math.h>
		#include <time.h>
		
		main() {
			double ngood, x, y, pi4, dd;
			int ntotal, nk, ng, ne;
			srand(time(NULL)); /* Initialisierung Zufallszahlengenrator */
			ngood = 0.0;
			for(ntotal = 0; ; ntotal++){ 
				x = rand()*1.0/RAND_MAX;
				y = rand()*1.0/RAND_MAX;
				if(x*x+y*y < 1.0) ngood++;
				pi4 = 4 * ngood / ntotal;
				dd = pi4 - M_PI; /* Differenz Ergebnis mit Pi*/
				if(dd < 0.0) dd = -dd; /* Falls negativ , positive Differenz */
				if (dd<=1e-7) break; /* Abbruch bei Genauigkeit*/
			}
			printf("Anzahl der Versuche %d pi=%g eps=%g\n", ntotal, pi4, dd);
		 }
	\end{lstlisting}

\end{frame}

\section{Imperative Programmierung mit Stil}

\begin{frame}{Imperative Programmierung mit Stil}{K\&R über Einrückung, Klammern, \ldots}
	
	\enquote{The body of a while can be one or more statements enclosed in braces or a single statement without braces.	
		In either case, we will always indent the statements controlled by the while by one tab stop
		(which we have shown as four spaces) so you can see at a glance which statements are inside
		the loop. \emp{The indentation emphasizes the logical structure of the program.} Although C
		compilers do not care about how a program looks, proper indentation and spacing are critical
		in making programs easy for people to read. We recommend writing only one statement per
		line, and using blanks around operators to clarify grouping. The position of braces is less
		important, although people hold passionate beliefs. We have chosen one of several popular
		styles. Pick a style that suits you, then use it consistently.}
	
\end{frame}

\begin{frame}[fragile]{Einfache Anweisungen}

	\begin{columns}
	\begin{column}{.5\textwidth}
	
	\begin{lstlisting}[gobble=4]
		if(a > b) max = a;
	\end{lstlisting}
	
	\begin{lstlisting}[gobble=4]
		if(a > b) 
			max = a;
	\end{lstlisting}
	
	\end{column}
	\begin{column}{.5\textwidth}
	
	\begin{lstlisting}[gobble=4,showspaces,showtabs]
		if(a > b) max = a;
	\end{lstlisting}
	
	\begin{lstlisting}[gobble=4,showspaces,showtabs]
		if(a > b) 
			max = a;
	\end{lstlisting}
	
	\end{column}
	\end{columns}
	
\end{frame}

\begin{frame}[fragile]{Blöcke}

	\begin{columns}
	\begin{column}{.5\textwidth}
	
	\begin{lstlisting}[gobble=4]
		if(a > b) {max = a;}
	\end{lstlisting}
	
	\begin{lstlisting}[gobble=4]
		if(a > b) {
			max = a;
		}
	\end{lstlisting}
	
	\begin{lstlisting}[gobble=4]
		if(a > b) 
		{
			max = a;
		}
	\end{lstlisting}
	
	\end{column}
	\begin{column}{.5\textwidth}
	
	\begin{lstlisting}[gobble=4,showspaces,showtabs]
		if(a > b) {max = a;}
	\end{lstlisting}
	
	\begin{lstlisting}[gobble=4,showspaces,showtabs]
		if(a > b) {
			max = a;
		}
	\end{lstlisting}
	
	\begin{lstlisting}[gobble=4,showspaces,showtabs]
		if(a > b) 
		{
			max = a;
		}
	\end{lstlisting}

	\end{column}
	\end{columns}
	
\end{frame}

\begin{frame}[fragile]{\texttt{if-else}}

	\begin{columns}
	\begin{column}{.5\textwidth}
	
	\begin{lstlisting}[gobble=4]
		if(a > b) {
			max = a;
		} else {
			max = b;
		}
	\end{lstlisting}
	
	\begin{lstlisting}[gobble=4]
		if(a > b)
			max = a;
		else
			max = b;
	\end{lstlisting}
	
	\end{column}
	\begin{column}{.5\textwidth}
	
	\begin{lstlisting}[gobble=4,showspaces,showtabs]
		if(a > b) {
			max = a;
		} else {
			max = b;
		}
	\end{lstlisting}
	
	\begin{lstlisting}[gobble=4,showspaces,showtabs]
		if(a > b)
			max = a;
		else
			max = b;
	\end{lstlisting}
	
	\end{column}
	\end{columns}
	
\end{frame}

\begin{frame}[fragile]{Schleifen}

	\begin{columns}
	\begin{column}{.5\textwidth}
	
	\begin{lstlisting}[gobble=4]
		while(i < 100) {
			printf("%d\n", i);
			i++;
		}
	\end{lstlisting}
	
	\begin{lstlisting}[gobble=4]
		for(i = 0; i < 100; i++) {
			printf("%d\n", i);
		}
	\end{lstlisting}
	
	\end{column}
	\begin{column}{.5\textwidth}
	
	\begin{lstlisting}[gobble=4,showspaces,showtabs]
		while(i < 100) {
			printf("%d\n", i);
			i++;
		}
	\end{lstlisting}
	
	\begin{lstlisting}[gobble=4,showspaces,showtabs]
		for(i = 0; i < 100; i++) {
			printf("%d\n", i);
		}
	\end{lstlisting}

	\end{column}
	\end{columns}
	
\end{frame}

\lstset{showspaces=false,showtabs=false}

\begin{frame}{Zusammenfassung}

	\begin{itemize}
		\item Gute Formatierung macht das Lesen und Verstehen leichter
		\item Anweisungen im gleichen Block werden gleich weit eingerückt
		\item Editoren unterstützen dabei (z.B. mit automatischer Einrückung)
		\item Weitere Stilelemente: Kommentare, Namen von Variablen und Funktionen, \ldots
	\end{itemize}

\end{frame}

\begin{frame}{C-Standards}

	\begin{itemize}
		\item Es gibt nicht \enquote{ein} C, sondern verschiedene Dialekte
		\begin{itemize}
			\item K\&R C
			\item ANSI C = ISO C = Standard C = C89 = C90 \emp{$\Leftarrow$ Unser Standard}
			\item C99
			\item \ldots
		\end{itemize}
		\item Beim Kompilieren kann \texttt{gcc} über Kommandozeilenparameter mitgeteilt werden, welchem Standard der Quelltext entspricht und wie streng auf Standardkonformität geachtet werden soll
		\begin{itemize}
			\item \texttt{-Wall} gibt \underline{all}e \underline{W}arnungen aus
			\item \texttt{-pedantic} achtet \enquote{pedantisch} auf Einhaltung der Standards
		\end{itemize}
	\end{itemize}

\end{frame}

\begin{frame}[fragile]{C-Standards}

	\begin{lstlisting}[gobble=4]
		#include <stdio.h> 
		
		main() {
			// This prints a string to the screen
			printf("Start\n");
			
			int i;
			
			for(i = 0; i < 10; i++) {
				printf("%d * %d = %d\n", i, i, i * i);
			}
		}
	\end{lstlisting}
	
	\begin{itemize}
		\item \texttt{gcc prog.c -o prog}: 0 Warnungen 
		\item \texttt{gcc prog.c -o prog -Wall}: 2 Warnungen
		\item \texttt{gcc prog.c -o prog -pedantic}: 2 Warnungen
		\item \texttt{gcc prog.c -o prog -Wall -pedantic}: 4 Warnungen
	\end{itemize}

\end{frame}

\begin{frame}[fragile]{C-Standards}

	\begin{lstlisting}[gobble=4]
		#include <stdio.h> 
		
		int main() {
			int i;
			
			/* This prints a string to the screen */
			printf("Start\n");
			
			for(i = 0; i < 10; i++) {
				printf("%d * %d = %d\n", i, i, i * i);
			}
			
			return 0;
		}
	\end{lstlisting}
	
	\begin{itemize}
		\item \texttt{gcc prog.c -o prog -Wall -pedantic}: 0 Warnungen \\ \rto \emp{Standardkonform}
	\end{itemize}

\end{frame}

\section{Globale Variablen}

\begin{frame}{Globale Variablen}
	
	\begin{itemize}
		\item Variablen, die außerhalb eines Blocks definiert werden, sind \emp{globale Variablen}
		\item Eine globale Variable ist überall sichtbar, d.h.
		\begin{itemize}
			\item jede Funktion kann ihren Wert lesen
			\item jede Funktion kann ihren Wert überschreiben
		\end{itemize}
		\item Eine globale Variable bekommt beim Programmstart automatisch einen Wert zugewiesen, lokale Variablen müssen initialisiert werden (sonst drohen undefiniertes Verhalten/Fehler)
	\end{itemize}
	
\end{frame}

\begin{frame}[fragile]{Globale Variablen}
	
	\begin{lstlisting}
		#include <stdio.h>
		
		int global;
		
		void multiply(int factor) {
			global = global * factor;
		}
		
		void add(int summand) {
			global = global + summand;
		}
		
		int main() {
			add(4);
			multiply(5);
			printf("global=%d\n", global);
			return 0;
		}
	\end{lstlisting}
	
\end{frame}

\section{Felder}

\begin{frame}[fragile]{Felder}

	\begin{itemize}
		\item Bisher: ein Variablenname = ein Speicherplatz
		\item \emp{Felder} erlauben, mehrere Werte eines Typs in einer Variable zu speichern
		\item Ein Feld kann mehrere Dimensionen haben
	\end{itemize}

	\begin{columns}
		\begin{column}{0.45\textwidth}
			\begin{lstlisting}[gobble=8]
				#include <stdio.h>
				
				main() {
					int numbers[10];
					int i;
					
					for(i = 0; i < 10; ++i) {
						/* Schleife von 0 bis 9 */
						numbers[i] = 0;
					}
				}
			\end{lstlisting}	
		\end{column}
		\begin{column}{0.55\textwidth}
			\begin{itemize}
				\item Zugriff über \texttt{variable[index]}
				\item Erstes Element hat Index \texttt{0}
				\item Letztes Element hat Index \texttt{Länge-1}
				\item Ob der Index im gültigen Bereich liegt, wird vom Programm nicht überprüft!
			\end{itemize}
		\end{column}
	\end{columns}
	
\end{frame}

\begin{frame}[fragile]{Felder}{Initialisierung}
	
	Felder können mit einem Ausdruck mit \lstinline[basicstyle=\small\ttfamily]|{}| initialisiert werden:
	\begin{lstlisting}[gobble=4]
		int a[3] = {17, 18, 19};
	\end{lstlisting}
	
	Dann kann die Feldgröße weggelassen werden:
	\begin{lstlisting}[gobble=4]
		int a[] = {17, 18, 19};
	\end{lstlisting}
	
	Mehrdimensionale Felder können mit verschachtelten \lstinline[basicstyle=\small\ttfamily]|{}| initialisiert werden:
	\begin{lstlisting}[gobble=4]
		int a[3][2] = {{1, 17}, {2, 18}, {3, 19}};
	\end{lstlisting}
	
	Dann darf nur die Größe der äußersten Dimension weggelassen werden:
	\begin{lstlisting}[gobble=4]
		int a[][2] = {{1, 17}, {2, 18}, {3, 19}};
	\end{lstlisting}

\end{frame}

\section{Vorbereitung Hausaufgabe}

\begin{frame}{Ein Blick in die Hausaufgabe}

	\enquote{Erweitern Sie das Plotprogramm \texttt{sin0.c} (Skript Seite 61) so, dass es zwei Kurven gleichzeitig
	zeichnen kann. Dabei soll die eine Kurve durch \texttt{x} und die andere Kurve durch \texttt{+} dargestellt
	werden. Falls beide Zeichen auf der gleichen Position landen, soll statt dessen das Zeichen \texttt{*}
	ausgegeben werden.}
	
	\rto Kombination von Schleife, Verzweigung, Standardausgabe

\end{frame}

\begin{frame}{Ein Blick in die Hausaufgabe}

	\enquote{Wir basteln uns ein Grafikprogramm.}
	
	\begin{itemize}
		\item Funktionsköpfe sind gegeben
		\item Berechnungen aus der Geometrie
		\item \texttt{math.h} benutzen, z.B. für Wurzelberechnung
	\end{itemize}
	
	\enquote{Nutzen Sie die Funktionen um ein Programm zu schreiben, das folgende Ausgabe erzeugt:\\\dots\\
	Hinweis: die Schwierigkeit hier besteht darin, sich zu überlegen, wie das Ergebnis von Funktionen wie \texttt{line} oder \texttt{rectangle} genutzt werden kann, um Ausgabezeichen an der richtigen Stelle zu erzeugen.}
	

\end{frame}

\section{Programmieren}

\begin{frame}{Programmieren}

	\begin{task}{~}
		Lösen Sie die Aufgaben auf dem Arbeitsblatt \texttt{handout06.pdf}.
	\end{task}

\end{frame}

\section{Abschluss}

\begin{frame}[fragile]{Lösungsvorschlag}{Aufgabe 1.2}
	
	\begin{lstlisting}[gobble=4]
		#include <stdio.h>
		
		/* takes a character c and returns a number 
		- if c is a letter, returns the number of the letter
		- otherwise, returns -1 
		*/
		int offset(int c) {
			if (c >= 65 && c <= 90) { 			/* Grossbuchstabe */
				return c - 65;
			} else if (c >= 97 && c <= 122) { 	/* Kleinbuchstabe */
				return c - 97;
			} else { 							/* Sonstiges */
				return -1;
			}
		}
	\end{lstlisting}
	
\end{frame}

\begin{frame}[fragile]{Lösungsvorschlag}{Aufgabe 2.1}
	
	\begin{lstlisting}[gobble=4]
		double array_sum(double nums[], int n) {
			int i;
			double sum;
			
			for(i = 0; i < n; i++)
				sum += nums[i];
			
			return sum;
		}
	\end{lstlisting}
	
\end{frame}

\begin{frame}[fragile]{Lösungsvorschlag}{Aufgabe 2.2}
	
	\begin{lstlisting}[gobble=4]
		void print_array(int a[3][3]) {
			int i, j;
			
			for(i = 0; i < 3; i++) {
				for(j = 0; j < 3; j++) {
					printf("%5d ", a[i][j]);
				}
				printf("\n");
			}
		}
	\end{lstlisting}
	
\end{frame}

\begin{frame}[fragile]{Lösungsvorschlag}{Aufgabe 2.3}
	
	\begin{lstlisting}[gobble=4]
		int main() {		
			int i, c, n;
			int count[26];
			
			for(i = 0; i < 26; ++i) {
				count[i] = 0;
			}
			
			while ((c = getchar()) != EOF) {
				n = offset(c);
				if (n != -1) {
					count[n]++;
				}
			}
			
			for(i = 0; i < 26; ++i) {
				printf("%c: %d\n", i + 65, count[i]);
			}
		}
	\end{lstlisting}
	
\end{frame}

	% end of presentation
	%%%%%%%%%%%%%%%%%%%%%%%%%%%%%%%%%%%%%%%%%%%%%%%%%%%%%%%%%%%%%%%%%%%%%%%
	% backup slides
	
	\appendix
	\newcounter{finalframe}
	\setcounter{finalframe}{\value{framenumber}}
	
	%%%%%%%%%%%%%%%%%%%%%%%%%%%%%%%%%%%%%%%%%%%%%%%%%%%%%%%%%%%%%%%%%%%%%%%
	% end of backup slides
	\setcounter{framenumber}{\value{finalframe}}
\end{document}

