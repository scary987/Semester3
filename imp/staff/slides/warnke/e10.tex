% !TeX spellcheck = de_DE
%% 
%% This is file 'beamer_sample.tex'
%% 	
%% by Mathias Winkel
%%
%% Problems, bugs and comments to 
%% mathias.winkel2@uni-rostock.de
%%
\RequirePackage{fix-cm} % because of font size substituion warnings
\documentclass[10pt]{beamer} % die 10pt sollten festgelegt bleiben, da dies die Groesse der Mathematikschrift etc. beeinflusst

\usepackage[ngerman]{babel}  % deutsche Bezeichnungen und Trennung etc
\usepackage[utf8]{inputenc}
\usepackage[T1]{fontenc}
\usepackage{csquotes}
\usepackage{hyperref}        % interne Hyperlinks
\usepackage[scaled]{beramono}
\usepackage{microtype}
\usepackage{listings}
\usepackage{ragged2e}
\usepackage{marvosym}
\usepackage{xcolor}
\usepackage{pgffor}

%******************************************
% For giving image sources, see http://tex.stackexchange.com/a/48485
\usepackage[absolute,overlay]{textpos}
\setbeamercolor{framesource}{fg=gray}
\setbeamerfont{framesource}{size=\tiny}

\setbeamercolor{plain}{fg=black,bg=white}

\newcommand{\source}[1]{\begin{textblock*}{4cm}(8.2cm,8.6cm)
		\begin{beamercolorbox}[ht=0.5cm,right]{framesource}
			\usebeamerfont{framesource}\usebeamercolor[fg]{framesource} Quelle: {#1}
		\end{beamercolorbox}
	\end{textblock*}}
%******************************************

\lstdefinestyle{customc}{
	tabsize=2,
	belowcaptionskip=1\baselineskip,
	breaklines=true,
	xleftmargin=\parindent,
	language=C,
	showstringspaces=false,
	basicstyle=\scriptsize\ttfamily,
	keywordstyle=\bfseries\color{green!40!black},
	commentstyle=\itshape\color{purple!40!black},
	identifierstyle=\color{blue},
	stringstyle=\color{orange},
}

\lstset{escapechar=@,style=customc}

\definecolor{links}{HTML}{2A1B81}
\hypersetup{colorlinks,linkcolor=,urlcolor=structure.fg}

\newcommand{\framefrompdf}[2]{
	\begin{frame}[plain]
		\hspace*{-8.5pt}
		\begin{beamercolorbox}[wd=\paperwidth,ht=\paperheight]{plain}
			\makebox[\paperwidth]{\hspace{8.5pt}\includegraphics[page=#2,width=\paperwidth]{#1}}
		\end{beamercolorbox}
	\end{frame}
}

\newcommand{\framesfrompdf}[3]{
	\foreach \n in {#2,...,#3}{\framefrompdf{#1}{\n}}
}

\newcommand{\framesfrompdfclipped}[8]{
	\foreach \n in {#2,...,#3}{
		\begin{frame}[plain]
			\vspace*{-18pt}\hspace*{-8.5pt}
			\makebox[\linewidth]{\includegraphics[page=\n,width=#8,clip,trim = #4 #5 #6 #7]{#1}}
		\end{frame}
	}
}

\newcommand{\picturefrompdf}[7]{
	\begin{center}
		%trim option's parameter order: left bottom right top
		\includegraphics[page=#2,width=#7,clip,trim = #3 #4 #5 #6]{#1} 
	\end{center}
}

\newcommand{\rto}{$\Rightarrow$ }

\newcommand{\emp}[1]{{\color{orange}{#1}}}

\newcommand{\img}[2]{
	\centering
	\includegraphics[#2]{figures/#1}
}

\newcommand{\questions}{
\begin{frame}
	
	\centering
	\Large
	Fragen?

\end{frame}
}

\newenvironment{task}[1]{
	\begin{block}{\textbf{Aufgabe} \hfill #1}
}{
	\end{block}
}

\AtBeginSection[]
{
	\begin{frame}{Plan für heute}
		\tableofcontents[currentsection]
	\end{frame}
}

\usepackage[uni,footuni,headlogo]{./unirostock/beamerthemeRostock}
% erster Parameter: Farbschema
%        moegliche Werte (nomen est omen): uni, inf, msf, ief, mnf, mef, juf, wsf, auf, thf, phf
%        Standardwert: uni
% zweiter Parameter: Fusszeile
%        moegliche Werte: footuni (Standard) - Fusszeile nach Handbuch des CD
%                         foottitle          - Autor und Title in der Fusszeile
%                         footheadings       - lebende Ueberschriften in der Fusszeile
%                         footuniheadings    - Autor und Uni sowie lebende Ueberschriften in der Fusszeile
% dritter Parameter: Kopfzeile
%        moegliche Werte: headlogo (Standard)- Inhalt von \mylogo in der Kopfzeile
%                         headtitle          - Vortragstitel
%                         headframetitle     - Folientitel
%                         headframesubtitle  - Folientitel und -untertitel
%        bitte nicht mehrere Varianten gleichzeitig angeben :-)

\setbeamercovered{invisible}
\addtobeamertemplate{block begin}{}{\justifying}

%%%%%%%%%%%% Festlegung der Titelseite %%%%%%%%%%%%%%%%%%%%%%%%%%%%%%%%%
\title[]{Imperative Programmierung}

\subtitle{Übung}

\author{\textsc{Tom Warnke}}

\institute{Universität Rostock, Institut für Informatik}

\titlegraphic{}

% ein alternatives Titelbild kann mittels 
%\titleimage{Dateiname.xyz}
% angegeben werden (auf vernuenftiges Seitenformat und Kontrastwerte achten, Skalierung und Abschneiden der oberen rechten Ecke passieren automatisch)
\titleimage{}

%%%%%%%%%%%% Festlegungen fuer Kopf- und Fusszeile %%%%%%%%%%%%%%%%%%%%
% Institutsname f\"ur die Fusszeile (nur wenn bei Paketeinbindung 'footuni' angegeben ist)
\footinstitute{Fakultät für Informatik und Elektrotechnik, Institut für Informatik}
% eigenes Logo oben rechts hinzufuegen (bitte auf vernuenftiges Format achten - ein zu hohes Logo verschiebt das Layout)
\renewcommand{\mylogo}{}

%%%%%%%%%%%%%%%%%%%%%%%%%%%%%%%%%%%%%%%%%%%%%%%%%%%%%%%%%%%%%%%%%%%%%%%

\date{}

%%%%%%%%%%%%%%%%%%%%%%%%%%%%%%%%%%%%%%%%%%%%%%%%%%%%%%%%%%%%%%%%%%%%%%%
\begin{document}

\maketitle

\begin{frame}{Anfang}

	\begin{itemize}
		\item Fragen?
		\item SVN-Probleme?
		\item Hausaufgaben angefangen?
	\end{itemize}

\end{frame}

\begin{frame}{Plan für heute}

	\tableofcontents

\end{frame}

\section{Nachbereitung Hausaufgabe 6}

\begin{frame}[fragile]{Lösungsvorschlag}{Aufgabe 6.1}

	\begin{lstlisting}[gobble=4]
		int RX(int k, int a, int b, int R[N][N]) {
			if(a == b) { /* a = b */
				return 1;
			} else if(k > 0) { /* OR: exists c : aRc and cRXb */
				int c;
				for(c=0;c<N;c++) {
					if(R[a][c] && RX(k-1,c,b,R)) return 1;
				}
			}
			return 0;
		}
		
		void tclosure(int T[N][N], int R[N][N]) {
			int i, j;
		
			for(i=0;i<N;i++)
				for(j=0;j<N;j++)
					T[i][j] = RX(N,i,j,R);
		}
	\end{lstlisting}

\end{frame}

\begin{frame}[fragile]{Lösungsvorschlag}{Aufgabe 6.2}

	\begin{lstlisting}[gobble=4]
		/* a..z + SPACE + COLON + STOP */
		
		#define N 29
		#define C_SPACE 26
		#define C_COMMA 27
		#define C_STOP  28
		
		char chars[]="abcdefghijklmnopqrstuvwxyz ,.";
		
		double Pc_ab[N][N][N]; /* wie wahrscheinlich ist das Triple a b c */
		double F_ab[N][N]; /* wie oft tritt das Paar a b nacheinander auf */
		double F; /* wie viele Triples verarbeitet */
	\end{lstlisting}

\end{frame}

\begin{frame}[fragile]{Lösungsvorschlag}{Aufgabe 6.2}

	\begin{lstlisting}[gobble=4]
		/* einzelnes Zeichen holen und entsprechenden chars[] index zurueckgeben */
		int getcc() {
			int c, haveSpace = 0;
			while(isspace(c=getchar())) haveSpace = 1;
			if(haveSpace) return (ungetc(c,stdin),C_SPACE);
			else if(c>='a' && c<='z') return c-'a';
			else if(c>='A' && c<='Z') return c-'A';
			else if(c==',') return C_COMMA;
			else if(c=='.') return C_STOP;
			else if(c==EOF) return EOF;
			else return getcc();
		}
	\end{lstlisting}

\end{frame}

\begin{frame}[fragile]{Lösungsvorschlag}{Aufgabe 6.2}

	\begin{lstlisting}[gobble=4]
		/* Wahrscheinlichkeitsmatrix erstellen */
		void fcount() {
			int a, b, c;
		
			/* Paare / Triples zaehlen */
			for(a=getcc(), b=getcc();
				c=getcc(), c!=EOF;
				Pc_ab[a][b][c]++, F_ab[a][b]++, F++, a=b, b=c) ;
		
			fprintf(stderr,"%f triples collected\n",F);
		
			for(a=0; a<N; a++)
				for(b=0; b<N; b++)
					if(F_ab[a][b] > 0.0)
						/* Wahrscheinlichkeit fuer Auftreten eines Triples berechnen */
						for(c=0; c<N;c++) Pc_ab[a][b][c] /= F_ab[a][b];
		}
	\end{lstlisting}

\end{frame}

\begin{frame}[fragile]{Lösungsvorschlag}{Aufgabe 6.2}

	\begin{lstlisting}[gobble=4]
		/* Zufallstext mit n Zeichen erzeugen*/
		void generate(int n) {
			int a, b, c;
			double s, r;
		
			for(a=C_STOP, b=C_SPACE; n; a=b, b=c, n--) {
		
				/* Zeichen mithilfe einer Zufallszahl aus Matrix auswaehlen*/
				for(s=0, r=drand48(), c=0;
					s += Pc_ab[a][b][c], s<r;
					c++) ;
				
				/* ausgewaehltes Zeichen ausgeben */
				putchar(chars[c]);
			}
		}
	\end{lstlisting}

\end{frame}

\begin{frame}[fragile]{Lösungsvorschlag}{Aufgabe 6.2}

	\begin{lstlisting}[gobble=4]
		int main(int argc, char *argv[]) {
			fcount();
			srand48(time(NULL));
			
			/* Laenge des zu erzeugenden Zufallstextes als Argument */
			generate(atoi(argv[1]));
			
			printf("\n");
			return 0;
		}
	\end{lstlisting}

\end{frame}

\section{Strukturen, Aufzählungen, Funktionsreferenzen, dynamischer Speicher}

\begin{frame}[fragile]{Definition von Strukturen}
	
	\enquote{Schablone} für Elemente mit mehreren Variablen
	\begin{itemize}
		\item Person (Name, Alter, Geschlecht) \\
			\begin{lstlisting}[gobble=4]
				struct Person {
					char name[20];
					int age;
					char sex;
				};
			\end{lstlisting}
		\item Buch (Titel, Autor) \\
			\begin{lstlisting}[gobble=4]
				struct Book {
					char author[20];
					char title[20];
				};
			\end{lstlisting}
	\end{itemize}

\end{frame}

\begin{frame}[fragile]{Strukturen sind Datentypen}
	
	\begin{itemize}
		\item Variable vom Typ Book anlegen:
			\begin{lstlisting}[gobble=4]
				struct Book faust = {"Goethe", "Faust 1"};
			\end{lstlisting}
		\item Auf Komponenten einer Variable zugreifen:
			\begin{lstlisting}[gobble=4]
				faust.title[6] = '2';
				printf("%s\n", faust.title);
			\end{lstlisting}
		\item Referenz auf eine Variable anlegen
			\begin{lstlisting}[gobble=4]
				struct Book *b = &faust;
			\end{lstlisting}	
		\item Auf Komponenten einer Referenz zugreifen:
			\begin{lstlisting}[gobble=4]
				printf("%s\n", (*b).author);
				printf("%s\n", b->author);
			\end{lstlisting}
						
	\end{itemize}
	
\end{frame}

\begin{frame}[fragile]{Strukturen}{Typdefinition ohne und mit \texttt{typedef}}

	\begin{columns}
		\begin{column}{.4\textwidth}
			\begin{lstlisting}[gobble=8]
				struct Student {
					int nClasses;
					char name[20];
					double gpa; 
				} joe;
				
				struct Student sue;
				
				sue.gpa = 5.0;
			\end{lstlisting}
		\end{column}
		\begin{column}{.4\textwidth}
			\begin{lstlisting}[gobble=8]
				typedef struct {
					int nClasses;
					char name [ 20 ];
					double gpa; 
				} Student;
				
				Student joe, sue;
				
				sue.gpa = 5.0;
			\end{lstlisting}
		\end{column}
	\end{columns}
	
\end{frame}

\begin{frame}[fragile]{Aufzählungstypen}{Typdefinition ohne und mit \texttt{typedef}}

	\begin{columns}
		\begin{column}{.4\textwidth}
			\begin{lstlisting}[gobble=8]
				enum suits {
					CLUBS, 
					HEARTS, 
					SPADES, 
					DIAMONDS
				} card1;
					
				enum suits card2;
			\end{lstlisting}
		\end{column}
		\begin{column}{.4\textwidth}
			\begin{lstlisting}[gobble=8]
				typedef enum {
					CLUBS, 
					HEARTS, 
					SPADES, 
					DIAMONDS
				} suit;
					
				suit card1, card2;
			\end{lstlisting}
		\end{column}
	\end{columns}
	
\end{frame}

\begin{frame}[fragile]{Funktionsreferenzen}

	\begin{lstlisting}[gobble=4]
		int times_two(int n) {
			return 2 * n;
		}
		
		void apply(int arr[], int len, int (*fun)(int)) {
			int i;
			for(i = 0; i < len; i++)
				arr[i] = fun(arr[i]);
		}
			
		int main() {
			int nums[] = {1, 3, 5, 7};
			apply(nums, 4, times_two);
			return 0;
		}
	\end{lstlisting}

\end{frame}

\begin{frame}[fragile]{Dynamischer Speicher}

	\begin{lstlisting}[gobble = 4]
		#include <stdlib.h>
		#include <stdio.h>
		
		int main(int argc, char *argv[]) {
			
			int i, size;
			int *arr;
			
			if (argc > 1)
				size = atoi(argv[1]);
			else 
				return 1;
			
			arr = (int *) malloc(size * sizeof(int)); /* Speicher allokieren */
			
			for(i = 0; i < size; i++)
				arr[i] = 0;
			
			free(arr); /* Speicher freigeben */
			
			return 0;
		}
	\end{lstlisting}

\end{frame}

\section{Programmieren}

\begin{frame}{Programmieren}

	\begin{task}{~}
		Lösen Sie die Aufgaben auf dem Arbeitsblatt \texttt{handout10.pdf}.
	\end{task}

\end{frame}

\section{Abschluss}

\begin{frame}[fragile]{Lösungsvorschlag}{Aufgabe 1.1}
	
	\begin{lstlisting}[gobble=4]
		#include <stdio.h>
				
		void swap(double **x, double **y) {
			double *tmp = *x;
			*x = *y;
			*y = tmp;
		}
	\end{lstlisting}
	
\end{frame}

\begin{frame}[fragile]{Lösungsvorschlag}{Aufgabe 1.2}
	
	\begin{lstlisting}[gobble=4]
		#include <stdio.h>
		#include <stdlib.h>
		
		typedef struct {
			char *author;
			char *title;
		} Book, *BookRef;
		
		int main () {
			BookRef a;
			int lauthor, ltitle;
			a = (BookRef) malloc(sizeof(Book));
			printf("Wie lang ist der Name des Autors? ");
			scanf("%d", &lauthor);
			a->author = (char *) malloc(sizeof(char)*(lauthor+1));
			printf("Wie ist der Name des Autors? ");
			scanf("%s", a->author);
			/* title einlesen analog */
			printf("%s - %s\n", a->author, a->title);
			free(a);
		} 
	\end{lstlisting}
	
\end{frame}

\begin{frame}[fragile]{Lösungsvorschlag}{Aufgabe 2.1}
	
	\begin{lstlisting}[gobble=4]
		#include <stdio.h> 
		#include <stdlib.h>
		
		typedef struct _l {
			int value;
			struct _l *next;
		} Elem, *List;
	\end{lstlisting}
	
\end{frame}

\begin{frame}[fragile]{Lösungsvorschlag}{Aufgabe 2.2}
	
	\begin{lstlisting}[gobble=4]
		List insert(List l, int v) {
			List new = (List) malloc(sizeof(Elem));
			new->value = v;
			new->next = l;
			return new;
		}
	\end{lstlisting}
	
\end{frame}

\begin{frame}[fragile]{Lösungsvorschlag}{Aufgabe 2.3}
	
	\begin{lstlisting}[gobble=4]
		void print(List l) {
			List runner = l;
			while(runner != NULL) {
				printf("%d ", runner->value);
				runner = runner->next;
			}
			printf("\n");
		}
	\end{lstlisting}
	
\end{frame}

\begin{frame}[fragile]{Lösungsvorschlag}{Aufgabe 2.4}
	
	\begin{lstlisting}[gobble=4]
		List map(List l, int (*f)(int)) {
			if(l == NULL) 
				return NULL;
			else {
				List mapped_tail, mapped_list;
				mapped_tail = map(l->next, f);
				mapped_list = insert(mapped_tail, f(l->value));
				return mapped_list;
			}
		}
	\end{lstlisting}
	
\end{frame}



	% end of presentation
	%%%%%%%%%%%%%%%%%%%%%%%%%%%%%%%%%%%%%%%%%%%%%%%%%%%%%%%%%%%%%%%%%%%%%%%
	% backup slides
	
	\appendix
	\newcounter{finalframe}
	\setcounter{finalframe}{\value{framenumber}}
	
	%%%%%%%%%%%%%%%%%%%%%%%%%%%%%%%%%%%%%%%%%%%%%%%%%%%%%%%%%%%%%%%%%%%%%%%
	% end of backup slides
	\setcounter{framenumber}{\value{finalframe}}
\end{document}

