% !TeX spellcheck = de_DE
\documentclass[]{article}

\usepackage[utf8]{inputenc}
\usepackage[T1]{fontenc}
\usepackage[scaled]{beramono}
\usepackage{amssymb}
\usepackage{amsmath}
\usepackage{graphicx}
\usepackage[top=2cm, bottom=2cm, left=2cm, right=4cm]{geometry}
\usepackage{csquotes}
\usepackage{listings}
\usepackage{hyperref}
\usepackage{booktabs}

\lstset{
	tabsize=2,
	language=Java,
	showstringspaces=false,
	basicstyle=\footnotesize\ttfamily,
}

%no indentation
\setlength\parindent{0pt}

\pagestyle{empty}

\begin{document}

	\section{Rekursion}
	
	\subsection{Ausgabe}
	
	Betrachten Sie die folgenden Programme zunächst ohne sie auszuführen.
	
	\noindent
	\begin{minipage}[t]{0.5\linewidth} %1
		\begin{lstlisting}[gobble=6]
			#include <stdio.h>
			
			void rec(int i) {
				if (i > 0) {
					printf("%d\n", i);
					rec(i - 1);
				}
			}
			
			int main() {
				rec(3);
				return 0;
			}
		\end{lstlisting}
	\end{minipage}
	\quad
	\begin{minipage}[t]{0.5\linewidth} %2
		\begin{lstlisting}[gobble=6]
			#include <stdio.h>
			
			void rec(int i) {
				if (i > 0) {
					rec(i - 1);
					printf("%d\n", i);
				}
			}
			
			int main() {
				rec(3);
				return 0;
			}
		\end{lstlisting}
	\end{minipage}
	
	\begin{enumerate}
		\item Simulieren Sie die Ausführung der Programme.
			Wie oft wird \texttt{rec} jeweils aufgerufen? 
			Mit welchen Argumenten?
			Was geben die Programme aus? 
		\item Überprüfen Sie Ihre Ergebnisse, indem Sie die Programme kompilieren und ausführen.
	\end{enumerate}
	
	\subsection{Quersumme}
	
	Schreiben Sie ein \textbf{rekursives} Programm, das die Quersumme einer Zahl berechnet.
	
	\subsection{Umrechnung ins Oktalsystem}
	
	Schreiben Sie ein \textbf{rekursives} Programm, das eine positive ganze Zahl in das oktale Zahlensystem übersetzt.
	Dabei sollen die Oktalstellen richtiger Reihenfolge ausgegeben werden, so dass Sie etwa für die Zahl $13_{10}$ die Ausgabe $15_8$ erhalten.
	
\end{document}