% !TeX spellcheck = de_DE
\documentclass[a4paper]{article}

\usepackage[utf8]{inputenc}
\usepackage[T1]{fontenc}
\usepackage[scaled]{beramono}
\usepackage{amssymb}
\usepackage{amsmath}
\usepackage{graphicx}
\usepackage[top=2cm, bottom=2cm, left=2cm, right=4cm]{geometry}
\usepackage{csquotes}
\usepackage{listings}
\usepackage{hyperref}
\usepackage{booktabs}

\lstset{
	tabsize=2,
	language=Java,
	showstringspaces=false,
	basicstyle=\footnotesize\ttfamily,
}

%no indentation
\setlength\parindent{0pt}

\pagestyle{empty}

\begin{document}

	\section{Funktionen}
	
	Verwenden Sie ab jetzt für alle Programmieraufgaben beim Kompilieren mit \texttt{gcc} die Kommandozeilenparameter \texttt{-Wall -pedantic} (z.B. \texttt{gcc -Wall -pedantic xyz.c -o xyz})
	
	\subsection{Funktionsaufrufe}
	
	Notieren Sie, welche Funktionen mit welchen Argumenten in welcher Reihenfolge aufgerufen werden.
	Beginnen Sie mit \texttt{main()}.
	\vspace{3ex}
	
	\noindent\begin{minipage}{0.6\linewidth}
			\begin{lstlisting}[gobble=8]
				#include <stdio.h>
						
				int square(int n) {
					return n * n;
				}
				
				void print_with_square(int number) {
					int s;
					s = square(number);	
					printf("The square of %d is %d\n", number, s);
				}
				
				int main() {	
					print_with_square(3);
					print_with_square(5);
					
					return 0;
				}
			\end{lstlisting}
		\end{minipage}
	
	\subsection{Alphabet}
	
	Schreiben Sie eine Funktion, die für ein gegebenes Zeichen eine Zahl zurückgibt.
	Falls das Zeichen ein Buchstabe ist (groß oder klein), soll zurückgegeben werden, der wievielte Buchstabe im Alphabet es ist.
	Falls das Zeichen kein Buchstabe ist, soll \texttt{-1} zurückgegeben werden.
	
	\clearpage
	
	\section{Felder}
	
	\subsection{Summe}
	
	Schreiben Sie eine Funktion \texttt{array\_sum}, die die Summe der Elemente eines \texttt{double}-Feldes berechnet und zurückgibt.
	Der Funktion wird neben dem Feld auch die Anzahl der Elemente im Feld als \texttt{int} übergeben.
	Nutzen Sie eine \texttt{for}-Schleife.
	Beispielsweise soll das Programm links die Ausgabe rechts liefern:
	
	\noindent\begin{minipage}[t][][b]{0.6\linewidth}
	\begin{lstlisting}[gobble=4]
		int main() {
			double ns[] = {3.0, 4.0, 5.0, 0.001};
			
			printf("Summe: %f\n", array_sum(ns, 4));
			
			return 0;
		}
	\end{lstlisting}
	\end{minipage}\qquad
	\begin{minipage}[t][][b]{0.3\linewidth}
	\begin{lstlisting}[gobble=4]
		Summe: 12.001000
	\end{lstlisting}
	\end{minipage}
	
	\subsection{Ausgabe}
	
	Schreiben Sie eine Funktion \texttt{print\_array}, die ein zweidimensionales Feld der Größe 3x3 auf dem Bildschirm ausgibt.
	Die Elemente des Feldes sind ganze Zahlen.
	Beispielsweise soll das Programm links die Ausgabe rechts liefern:
	
	\noindent\begin{minipage}[t][][b]{0.6\linewidth}
	\begin{lstlisting}[gobble=4]
		int main() {
			int arr[3][3] = { 
				{1, 2, 3},
				{4, 5, 6},
				{7, 8, 9} };
				
			print_array(arr);
			
			return 0;
		}
	\end{lstlisting}
	\end{minipage}\qquad
	\begin{minipage}[t][][b]{0.3\linewidth}
	\begin{lstlisting}[gobble=4]
		1     2     3
		4     5     6
		7     8     9
	\end{lstlisting}
	\end{minipage}
	
	\noindent Versuchen Sie, \texttt{print\_array} mit Arrays anderer Größen aufzurufen.
	Können Sie unerwünschtes Verhalten im Programm hervorrufen?
	
	\subsection{Textstatistik}
	
	Schreiben Sie ein Programm, das die Häufigkeit aller Buchstaben in seiner Eingabe zählt.
	Sie können ein \texttt{int}-Feld der Größe \texttt{26} verwenden, in dem der $x$-te Wert des Felds die Häufigkeit des $x$-ten Buchstaben ist.
	Lesen Sie die Eingabe Zeichen für Zeichen ein und erhöhen sie den entsprechenden Wert, wenn Sie einen Buchstaben einlesen.
	Geben Sie am Ende die Anzahl aller Buchstaben aus.
	\\\textit{Hinweis: Nutzen Sie Ihre Funktion aus der Aufgabe \enquote{Alphabet}.}
	
\end{document}