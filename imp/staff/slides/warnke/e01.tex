% !TeX spellcheck = de_DE
%% 
%% This is file 'beamer_sample.tex'
%% 	
%% by Mathias Winkel
%%
%% Problems, bugs and comments to 
%% mathias.winkel2@uni-rostock.de
%%
\RequirePackage{fix-cm} % because of font size substituion warnings
\documentclass[10pt]{beamer} % die 10pt sollten festgelegt bleiben, da dies die Groesse der Mathematikschrift etc. beeinflusst

\usepackage[ngerman]{babel}  % deutsche Bezeichnungen und Trennung etc
\usepackage[utf8]{inputenc}
\usepackage[T1]{fontenc}
\usepackage{csquotes}
\usepackage{hyperref}        % interne Hyperlinks
\usepackage[scaled]{beramono}
\usepackage{microtype}
\usepackage{listings}
\usepackage{ragged2e}
\usepackage{marvosym}
\usepackage{xcolor}
\usepackage{pgffor}

%******************************************
% For giving image sources, see http://tex.stackexchange.com/a/48485
\usepackage[absolute,overlay]{textpos}
\setbeamercolor{framesource}{fg=gray}
\setbeamerfont{framesource}{size=\tiny}

\setbeamercolor{plain}{fg=black,bg=white}

\newcommand{\source}[1]{\begin{textblock*}{4cm}(8.2cm,8.6cm)
		\begin{beamercolorbox}[ht=0.5cm,right]{framesource}
			\usebeamerfont{framesource}\usebeamercolor[fg]{framesource} Quelle: {#1}
		\end{beamercolorbox}
	\end{textblock*}}
%******************************************

\lstdefinestyle{customc}{
	tabsize=2,
	belowcaptionskip=1\baselineskip,
	breaklines=true,
	xleftmargin=\parindent,
	language=C,
	showstringspaces=false,
	basicstyle=\scriptsize\ttfamily,
	keywordstyle=\bfseries\color{green!40!black},
	commentstyle=\itshape\color{purple!40!black},
	identifierstyle=\color{blue},
	stringstyle=\color{orange},
}

\lstset{escapechar=@,style=customc}

\definecolor{links}{HTML}{2A1B81}
\hypersetup{colorlinks,linkcolor=,urlcolor=structure.fg}

\newcommand{\framefrompdf}[2]{
	\begin{frame}[plain]
		\hspace*{-8.5pt}
		\begin{beamercolorbox}[wd=\paperwidth,ht=\paperheight]{plain}
			\makebox[\paperwidth]{\hspace{8.5pt}\includegraphics[page=#2,width=\paperwidth]{#1}}
		\end{beamercolorbox}
	\end{frame}
}

\newcommand{\framesfrompdf}[3]{
	\foreach \n in {#2,...,#3}{\framefrompdf{#1}{\n}}
}

\newcommand{\framesfrompdfclipped}[8]{
	\foreach \n in {#2,...,#3}{
		\begin{frame}[plain]
			\vspace*{-18pt}\hspace*{-8.5pt}
			\makebox[\linewidth]{\includegraphics[page=\n,width=#8,clip,trim = #4 #5 #6 #7]{#1}}
		\end{frame}
	}
}

\newcommand{\picturefrompdf}[7]{
	\begin{center}
		%trim option's parameter order: left bottom right top
		\includegraphics[page=#2,width=#7,clip,trim = #3 #4 #5 #6]{#1} 
	\end{center}
}

\newcommand{\rto}{$\Rightarrow$ }

\newcommand{\emp}[1]{{\color{orange}{#1}}}

\newcommand{\img}[2]{
	\centering
	\includegraphics[#2]{figures/#1}
}

\newcommand{\questions}{
\begin{frame}
	
	\centering
	\Large
	Fragen?

\end{frame}
}

\newenvironment{task}[1]{
	\begin{block}{\textbf{Aufgabe} \hfill #1}
}{
	\end{block}
}

\AtBeginSection[]
{
	\begin{frame}{Plan für heute}
		\tableofcontents[currentsection]
	\end{frame}
}

\usepackage[uni,footuni,headlogo]{./unirostock/beamerthemeRostock}
% erster Parameter: Farbschema
%        moegliche Werte (nomen est omen): uni, inf, msf, ief, mnf, mef, juf, wsf, auf, thf, phf
%        Standardwert: uni
% zweiter Parameter: Fusszeile
%        moegliche Werte: footuni (Standard) - Fusszeile nach Handbuch des CD
%                         foottitle          - Autor und Title in der Fusszeile
%                         footheadings       - lebende Ueberschriften in der Fusszeile
%                         footuniheadings    - Autor und Uni sowie lebende Ueberschriften in der Fusszeile
% dritter Parameter: Kopfzeile
%        moegliche Werte: headlogo (Standard)- Inhalt von \mylogo in der Kopfzeile
%                         headtitle          - Vortragstitel
%                         headframetitle     - Folientitel
%                         headframesubtitle  - Folientitel und -untertitel
%        bitte nicht mehrere Varianten gleichzeitig angeben :-)

\setbeamercovered{invisible}
\addtobeamertemplate{block begin}{}{\justifying}

%%%%%%%%%%%% Festlegung der Titelseite %%%%%%%%%%%%%%%%%%%%%%%%%%%%%%%%%
\title[]{Imperative Programmierung}

\subtitle{Übung}

\author{\textsc{Tom Warnke}}

\institute{Universität Rostock, Institut für Informatik}

\titlegraphic{}

% ein alternatives Titelbild kann mittels 
%\titleimage{Dateiname.xyz}
% angegeben werden (auf vernuenftiges Seitenformat und Kontrastwerte achten, Skalierung und Abschneiden der oberen rechten Ecke passieren automatisch)
\titleimage{}

%%%%%%%%%%%% Festlegungen fuer Kopf- und Fusszeile %%%%%%%%%%%%%%%%%%%%
% Institutsname f\"ur die Fusszeile (nur wenn bei Paketeinbindung 'footuni' angegeben ist)
\footinstitute{Fakultät für Informatik und Elektrotechnik, Institut für Informatik}
% eigenes Logo oben rechts hinzufuegen (bitte auf vernuenftiges Format achten - ein zu hohes Logo verschiebt das Layout)
\renewcommand{\mylogo}{}

%%%%%%%%%%%%%%%%%%%%%%%%%%%%%%%%%%%%%%%%%%%%%%%%%%%%%%%%%%%%%%%%%%%%%%%

\date{15.10.2019} % Hier kann das Datum des Vortrages festgelegt werden (fuer Fusszeile etc.)

%%%%%%%%%%%%%%%%%%%%%%%%%%%%%%%%%%%%%%%%%%%%%%%%%%%%%%%%%%%%%%%%%%%%%%%
\begin{document}

\maketitle  

\begin{frame}{Plan für heute}

	\tableofcontents

\end{frame}

\section{Kennenlernen}

\begin{frame}{Wer ich bin}
	
	\begin{block}{M.Sc. Tom Warnke (alias \enquote{Du})}
		Wissenschaftlicher Mitarbeiter am \href{https://mosi.informatik.uni-rostock.de}{Lehrstuhl für Modellierung und Simulation}\\
		\MVAt~ \href{mailto:tom.warnke@uni-rostock.de}{tom.warnke@uni-rostock.de}\\
		\Mundus~ \href{https://mosi.informatik.uni-rostock.de/team/staff/tom-warnke/}{https://mosi.informatik.uni-rostock.de/team/staff/tom-warnke/}\\
		\Letter~ \href{https://www.informatik.uni-rostock.de/ueber_uns/konrad_zuse_haus/}{Konrad-Zuse-Haus}, Raum 228\\
		\Telefon~ 0381/498-7609\\
	\end{block}
	
\end{frame}

\begin{frame}{Wer Sie sind}{Standpunkte}
	
	\begin{task}{}
		Die Rückwand des Raums und das Whiteboard stellen je eine Antwort auf die gestellte Frage dar.
		Drücken Sie Ihren Standpunkt aus, indem Sie sich an den Enden des Raums oder dazwischen positionieren.
		
		\only<2->{\vspace{3ex}}
		
		\only<2>{Wie viel Informatik hatten Sie in der Schule?\\[1ex] Rückwand = Leistungskurs/Hauptfach o. ä.\\Whiteboard = Gar nicht}
		\only<3>{Was für ein Betriebssystem nutzen Sie?\\[1ex] Rückwand = Linux\\Whiteboard = Windows}
		\only<4>{Wie viel Erfahrung haben Sie im Programmieren?\\[1ex] Rückwand = Gar keine\\Whiteboard = Sehr viel}
		\only<5>{Wie vertraut sind Sie mit Arbeit auf der Kommandozeile (Linux, Windows)?\\[1ex] Rückwand = Gar nicht\\Whiteboard = Sehr}
		\only<6>{Wie vertraut sind Sie mit Arbeit in einer IDE (Eclipse, Visual Studio, ...)?\\[1ex] Rückwand = Gar nicht\\Whiteboard = Sehr}
		\only<7>{Was wissen Sie über Versionskontrollsysteme (SVN, git)?\\[1ex] Rückwand = Was ist das?\\Whiteboard = Benutze ich täglich}
	\end{task}
	
\end{frame}

\section{Organisatorisches}

\begin{frame}{Wann und Wo}

	\begin{itemize}
		\item Jede Woche Mittwoch \emp{Vorlesung}
		\item Jede Woche \emp{Übung} hier im Raum 310
		\begin{itemize}
			\item Gruppe 1: Dienstag, 9:15--10:45
			\item Gruppe 2: Freitag, 11:15--12:45
		\end{itemize}
		\item Übungsaufgaben am Rechner und auf Papier (Stift mitbringen)
		\item Getestete Referenzssoftware gibt es
		\begin{itemize}
			\item auf den Rechnern hier im Raum 310
			\item auf dem \href{https://www.itmz.uni-rostock.de/onlinedienste/anwendungsserver-des-itmz/anwendungsserver-des-itmz/}{Anwendungsserver unicomp}
		\end{itemize}
		\item Alternativ die Software auf dem eigenen Rechner installieren
		\item Folien werden nach Vorlesung/Übung im Stud.IP hochgeladen
	\end{itemize}

\end{frame}

\begin{frame}{Prüfungszulassung}
	
	\begin{itemize}
		\item \emp{Hausaufgabenblätter} werden bei Stud.IP hochgeladen
		\item Die Aufgaben werden von in Gruppen bearbeitet und abgegeben
		\item Sie bekommen die abgegebenen Hausaufgaben korrigiert zurück
		\item Sie werden zur Prüfung zugelassen, wenn Sie in den Hausaufgaben mindestens 50\% der Punkte erreichen
	\end{itemize}
	
	(Weitere Informationen in der Vorlesung)
	
\end{frame}

\section{Erste Schritte in C}

\begin{frame}{Ein erstes Programm}
	
	\begin{itemize}
		\item Wir nutzen die Programmiersprache C
		\item Arbeitsablauf:
		\begin{enumerate}
			\item Quellcode schreiben
			\item Programm kompilieren
			\begin{itemize}
				\item Fehler? \rto zurück zu 1
			\end{itemize}
			\item Programm ausführen
			\begin{itemize}
				\item Fehler? \rto zurück zu 1
			\end{itemize}
		\end{enumerate}
	\end{itemize}
	
\end{frame}

\begin{frame}{Die Schritte im Einzelnen}
	
	NotePad++ 
	\begin{itemize}
		\item Nutzen wir, um Quellcode zu schreiben
		\item Bietet Unterstützung für Programmiersprachen
		\item Syntax Highlighting \& -Folding, Auto-completion, ...
	\end{itemize}	
	
	Alternativen: vim, emacs, \dots
	
\end{frame}

\begin{frame}[fragile]{Die Schritte im Einzelnen}
	
	Programmieren mit C
	\begin{itemize}
		\item In den frühen 70er Jahren entwickelt
		\item Grundlage fast aller modernen imperativen Programmiersprachen
	\end{itemize}	
	
	\begin{lstlisting}[gobble=4]
		#include <stdio.h> /* Bibliothek laden */
		
		main() { /* Hauptfunktion */
			printf("Hello World!\n"); /* Ausgabe */
		}
	\end{lstlisting}
	
\end{frame}

\begin{frame}{Die Schritte im Einzelnen}
	
	MinGW
	\begin{itemize}
		\item Kommandozeile als Nutzerschnittstelle
		\item Autovervollständigung mit \texttt{TAB}
		\item Befehlshistorie mit $\uparrow$
		\item Navigieren durch Verzeichnisstruktur, Starten von Programmen
		\item \texttt{\$ pwd}: zeigt das aktuelle Verzeichnis
		\item \texttt{\$ ls}: zeigt Inhalt des aktuellen Verzeichnisses
		\item \texttt{\$ cd} \textsl{dir}: wechselt in das angegebene Verzeichnis \textsl{dir}
	\end{itemize}	
	
\end{frame}

\begin{frame}[fragile]{Die Schritte im Einzelnen}
	
	GCC
	\begin{itemize}
		\item GNU Compiler Collection
		\item Starten über MinGW
		\item \verb|$ gcc input.c -o output.exe|: kompiliert die Quelldatei \texttt{input.c} zum ausführbaren Programm \texttt{output.exe}
		\item \verb|$ gcc --help|: Liste weiterer Parameter
	\end{itemize}	
	
\end{frame}

\section{Abschluss}

\begin{frame}{Abschluss}
	
	\begin{itemize}
		\item Bei Problemen: \href{mailto:tom.warnke@uni-rostock.de}{Mail schreiben}
		\item Programmieren lernt man nur durch Programmieren
		\item Jedes Problem eines Programmieranfängers wurde schon mal gelöst\\(Die Welt ist eine Google)
	\end{itemize}
	
	Linkliste zur Inspiration:
	\begin{itemize}
		\item \href{https://www.youtube.com/watch?v=nKIu9yen5nc}{Youtube-Video von Code.org}
		\item \href{http://antrikshy.com/blog/how-i-got-started-with-programming-side-projects}{How I Got Started With Programming Side Projects}
	\end{itemize}
	
\end{frame}

	% end of presentation
	%%%%%%%%%%%%%%%%%%%%%%%%%%%%%%%%%%%%%%%%%%%%%%%%%%%%%%%%%%%%%%%%%%%%%%%
	% backup slides
	
	\appendix
	\newcounter{finalframe}
	\setcounter{finalframe}{\value{framenumber}}
	
	%%%%%%%%%%%%%%%%%%%%%%%%%%%%%%%%%%%%%%%%%%%%%%%%%%%%%%%%%%%%%%%%%%%%%%%
	% end of backup slides
	\setcounter{framenumber}{\value{finalframe}}
\end{document}

