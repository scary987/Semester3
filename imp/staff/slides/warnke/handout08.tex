% !TeX spellcheck = de_DE
\documentclass[]{article}

\usepackage[utf8]{inputenc}
\usepackage[T1]{fontenc}
\usepackage[scaled]{beramono}
\usepackage{amssymb}
\usepackage{amsmath}
\usepackage{graphicx}
\usepackage[top=2cm, bottom=2cm, left=2cm, right=4cm]{geometry}
\usepackage{csquotes}
\usepackage{listings}
\usepackage{hyperref}
\usepackage{booktabs}

\lstset{
	tabsize=2,
	language=Java,
	showstringspaces=false,
	basicstyle=\footnotesize\ttfamily,
}

%no indentation
\setlength\parindent{0pt}

\pagestyle{empty}

\begin{document}

	\section{Iteration und Rekursion}
	
	\subsection{Schleifenarten und Rekursion}
	
	Schreiben Sie vier Programme, die alle durch 3 teilbaren Zahlen von 3 bis 30 rückwärts ausgeben.
	Nutzen Sie je einmal \texttt{for}-, \texttt{while}- und \texttt{do}-\texttt{while}-Schleife sowie Rekursion.
	
	\begin{minipage}[t][][b]{0.3\linewidth}
	\begin{lstlisting}[gobble=4]
		30
		27
		24
		21
		18
		15
		12
		9
		6
		3
	\end{lstlisting}
	\end{minipage}
	
	\subsection{Summe in einem Feld}
	
	Schreiben Sie je eine iterative \emph{und} eine rekursive Funktion, die die Summe der Zahlen in einem Feld bestimmt.
	Die Funktion bekommt neben dem Feld auch die Länge des Feldes übergeben.
	Beispielsweise soll das Programm links die Ausgabe rechts liefern:
	
	\begin{minipage}[t][][b]{0.6\linewidth}
	\begin{lstlisting}[gobble=4]
		int main() {
			int three_nums[] = {1, 42, 3};
			int empty_array[0];
			printf("%d\n", array_sum_rec(three_nums, 3));
			printf("%d\n", array_sum_iter(empty_array, 0));
			  
			return 0;
		}
	\end{lstlisting}
	\end{minipage}\qquad
	\begin{minipage}[t][][b]{0.3\linewidth}
	\begin{lstlisting}[gobble=4]
		46
		0
	\end{lstlisting}
	\end{minipage}
		
	\subsection{Maximum in einem Feld}
	
	Schreiben Sie je eine iterative \emph{und} eine rekursive Funktion, die die größte Zahl in einem Feld von positiven Zahlen bestimmt.
	Die Funktion bekommt neben dem Feld auch die Länge des Feldes übergeben.
	Für leere Felder soll $-1$ zurückgegeben werden.
	Beispielsweise soll das Programm links die Ausgabe rechts liefern:
	
	\begin{minipage}[t][][b]{0.6\linewidth}
	\begin{lstlisting}[gobble=4]
		int main() {
			int three_nums[] = {1, 42, 3};
			int empty_array[0];
			printf("%d\n", array_max_rec(three_nums, 3));
			printf("%d\n", array_max_iter(empty_array, 0));
			  
			return 0;
		}
	\end{lstlisting}
	\end{minipage}\qquad
	\begin{minipage}[t][][b]{0.3\linewidth}
	\begin{lstlisting}[gobble=4]
		42
		-1
	\end{lstlisting}
	\end{minipage}
	
	\clearpage

	\section{Referenzen}
	
	\subsection{Swap für Referenzen}
	
	Schreiben Sie eine Funktion \texttt{swap}, die die Werte von zwei Referenzen austauscht.
	Wenn also \texttt{a} auf \texttt{x} zeigt und \texttt{b} auf \texttt{y}, soll nach Anwendung von \texttt{swap} \texttt{a} auf \texttt{y} zeigen und \texttt{b} auf \texttt{x}.
	Die Werte von \texttt{x} und \texttt{y} sollen sich dabei nicht verändern.
	Beispielsweise soll das Programm links die Ausgabe rechts liefern:

	\begin{minipage}[t][][b]{0.6\linewidth}
	\begin{lstlisting}[gobble=4]
		int main() {    
			double x = 3.0, y = 5.0;
			double *a = &x, *b = &y;
			printf("x = %0.1f, x = %0.1f, *a = %0.1f, *b = %0.1f\n", 
				x, y, *a, *b);
			swap(&a, &b);
			printf("x = %0.1f, x = %0.1f, *a = %0.1f, *b = %0.1f\n", 
				x, y, *a, *b);
			return 0;
		}
	\end{lstlisting}
	\end{minipage}\qquad
	\begin{minipage}[t][][b]{0.35\linewidth}
	\begin{lstlisting}[gobble=4]
		x = 3.0, y = 5.0, *a = 3.0, *b = 5.0
		x = 3.0, y = 5.0, *a = 5.0, *b = 3.0
	\end{lstlisting}
	\end{minipage}
	
	\subsection{Call by reference}
	
	Schreiben Sie eine Funktion, die ein \texttt{int}-Feld, die Länge des Feldes und Referenzen auf zwei \texttt{int}-Variablen übergeben bekommt.
	Die Funktion bestimmt Minimum und Maximum der Zahlen im Feld und schreibt diese in die übergebenen Referenzen.
	
	
	\subsection{Vergleich von Zeichenketten}
	
	Zeichenketten sind in C als Felder von \texttt{char}s umgesetzt.
	Das Ende eine Zeichenkette wird durch den char-Wert \lstinline|'\0'| signalisiert. 
	Schreiben Sie eine Funktion \texttt{equals}, die zwei Zeichenketten vergleicht.
	Die Funktion nimmt als Parameter zwei Referenzen auf char entgegen und gibt ein int zurück: 1, wenn die Zeichenketten gleich sind und 0, wenn sie sich unterscheiden. 
	Sie können die folgende \texttt{main}-Funktion zum Testen verwenden:
	
	\begin{lstlisting}[gobble=4]
		int main() {
		  
		  char c1[20], c2[20];
		  
		  printf("Enter string 1: ");
		  scanf("%s", c1);
		  printf("Enter string 2: ");
		  scanf("%s", c2);
		  
		  printf("Comparison result: %d\n", equals(c1, c2));
		  
		  return 0;
		}
	\end{lstlisting}

\end{document}