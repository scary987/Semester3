% !TeX spellcheck = de_DE
%% 
%% This is file 'beamer_sample.tex'
%% 	
%% by Mathias Winkel
%%
%% Problems, bugs and comments to 
%% mathias.winkel2@uni-rostock.de
%%
\RequirePackage{fix-cm} % because of font size substituion warnings
\documentclass[10pt]{beamer} % die 10pt sollten festgelegt bleiben, da dies die Groesse der Mathematikschrift etc. beeinflusst

\usepackage[ngerman]{babel}  % deutsche Bezeichnungen und Trennung etc
\usepackage[utf8]{inputenc}
\usepackage[T1]{fontenc}
\usepackage{csquotes}
\usepackage{hyperref}        % interne Hyperlinks
\usepackage[scaled]{beramono}
\usepackage{microtype}
\usepackage{listings}
\usepackage{ragged2e}
\usepackage{marvosym}
\usepackage{xcolor}
\usepackage{pgffor}

%******************************************
% For giving image sources, see http://tex.stackexchange.com/a/48485
\usepackage[absolute,overlay]{textpos}
\setbeamercolor{framesource}{fg=gray}
\setbeamerfont{framesource}{size=\tiny}

\setbeamercolor{plain}{fg=black,bg=white}

\newcommand{\source}[1]{\begin{textblock*}{4cm}(8.2cm,8.6cm)
		\begin{beamercolorbox}[ht=0.5cm,right]{framesource}
			\usebeamerfont{framesource}\usebeamercolor[fg]{framesource} Quelle: {#1}
		\end{beamercolorbox}
	\end{textblock*}}
%******************************************

\lstdefinestyle{customc}{
	tabsize=2,
	belowcaptionskip=1\baselineskip,
	breaklines=true,
	xleftmargin=\parindent,
	language=C,
	showstringspaces=false,
	basicstyle=\scriptsize\ttfamily,
	keywordstyle=\bfseries\color{green!40!black},
	commentstyle=\itshape\color{purple!40!black},
	identifierstyle=\color{blue},
	stringstyle=\color{orange},
}

\lstset{escapechar=@,style=customc}

\definecolor{links}{HTML}{2A1B81}
\hypersetup{colorlinks,linkcolor=,urlcolor=structure.fg}

\newcommand{\framefrompdf}[2]{
	\begin{frame}[plain]
		\hspace*{-8.5pt}
		\begin{beamercolorbox}[wd=\paperwidth,ht=\paperheight]{plain}
			\makebox[\paperwidth]{\hspace{8.5pt}\includegraphics[page=#2,width=\paperwidth]{#1}}
		\end{beamercolorbox}
	\end{frame}
}

\newcommand{\framesfrompdf}[3]{
	\foreach \n in {#2,...,#3}{\framefrompdf{#1}{\n}}
}

\newcommand{\framesfrompdfclipped}[8]{
	\foreach \n in {#2,...,#3}{
		\begin{frame}[plain]
			\vspace*{-18pt}\hspace*{-8.5pt}
			\makebox[\linewidth]{\includegraphics[page=\n,width=#8,clip,trim = #4 #5 #6 #7]{#1}}
		\end{frame}
	}
}

\newcommand{\picturefrompdf}[7]{
	\begin{center}
		%trim option's parameter order: left bottom right top
		\includegraphics[page=#2,width=#7,clip,trim = #3 #4 #5 #6]{#1} 
	\end{center}
}

\newcommand{\rto}{$\Rightarrow$ }

\newcommand{\emp}[1]{{\color{orange}{#1}}}

\newcommand{\img}[2]{
	\centering
	\includegraphics[#2]{figures/#1}
}

\newcommand{\questions}{
\begin{frame}
	
	\centering
	\Large
	Fragen?

\end{frame}
}

\newenvironment{task}[1]{
	\begin{block}{\textbf{Aufgabe} \hfill #1}
}{
	\end{block}
}

\AtBeginSection[]
{
	\begin{frame}{Plan für heute}
		\tableofcontents[currentsection]
	\end{frame}
}

\usepackage[uni,footuni,headlogo]{./unirostock/beamerthemeRostock}
% erster Parameter: Farbschema
%        moegliche Werte (nomen est omen): uni, inf, msf, ief, mnf, mef, juf, wsf, auf, thf, phf
%        Standardwert: uni
% zweiter Parameter: Fusszeile
%        moegliche Werte: footuni (Standard) - Fusszeile nach Handbuch des CD
%                         foottitle          - Autor und Title in der Fusszeile
%                         footheadings       - lebende Ueberschriften in der Fusszeile
%                         footuniheadings    - Autor und Uni sowie lebende Ueberschriften in der Fusszeile
% dritter Parameter: Kopfzeile
%        moegliche Werte: headlogo (Standard)- Inhalt von \mylogo in der Kopfzeile
%                         headtitle          - Vortragstitel
%                         headframetitle     - Folientitel
%                         headframesubtitle  - Folientitel und -untertitel
%        bitte nicht mehrere Varianten gleichzeitig angeben :-)

\setbeamercovered{invisible}
\addtobeamertemplate{block begin}{}{\justifying}

%%%%%%%%%%%% Festlegung der Titelseite %%%%%%%%%%%%%%%%%%%%%%%%%%%%%%%%%
\title[]{Imperative Programmierung}

\subtitle{Übung}

\author{\textsc{Tom Warnke}}

\institute{Universität Rostock, Institut für Informatik}

\titlegraphic{}

% ein alternatives Titelbild kann mittels 
%\titleimage{Dateiname.xyz}
% angegeben werden (auf vernuenftiges Seitenformat und Kontrastwerte achten, Skalierung und Abschneiden der oberen rechten Ecke passieren automatisch)
\titleimage{}

%%%%%%%%%%%% Festlegungen fuer Kopf- und Fusszeile %%%%%%%%%%%%%%%%%%%%
% Institutsname f\"ur die Fusszeile (nur wenn bei Paketeinbindung 'footuni' angegeben ist)
\footinstitute{Fakultät für Informatik und Elektrotechnik, Institut für Informatik}
% eigenes Logo oben rechts hinzufuegen (bitte auf vernuenftiges Format achten - ein zu hohes Logo verschiebt das Layout)
\renewcommand{\mylogo}{}

%%%%%%%%%%%%%%%%%%%%%%%%%%%%%%%%%%%%%%%%%%%%%%%%%%%%%%%%%%%%%%%%%%%%%%%
\usepackage{tikz}

\date{}

%%%%%%%%%%%%%%%%%%%%%%%%%%%%%%%%%%%%%%%%%%%%%%%%%%%%%%%%%%%%%%%%%%%%%%%
\begin{document}

\maketitle

\begin{frame}{Anfang}

	\begin{itemize}
		\item Fragen?
		\item SVN-Probleme?
		\item Hausaufgaben angefangen?
	\end{itemize}

\end{frame}

\begin{frame}{Plan für heute}

	\tableofcontents

\end{frame}

\section{Rückblick}

\begin{frame}[fragile]{Rückblick}{Kontrollstrukturen, symbolische Konstanten, Standardeingabe \& -ausgabe}

	\begin{lstlisting}[gobble=4]
		#include <stdio.h>
		
		int main() {
			
			int c;
			
			while ((c = getchar()) != EOF) {
				if (c >= 97 && c <= 122) {
					putchar(c - 32);
				} else {
					putchar(c);
				}
			}
			
			return 0;
		}
	\end{lstlisting}

\end{frame}

\section{Nachbereitung Hausaufgabe}

\begin{frame}[fragile]{Lösungsvorschlag}{Aufgabe 1.3}

	\begin{lstlisting}[gobble=4]
		#include <stdio.h> 
		
		int main() {
			
			float euro;
			printf("Bitte geben...");
			scanf("%f", &euro);
			
			while (euro != 0) {
				printf("Der Betrag entspricht %.2f USD\n", euro * 1.17);
				scanf("%f", &euro);
			}
			
			return 0;
			
		}
	\end{lstlisting}

\end{frame}

\begin{frame}[fragile]{Lösungsvorschlag}{Aufgabe 1.4}

	\begin{lstlisting}[gobble=4]
		#include <stdio.h>
		
		int main() {
			
			int n, n1;
			printf("Zahl? ");
			scanf("%d", &n);
			printf("%d[10] = ", n);
			
			while (n > 0) {
				n1 = n / 2;
				printf("%d",n - 2 * n1);
				n = n1;
			}
			
			printf("[2]\n");
			return 0;
		}
	\end{lstlisting}

\end{frame}

\section{Funktionen}

\begin{frame}[fragile]{Funktionen}

	\begin{lstlisting}[gobble=4]
		#include <stdio.h>
		
		void print_power(int base, int exp) {
			int pow = 1;
			int i = 0;
			
			for(;i < exp; i++) pow *= base;
			
			printf("%d ^ %d = %d\n", base, exp, pow);
		}
		
		int main() {
			print_power(2, 4);
			return 0;
		}
	\end{lstlisting}
	
\end{frame}

\begin{frame}[fragile]{Funktionen}

	\begin{lstlisting}[gobble=4]
		#include <stdio.h>
		
		int calculate_power(int base, int exp) {
			int pow = 1;
			int i = 0;
			
			for(;i < exp; i++) pow *= base;
			
			return pow;
		}
		
		int main() {
			printf("%d\n", calculate_power(2, 4));
			return 0;
		}
	\end{lstlisting}
	
\end{frame}

\section{Vorbereitung Hausaufgabe}

\begin{frame}{Ein Blick in die Hausaufgabe}

	\enquote{Schreiben Sie ein Programm, das die Zahl $\pi$ durch Monte-Carlo-Simulation berechnet.
	Nutzen Sie dafür folgende Idee: Wenn Sie eine große Zahl von zufälligen Punkten im Einheitsquadrat wählen, ist im Mittel bei $\pi$/4 dieser Punkte der Abstand vom Ursprung nicht
	größer als 1. (Für einen Punkt $(x, y)$ erhalten Sie den Abstand durch $\sqrt{x2 + y2}$).
	Wie viele zufällige Punkte im Einheitsquadrat benötigt Ihr Rechner, um $\pi$ auf sieben Nachkommastellen genau zu bestimmen?}

\end{frame}

\def\x{0}
\def\y{0}
\def\k{0}
\def\radius{5}


\begin{frame}{Monte-Carlo-Simulation}
	
	\begin{columns}
		\begin{column}{0.6\textwidth}
			\centering
			\begin{tikzpicture}
				\draw[fill=gray, opacity=0.1] (\radius,0) arc(0:90:\radius) -- (0,0) -- cycle;
				\draw[gray, opacity=0.25] (0,0) rectangle (\radius,\radius);
				\draw[->] (0,0) -- (1.1*\radius,0);
				\draw[->] (0,0) -- (0,1.1*\radius);
				\foreach \i in {1,2,...,100}{%
					\typeout{Point \i}%
					\pgfmathsetmacro\x{\radius*rnd}%
					\typeout{X \x}%
					\pgfmathsetmacro\y{\radius*rnd}%
					\typeout{Y \y}%
					\pgfmathsetmacro\k{(pow(\x,2)+pow(\y,2)) <pow(\radius,2)}%
					\typeout{im Kreis?: \k}%
					\pgfmathparse{ifthenelse(\k==1,"red","blue")}%
					\fill[\pgfmathresult] (\x,\y)circle(1pt);%
				}
			\end{tikzpicture}
		\end{column}
		\begin{column}{0.4\textwidth}
			\[\dfrac{\color{red} \#rot}{{\color{red} \#rot} + {\color{blue} \#blau}} \approx \dfrac{\pi}{4}\]
		\end{column}
	\end{columns}
\end{frame}

\section{Programmieren}

\begin{frame}{Programmieren}

	\begin{task}{~}
		Lösen Sie die Aufgaben auf dem Arbeitsblatt \texttt{handout05.pdf}.
	\end{task}

\end{frame}

\section{Abschluss}

\begin{frame}[fragile]{Lösungsvorschlag}{Aufgabe 4}

	\begin{lstlisting}[gobble=4]
		#include <stdio.h>
		
		int main() {
			int c, ws = 0;
			while((c = getchar()) != EOF) {
				if (ws) {
					if(c != ' ' && c != '\t') {
						putchar(c);
						ws = 0;
					}
				} else {
					if(c != ' ' && c != '\t') {
						putchar(c);
					} else {
						putchar(' ');
						ws = 1;
					}
				}
			}
			return 0;
		}
	\end{lstlisting}

\end{frame}

\begin{frame}[fragile]{Lösungsvorschlag}{Aufgabe 4 - Mit explizitem Zustand analog zur Vorlesung Folie 55}

	\begin{lstlisting}[basicstyle=\tiny\ttfamily,gobble=4]
		#include <stdio.h>
		#define WHITESPACES 0
		#define WORD 1
		
		int main() {
			int c, state = WORD;			
			while((c = getchar()) != EOF) {
				if (state == WHITESPACES) {
					if(c != ' ' && c != '\t') {
						putchar(c);
						state = WORD;
					}
				} else {
					if(c != ' ' && c != '\t') {
						putchar(c);
					} else {
						putchar(' ');
						state = WHITESPACES;
					}
				}
			}
			return 0;
		} 
	\end{lstlisting}

\end{frame}

\begin{frame}[fragile]{Lösungsvorschlag}{Aufgabe 5.1}

	\begin{lstlisting}[gobble=4]
		double bmi(double height, double weight) {
			return weight/(height*height);
		}
	\end{lstlisting}
	
\end{frame}


\begin{frame}[fragile]{Lösungsvorschlag}{Aufgabe 5.2}

	\begin{lstlisting}[gobble=4]
		#include <stdio.h>
		
		int isPrime(int n) {
			int f, isprime = 1;
			for(f = 2; isprime && f*f < n; f++) {
				if(n % f == 0) isprime = 0;
			}
			return isprime;
		}
			
		int main() {
			int limit, number;
			printf("Search limit: ");
			scanf("%d", &limit);
			for(number = 2; number <= limit; number++) {
				if(isPrime(number)) {
					printf("Found prime: %d\n", number);
				}
			}
		}	 
	\end{lstlisting}
	
\end{frame}

\begin{frame}{Abschluss}
	
	\begin{itemize}
		\item Üben
		\item Üben
		\item Üben
	\end{itemize}
	
\end{frame}

	% end of presentation
	%%%%%%%%%%%%%%%%%%%%%%%%%%%%%%%%%%%%%%%%%%%%%%%%%%%%%%%%%%%%%%%%%%%%%%%
	% backup slides
	
	\appendix
	\newcounter{finalframe}
	\setcounter{finalframe}{\value{framenumber}}
	
	%%%%%%%%%%%%%%%%%%%%%%%%%%%%%%%%%%%%%%%%%%%%%%%%%%%%%%%%%%%%%%%%%%%%%%%
	% end of backup slides
	\setcounter{framenumber}{\value{finalframe}}
\end{document}

