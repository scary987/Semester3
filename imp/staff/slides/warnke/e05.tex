% !TeX spellcheck = de_DE
%% 
%% This is file 'beamer_sample.tex'
%% 	
%% by Mathias Winkel
%%
%% Problems, bugs and comments to 
%% mathias.winkel2@uni-rostock.de
%%
\RequirePackage{fix-cm} % because of font size substituion warnings
\documentclass[10pt]{beamer} % die 10pt sollten festgelegt bleiben, da dies die Groesse der Mathematikschrift etc. beeinflusst

\usepackage[ngerman]{babel}  % deutsche Bezeichnungen und Trennung etc
\usepackage[utf8]{inputenc}
\usepackage[T1]{fontenc}
\usepackage{csquotes}
\usepackage{hyperref}        % interne Hyperlinks
\usepackage[scaled]{beramono}
\usepackage{microtype}
\usepackage{listings}
\usepackage{ragged2e}
\usepackage{marvosym}
\usepackage{xcolor}
\usepackage{pgffor}

%******************************************
% For giving image sources, see http://tex.stackexchange.com/a/48485
\usepackage[absolute,overlay]{textpos}
\setbeamercolor{framesource}{fg=gray}
\setbeamerfont{framesource}{size=\tiny}

\setbeamercolor{plain}{fg=black,bg=white}

\newcommand{\source}[1]{\begin{textblock*}{4cm}(8.2cm,8.6cm)
		\begin{beamercolorbox}[ht=0.5cm,right]{framesource}
			\usebeamerfont{framesource}\usebeamercolor[fg]{framesource} Quelle: {#1}
		\end{beamercolorbox}
	\end{textblock*}}
%******************************************

\lstdefinestyle{customc}{
	tabsize=2,
	belowcaptionskip=1\baselineskip,
	breaklines=true,
	xleftmargin=\parindent,
	language=C,
	showstringspaces=false,
	basicstyle=\scriptsize\ttfamily,
	keywordstyle=\bfseries\color{green!40!black},
	commentstyle=\itshape\color{purple!40!black},
	identifierstyle=\color{blue},
	stringstyle=\color{orange},
}

\lstset{escapechar=@,style=customc}

\definecolor{links}{HTML}{2A1B81}
\hypersetup{colorlinks,linkcolor=,urlcolor=structure.fg}

\newcommand{\framefrompdf}[2]{
	\begin{frame}[plain]
		\hspace*{-8.5pt}
		\begin{beamercolorbox}[wd=\paperwidth,ht=\paperheight]{plain}
			\makebox[\paperwidth]{\hspace{8.5pt}\includegraphics[page=#2,width=\paperwidth]{#1}}
		\end{beamercolorbox}
	\end{frame}
}

\newcommand{\framesfrompdf}[3]{
	\foreach \n in {#2,...,#3}{\framefrompdf{#1}{\n}}
}

\newcommand{\framesfrompdfclipped}[8]{
	\foreach \n in {#2,...,#3}{
		\begin{frame}[plain]
			\vspace*{-18pt}\hspace*{-8.5pt}
			\makebox[\linewidth]{\includegraphics[page=\n,width=#8,clip,trim = #4 #5 #6 #7]{#1}}
		\end{frame}
	}
}

\newcommand{\picturefrompdf}[7]{
	\begin{center}
		%trim option's parameter order: left bottom right top
		\includegraphics[page=#2,width=#7,clip,trim = #3 #4 #5 #6]{#1} 
	\end{center}
}

\newcommand{\rto}{$\Rightarrow$ }

\newcommand{\emp}[1]{{\color{orange}{#1}}}

\newcommand{\img}[2]{
	\centering
	\includegraphics[#2]{figures/#1}
}

\newcommand{\questions}{
\begin{frame}
	
	\centering
	\Large
	Fragen?

\end{frame}
}

\newenvironment{task}[1]{
	\begin{block}{\textbf{Aufgabe} \hfill #1}
}{
	\end{block}
}

\AtBeginSection[]
{
	\begin{frame}{Plan für heute}
		\tableofcontents[currentsection]
	\end{frame}
}

\usepackage[uni,footuni,headlogo]{./unirostock/beamerthemeRostock}
% erster Parameter: Farbschema
%        moegliche Werte (nomen est omen): uni, inf, msf, ief, mnf, mef, juf, wsf, auf, thf, phf
%        Standardwert: uni
% zweiter Parameter: Fusszeile
%        moegliche Werte: footuni (Standard) - Fusszeile nach Handbuch des CD
%                         foottitle          - Autor und Title in der Fusszeile
%                         footheadings       - lebende Ueberschriften in der Fusszeile
%                         footuniheadings    - Autor und Uni sowie lebende Ueberschriften in der Fusszeile
% dritter Parameter: Kopfzeile
%        moegliche Werte: headlogo (Standard)- Inhalt von \mylogo in der Kopfzeile
%                         headtitle          - Vortragstitel
%                         headframetitle     - Folientitel
%                         headframesubtitle  - Folientitel und -untertitel
%        bitte nicht mehrere Varianten gleichzeitig angeben :-)

\setbeamercovered{invisible}
\addtobeamertemplate{block begin}{}{\justifying}

%%%%%%%%%%%% Festlegung der Titelseite %%%%%%%%%%%%%%%%%%%%%%%%%%%%%%%%%
\title[]{Imperative Programmierung}

\subtitle{Übung}

\author{\textsc{Tom Warnke}}

\institute{Universität Rostock, Institut für Informatik}

\titlegraphic{}

% ein alternatives Titelbild kann mittels 
%\titleimage{Dateiname.xyz}
% angegeben werden (auf vernuenftiges Seitenformat und Kontrastwerte achten, Skalierung und Abschneiden der oberen rechten Ecke passieren automatisch)
\titleimage{}

%%%%%%%%%%%% Festlegungen fuer Kopf- und Fusszeile %%%%%%%%%%%%%%%%%%%%
% Institutsname f\"ur die Fusszeile (nur wenn bei Paketeinbindung 'footuni' angegeben ist)
\footinstitute{Fakultät für Informatik und Elektrotechnik, Institut für Informatik}
% eigenes Logo oben rechts hinzufuegen (bitte auf vernuenftiges Format achten - ein zu hohes Logo verschiebt das Layout)
\renewcommand{\mylogo}{}

%%%%%%%%%%%%%%%%%%%%%%%%%%%%%%%%%%%%%%%%%%%%%%%%%%%%%%%%%%%%%%%%%%%%%%%
\usepackage{tikz}

\date{}

%%%%%%%%%%%%%%%%%%%%%%%%%%%%%%%%%%%%%%%%%%%%%%%%%%%%%%%%%%%%%%%%%%%%%%%
\begin{document}

\maketitle

\begin{frame}{Plan für heute}

	\tableofcontents

\end{frame}

\section{Rückblick auf die Vorlesung}

	\framesfrompdf{../../../public/lecture/23001-02-cIntro.pdf}{38}{39}
	
	\begin{frame}[fragile]{Symbolische Konstanten}{Vordefinierte Konstanten}
		
		\begin{lstlisting}[gobble=4]
			#include <stdio.h>
			
			int main() {
				printf("in line %d\n", __LINE__);
				return 0;
			}
		\end{lstlisting}
	
	\end{frame}
	
	\framesfrompdf{../../../public/lecture/23001-02-cIntro.pdf}{47}{48}
	
	\framesfrompdf{../../../public/lecture/23001-02-cIntro.pdf}{57}{59}

\section{Vorbereitung Hausaufgabe}

\begin{frame}{Ein Blick in die Hausaufgabe}

	{\footnotesize\enquote{Schreiben Sie ein Programm, das eine positive ganze Zahl in das binäre Zahlensystem übersetzt.
	Dabei dürfen Sie die Binärstellen in umgedrehter Reihenfolge ausgeben (d.h., niederwertigste
	Stelle zuerst), so dass Sie etwa für die Zahl $13_{10}$ die Ausgabe $1011_2$ erhalten (in normaler Reihung
	wäre die Ausgabe $1101_2$).}}

\end{frame}

\begin{frame}{Veranschaulichung von Zahlensystemen}

	\begin{columns}
		\begin{column}{0.6\linewidth}
			\begin{tabular}{|c|c|c|c|c|l|}
				\hline $2^0$ & $2^1$ & $2^2$ & $2^3$ & \dots & \\ 
				\hline 1 & 0 & 1 & 1 & \dots & = $1011_{2}$\\
				\hline
				\hline $8^0$ & $8^1$ & $8^2$ & $8^3$ & \dots & \\ 
				\hline 5 & 1 & 0 & 0 & \dots & = $51_{8}$\\
				\hline
				\hline $10^0$ & $10^1$ & $10^2$ & $10^3$ & \dots & \\ 
				\hline 3 & 1 & 0 & 0 & \dots & = $31_{10}$\\ 
				\hline 
			\end{tabular} 
		\end{column}
		\begin{column}{0.4\linewidth}
				(Umgekehrte Anordnung der Ziffern laut Aufgabenstellung)
		\end{column}
	\end{columns}
	Vorgehen zur Umrechnung:
	\begin{enumerate}
		\item Rest bei Division mit Basis des Zielzahlsystems ausgeben
		\item Neue Zahl durch \emp{ganzzahlige} Division durch Basis des Zielzahlsystems
		\item Weiter bei 1. bis neue Zahl 0 ist
	\end{enumerate}

\end{frame}

	% end of presentation
	%%%%%%%%%%%%%%%%%%%%%%%%%%%%%%%%%%%%%%%%%%%%%%%%%%%%%%%%%%%%%%%%%%%%%%%
	% backup slides
	
	\appendix
	\newcounter{finalframe}
	\setcounter{finalframe}{\value{framenumber}}
	
	%%%%%%%%%%%%%%%%%%%%%%%%%%%%%%%%%%%%%%%%%%%%%%%%%%%%%%%%%%%%%%%%%%%%%%%
	% end of backup slides
	\setcounter{framenumber}{\value{finalframe}}
\end{document}

