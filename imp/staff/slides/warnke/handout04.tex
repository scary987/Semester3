% !TeX spellcheck = de_DE
\documentclass[]{article}

\usepackage[utf8]{inputenc}
\usepackage[T1]{fontenc}
\usepackage[scaled]{beramono}
\usepackage{amssymb}
\usepackage{amsmath}
\usepackage{graphicx}
\usepackage[top=2cm, bottom=2cm, left=2cm, right=4cm]{geometry}
\usepackage{csquotes}
\usepackage{listings}
\usepackage{hyperref}
\usepackage{booktabs}

\lstset{
	tabsize=2,
	language=Java,
	showstringspaces=false,
	basicstyle=\footnotesize\ttfamily,
}

%no indentation
\setlength\parindent{0pt}

\pagestyle{empty}

\begin{document}

	\section{Blöcke}
	
	In C können mit geschweiften Klammern \lstinline|{}| Blöcke gebildet werden.
	Welche Ausgabe erzeugen die folgenden Programme?
	Überprüfen Sie Ihre Vermutung, indem Sie die Programme kompilieren und ausführen.
	War Ihre Vermutung korrekt? Warum (nicht)?
	
	\noindent\begin{minipage}{.48\textwidth}
	\begin{lstlisting}[gobble=4,frame=single]
		#include <stdio.h>
		
		main() {
			
			int b = 3;
			{
				int b = 5;
			}
			printf("b=%d\n", b);
		}
	\end{lstlisting}
	\end{minipage}\hspace{0.04\textwidth}
	\begin{minipage}{.48\textwidth}
	\begin{lstlisting}[gobble=4,frame=single]
		#include <stdio.h>
		
		main() {
			
			int b = 3;
			{
				b = 5;
			}
			printf("b=%d\n", b);
		}
	\end{lstlisting}
	\end{minipage}
	

	\section{Kontrollstrukturen}
	
	\subsection{Schleifen I}
	
	\begin{lstlisting}[gobble=4]
		int i = 12;
		int counter = 0;
		
		while ((i % 2) == 0) {
			i = i / 2;
			counter += 1;
		}
	\end{lstlisting}
	
	\begin{enumerate}
		\item Simulieren Sie die Ausführung des gegebenen Programmfragments (z.B. auf einem Blatt Papier).
		Was berechnet das Programmfragment? 
		\item Erweitern Sie das Programm so, dass \texttt{i} vom Nutzer eingelesen wird.
		\item Wandeln Sie die \texttt{while}-Schleife in eine gleichwertige \texttt{for}-Schleife um (Vorlesungsfolie 32).
%		\item Recherchieren Sie, was \texttt{break} und \texttt{continue} in Schleifen bewirken.
	\end{enumerate}
	
	\subsection{Schleifen II}
	
	\begin{lstlisting}[gobble=4]
		#include <stdio.h>
		
		int main() {
			
			float i = 0;
			
			while(i != 10.0) {
				printf("%f\n", i);
				i += 0.1;
			}
			
			return 0;
		}
	\end{lstlisting}
	
	\begin{enumerate}
		\item Welche Ausgabe erzeugt das gegebene Programm?
		\item Überprüfen Sie Ihre Vermutung, indem Sie das Programm ausführen.
	\end{enumerate}
	
	\subsection{Schleifen III}
	
	Schreiben Sie ein Programm, das eine Reihe von Zahlen einliest und deren Durchschnitt ausgibt.
	Das Ende der Zahlenreihe wird durch eine negative Zahl signalisiert.
	
	\textit{Bonus: Ihr Programm ignoriert Eingaben, die keine Zahlen sind.}
	
	\section{Standardeingabe \& Standardausgabe}
	
	Schreiben Sie ein Programm, das Zeichen von der Standardausgabe einliest und auf die Standardausgabe zurückschreibt.
	Dabei sollen alle Kleinbuchstaben durch Großbuchstaben ersetzt werden.
	Testen Sie Ihr Programm mit Dateien als Ein- und Ausgabe (als Eingabe können Sie z.B. den Programmquellcode verwenden).
	
	\textit{Hinweis: Zeichen werden als Zahlen kodiert: siehe Vorlesungsfolie 41 und/oder Internetrecherche nach \enquote{ASCII}.}

	
\end{document}