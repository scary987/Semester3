% !TeX spellcheck = de_DE
\documentclass[a4paper]{article}

\usepackage[utf8]{inputenc}
\usepackage[T1]{fontenc}
\usepackage[scaled]{beramono}
\usepackage{amssymb}
\usepackage{amsmath}
\usepackage{graphicx}
\usepackage[top=2cm, bottom=1cm, left=2cm, right=4cm]{geometry}
\usepackage{csquotes}
\usepackage{listings}
\usepackage{hyperref}
\usepackage{booktabs}

\lstset{
	tabsize=2,
	language=Java,
	showstringspaces=false,
	basicstyle=\footnotesize\ttfamily,
}

%no indentation
\setlength\parindent{0pt}

\pagestyle{empty}

\begin{document}

	\section{Rekursion}
	
	Was ist jeweils die Ausgabe der folgenden Programme?
	
	\noindent
	\begin{minipage}[t]{0.5\linewidth} %1
		\begin{lstlisting}[gobble=6]
			#include <stdio.h>
			
			void rec(int i) {
				if (i > 0) {
					printf("%d\n", i);
					rec(i - 1);
				}
			}
			
			int main() {
				rec(10);
				return 0;
			}
		\end{lstlisting}
	\end{minipage}
	\quad
	\begin{minipage}[t]{0.5\linewidth} %2
		\begin{lstlisting}[gobble=6]
			#include <stdio.h>
			
			void rec(int i) {
				if (i > 0) {
					rec(i - 1);
					printf("%d\n", i);
				}
			}
			
			int main() {
				rec(10);
				return 0;
			}
		\end{lstlisting}
	\end{minipage}
	
	\section{Relationen als Matrizen}
	
	\subsection{Ausgabe}
	
	Schreiben Sie eine Funktion \texttt{print}, die eine binäre Matrix übergeben bekommt, in der eine Relation kodiert ist.
	Die Funktion soll alle Paare von Elementen ausgeben, die gemäß der Relation miteinander in Beziehung stehen.
	Definieren Sie die Größe der Matrix mit einer symbolischen Konstante.
	Beispielsweise soll das Programm links die Ausgabe rechts liefern:
	
	\noindent\begin{minipage}[t][][b]{0.6\linewidth}
	\begin{lstlisting}[gobble=4]
		#define SIZE 3
		
		int main() {
			int arr[SIZE][SIZE] = { 
				{0, 0, 1},
				{0, 1, 1},
				{0, 0, 0} };
				
			print(arr);
			
			return 0;
		}
	\end{lstlisting}
	\end{minipage}\qquad
	\begin{minipage}[t][][b]{0.3\linewidth}
	\begin{lstlisting}[gobble=4]
		(0, 2)
		(1, 1)
		(1, 2)
	\end{lstlisting}
	\end{minipage}
	
	\subsection{Nachfolger}
	
	Schreiben Sie eine Funktion, die eine binäre Matrix übergeben bekommt und so überschreibt, dass in der Matrix die Relation Nachfolger kodiert ist.
	Beispielsweise soll das Programm links die Ausgabe rechts liefern:
	
	\noindent\begin{minipage}[t][][b]{0.6\linewidth}
	\begin{lstlisting}[gobble=4]
		#define SIZE 3
		
		int main() {
			
			int arr[SIZE][SIZE];
			
			create(arr);
			
			print(arr);
			
			return 0;
		} 
	\end{lstlisting}
	\end{minipage}\qquad
	\begin{minipage}[t][][b]{0.3\linewidth}
	\begin{lstlisting}[gobble=4]
		(1, 0)
		(2, 1)
	\end{lstlisting}
	\end{minipage}
	
	\clearpage
	
	\section{Textstatistik}
	
	\subsection{Bedingte Wahrscheinlichkeiten}
	
	Schreiben Sie eine Funktion, die Zeichen von der Standardeingabe einliest und bestimmt, welchem Prozentsatz der Zeilenumbrüche in der Eingabe ein Semikolon vorausgeht.
	Testen Sie die Funktion mit einer Datei, beispielsweise dem Quelltext ihres Programms.
	
	\subsection{Tripel zählen}
	
	Schreiben Sie eine Funktion, die eine Folge von Ziffern von der Standardeingabe einliest und darin die Häufigkeit aller Paare und Tripel von aufeinanderfolgenden Ziffern zählt.
	Zeichen, die keine Ziffern sind, sollen ignoriert werden.
	Die Funktion soll am Ende die absolute Häufigkeit jedes Paars und jedes Tripels ausgeben.
	Testen Sie Ihre Implementierung mit den Nachkommastellen von Pi, z.B. von \href{http://www.pibel.de/}{pibel.de}.
	
\end{document}