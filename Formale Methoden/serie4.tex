\documentclass[10pt,a4paper]{article}
\usepackage[utf8]{inputenc}
\usepackage[T1]{fontenc}
\usepackage{amsmath}
\usepackage{amsfonts}
\usepackage{amssymb}
\usepackage{graphicx}
\title{serie - 3  Tobias Reincke 218203884}
\begin{document}

\section{Aufgabe 1}


\section{Aufgabe 2}
 \subparagraph{Verschärfung}
 Eine Verschärfung tritt auf, wenn die die Beweislage, also die Preconditions, von P' auf P erweitert werden. (P' $\rightarrow$P). D.h die  Postconditions können gleich bleiben, also Q$\rightarrow$Q. Da die Ursprüngliche Menge der Beweislage noch das Gleiche beweist bzw. noch zu dem demselben führt, gilt der Satz. $\{P'\} S  \{Q\} \rightarrow \{P\} S\{Q\} $ wenn $P'\rightarrow P$. Dabei ist trivial: $|\{P\}|\geq|\{P' \}|$

 \subparagraph{Abschwächung} 
  Eine Abschwächung tritt auf, wenn die das Gefolgerte, also die Postconditions, von Q auf Q' erweitert werden. (Q $\rightarrow$Q'). D.h die  Preconditions können gleich bleiben, also P$\rightarrow$P.$\{P\} S  \{Q\} \rightarrow \{P\} S \{Q'\} $. Da die Beweislage gleich bleibt kann man, $\{P\} S  \{Q\} $als $ \{P\} S \{Q'\} $ darstellen, wenn$Q \rightarrow Q' $. Dabei ist trivial: $|\{Q\}|\geq|\{Q'\}|$
i\subparagraph{Verstärkung $\wedge$ Abschwächung}
Wenn man beides Anwendet, ist das so als würde man die eine kleinere  logische Implikation auf eine größere Erweitern (Beweist mehr mit weniger Material.) Die Aussage ist  also schon gegeben.
$\text{cons}_\text{cons}$:





	
	
\end{document}