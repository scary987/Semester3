\documentclass[10pt,a4paper]{article}
\usepackage[utf8]{inputenc}
\usepackage[T1]{fontenc}
\usepackage{amsmath}
\usepackage{amsfonts}
\usepackage{amssymb}
\usepackage{graphicx}
\title{serie - 3  Tobias Reincke 218203884}
\begin{document}

\section{Aufgabe 1}
\subsection{a}
Falsch. Der Erste Satz sagt, dass die Postconditions = Precondtions, wenn eine Zuweisung gemacht wird, die in Preconditions bereits enthalten ist.
Das zweite macht die umgedrehte Folgerung. Die Belegung ist nicht in jeder Precondition enthalten.  Die stimmt aber nicht. Man muss von P auf Q (über ein S) folgern können, nicht andersherum.
\subsection{b}
Falsch. P S Q  gilt, falls es beweisbar ist ist, beziehungsweise . Aber dieser Beweis setzt nicht vorraus, dass P = Q, oder S dem "skip"-Befehl entspricht.
\subsection{c}

Nicht die konkreteste Aussage, aber wenn aber wenn x = 1 gilt und dem x := x+1 zugewiesen wird, also x := 1+1 = 2 gilt x>1 und somit auch x>=1. Nicht die konkreteteste Aussage, aber P, S  und Q stimmen überein.
\subsection{d} Falls die mit {tt} die Aussagen in P gemeint waren, kann nach einer Zustandsänderung nur auf P basierende  wahre Aussagen $\in$ Q sein. Somit wäre es kein gültiges Hoare-Tupel. \\
Falls damit jedoch eine Aussagenvariable, mit wahrer Belegung, gemeint wahr, kann diese in S auch einfach negiert werden (zum Beispiel). 
\section{Aufgabe 2}

 \subparagraph{Verschärfung}
 Eine Verschärfung tritt auf, wenn die die Beweislage, also die Preconditions, von P' auf P erweitert werden. (P' $\rightarrow$P). D.h die  Postconditions können gleich bleiben, also Q$\rightarrow$Q. Da die Ursprüngliche Menge der Beweislage noch das Gleiche beweist bzw. noch zu dem demselben führt, gilt der Satz. $\{P'\} S  \{Q\} \rightarrow \{P\} S\{Q\} $ wenn $P'\rightarrow P$. Dabei ist trivial: $|\{P\}|\geq|\{P' \}|$

 \subparagraph{Abschwächung} 
  Eine Abschwächung tritt auf, wenn die das Gefolgerte, also die Postconditions, von Q auf Q' erweitert werden. (Q $\rightarrow$Q'). D.h die  Preconditions können gleich bleiben, also P$\rightarrow$P.$\{P\} S  \{Q\} \rightarrow \{P\} S \{Q'\} $. Da die Beweislage gleich bleibt kann man, $\{P\} S  \{Q\} $als $ \{P\} S \{Q'\} $ darstellen, wenn$Q \rightarrow Q' $. Dabei ist trivial: $|\{Q\}|\geq|\{Q'\}|$
i\subparagraph{Verstärkung $\wedge$ Abschwächung}
Wenn man beides Anwendet, ist das so als würde man die eine kleinere  logische Implikation auf eine größere Erweitern (Beweist mehr mit weniger Material.) Die Aussage ist  also schon gegeben (durch Semantische Equivalenz auf kleinerer Ebene.).
$\text{cons}_\text{cons}$:

\pagebreak
\section{Aufgabe 3} 

\subsection{initials}
$ \{a \ge 0 \} k:= 0 \{a \ge 0, k \ge 0, k = 0 \} \rightarrow$ \newline 
$ \{a \ge 0, k \ge 0, k =0 \} n:=0 \{a \ge 0, k \ge 0, k =0 , n\ge 0 , n = 0\}$ \newline
 $\{a \ge 0, k \ge 0, k =0 , n\ge 0 , n = 0\} m := 0 \{a \ge 0, k \ge 0, k =0 , n\ge 0 , n = 0, m \ge 1, m = 1 , m = 2 * 0 +1 = 2 * n +1 \}$
 \newline
 
\subsection{while-Loop}
\subparagraph{precondition}
{n$\ge$0, m$\ge$1, m = 2*(n)+1, k -> A[n*n], $(n-1)*(n-1)<a$ }

 \subparagraph{B[ k<a]=tt} $\rightarrow$
 \\
   n ->A[n]+1
   k -> A[k]+ A[m] = $n*n+  2n+1$ \\
   m ->2 + A[2n-1]+1 = A[2(n)+1]\\
$\rightarrow$
$\{(n-1)*(n-1)<a, n>n-1, k=2*(n-1)+1+(n-1)*(n-1) = n*n, m= 2(n)+1 \}$

\subparagraph{if B[k$<$a]=ff}
$\rightarrow $ \\
\{n$\ge$0, m$\ge$1, m = $2*(n)+1$, k -> A[$n*n$], $(n-1)*(n-1)<a$, A[n]*[n]=A[k]$\ge$a \}
$\rightarrow$  \\$(n-1)*(n-1)$<a$\le$n*n \\ $\rightarrow$ $(n-1)<\sqrt{a}\le n$
	
	
\end{document}