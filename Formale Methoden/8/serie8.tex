\documentclass[]{article}
\usepackage{amssymb}
	\usepackage{euler}

%opening
\title{Formale Methoden - Serie 8}
\author{Tobias Reincke \\ Matrikelnummer 218203884}
\date{}

\begin{document}

\maketitle



\section*{Aufgabe 1}
   Eine Ablaufformel ist eine boolesche Kombination (Verknüpfung) aus Schrittformel und dem $\square$ ('always') Zeichen. Wenn x eine Variable im Zustand, dann x' die Variable im Folgezustand.
   \\ \\
   "Schritt (s1 s2) erfüllt Schrittformel f falls f bei folgender Belegung f erfüllt ist:
   f(v) = s1(v), f(v‘) = s2(v) für alle v in Var" (D.h es muss eine Aussage für den nächsten Schritt machen.)
	\subsection*{a)} Ja. 
    \subsection*{b)} Nein, das $\square$ fehlt. Eine Ablaufformel muss eine Aussage für jeden Zustand machen.
	\subsection*{c)} Ja.
	\subsection*{d)} Ja.
	\subsection*{e)} Nein. \\ Strong fair impliziert einen Zustand, in dem es nicht möglich ist zu stottern für eine Variable. Im Vergleich zu anderen Variablen muss sie sich verändern, der Zustand muss sich ändern.
	\subsection*{f)} Nein. 

\section*{Aufgabe 2}
Das erste Statechart: \\
Sei A der Initialzustand des Statescharts, B der Zustand nachdem a ausgeführt wurde, und C der Zustand, nachdem b ausgeführt wurde. z sei die Variable, die beschreibt, welchen Zustand der Prozess hat.\\
TLA: $\phi$ = (z = A)$\land \square$ (z' = IF z = A THEN B ELSE IF z = B THEN C ELSE IF z = C THEN A) \\ \\

DAS zweite Statechart: \\ Sei Z der Zustand, der den Prozess wiederspiegelt.\\ Seien y,x die Zustandsübergänge. f sei die Variable, die den Zustandsübergang darstellt.


TLA: $\phi$ = (z=z) $\land \ \square$(z'=z $\land$ (f'=x$\lor$f'=y) )  

\section*{Aufgabe 3}
\subsection*{a)}
x: 0 -5 -4 -3 -2 -1  0...\\
y: 0   5  4  3  2  1  0...\\
(und genauso weiter)
\subsection*{b)} 
x: 0 -5 -4 -4 -3 -2 -1  0  1  2  3 0 ...                                                                                            \\	
y: 0  5  4  4  4  3   3  3  3  3  3 0 ...                         
\subsection*{c) } 
$\Psi$ ist in $\sigma$ erfüllt.   x + x' ist immer 2(x)+1 außer für x = 0, dann -5. Sonst ist 2(x)-1 für jede negative Zahl x(x ist immer negativ außer in 0) immer negativ.      \\                                                         \\
$\Psi$ ist in $\sigma'$ nicht  erfüllt für den Fall, dass x = 0 und stottert. Dann gilt:\\ x + x' = 0 + 0 = 0 $\nless$ 0. \\ \\
$\Xi$ ist in $\sigma$ erfüllt, da bei jedem Schritt sich beide 0 nähern und mit dem $|5|$ (nach 0 im ersten Zustand) anfangen. Beide werden gleichzeitig 0 und darauf x = 5 und -5.
\\
$\Xi$ ist in $\sigma'$ nicht erfüllt, wenn entweder x stottert oder y, wenn beide den gleichen Betrag hatten. Beispiel: \\ \ \\
x:0 -5 -5 ...\\
y:0  5  4 ...\\
$\rightarrow$ -5 $\neq$ - (4) \\
Dasselbe gilt für jeden Zustand in dem die Beträge nicht gleich sind, nach potenziellen Stottern einer (oder beider).

\section*{Aufgabe 4}
Seien A,B,C,D,E Philosophen . \\ A1,A2,... Die jeweiligen Hände des Philosphen.
Emulieren des StateCharts von letzter Woche. 



 $ \phi =  ( A1 = 0 \land B1 = 0 \land C1 = 0 \land D2 = 0 \land E2 = 0 \land A2 = 0 \land B2 = 0 \land C2 = 0 \land D2 = 0 \land E2 = 0 \land A = THINKING \land B = THINKING \land C = THINKING \land D=THINKING \land E=THINKING) \land \\ $ \textit{Jetzt die Folgezustände\\} $\square([
 ((A1'= \ IF \ B2 \ = \ 0 \ THEN \ 1 \ ELSE \ 0 ) \lor(B2'= \ IF \ A1 \ = \ 0 \ THEN \ 1 \ ELSE \ 0) )  \land  \\
 ((B1'= \ IF \ C2 \ = \ 0 \ THEN \ 1 \ ELSE \ 0) \lor(C2'= \ IF \ B1 \ = \ 0 \ THEN \ 1 \ ELSE \ 0)) \land  \\
 ((C1'= \ IF \ D2 \ = \ 0 \ THEN \ 1  \ ELSE \ 0) \lor(D2'= \ IF \ C1 \ = \ 0 \ THEN \ 1  \ ELSE \ 0 ))  \land  \\
 ((D1'= \ IF \ E2 \ = \ 0 \ THEN \ 1  \ ELSE \ 0) \lor(E2'= \ IF \ D1 \ = \ 0 \ THEN \ 1  \ ELSE \ 0 ))  \land \\ 
 ((E1'= \ IF \ A2 \ = \ 0 \ THEN \ 1  \ ELSE \ 0) \lor(A2'= \ IF \ A1 \ = \ 0 \ THEN \ 1  \ ELSE \ 0 )) 
  ]_{(A1,A2,B1,B2,C1,C2,D1,D2,E1,E2)} \\ 
  \land (A'= \ IF \ (A1=1 \land A2 =1) \ THEN \ EATING \ ELSE \ THINKING ) \\
  \land (B'= \ IF \ (B1=1 \land B2 =1) \ THEN \ EATING \ ELSE \ THINKING ) \\
  \land (C'= \ IF \ (C1=1 \land C2 =1) \ THEN \ EATING \ ELSE \ THINKING ) \\
  \land (D'= \ IF \ (D1=1 \land D2 =1) \ THEN \ EATING \ ELSE \ THINKING ) \\
  \land (E'= \ IF \ (E1=1 \land E2 =1) \ THEN \ EATING \ ELSE \ THINKING ) )\\ \rightarrow $
  Da Variablen stottern können, ist jeder Folgezustand wohldefiniert. \\(Weil die  Folgezustände nur für jeweils die Hälfte der Hände definiert ist.) \\
  Zur Erklärung:\\ Eine von beiden anliegenden Händen wird 1 gesetzt (die andere ist 0), oder beide stottern. Falls eine 1 ist, bleibt sie 1 (stottern) oder wird 0. Die andere wird automatisch stottern und 0 bleiben. 
  
  
 





\end{document}