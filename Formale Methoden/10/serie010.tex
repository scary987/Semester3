\documentclass[]{article}
\usepackage{listings}
%opening
\title{Formale Methoden Serie 10}
\author{Tobias Reincke\\Matrikelnummer: 218203884}

\begin{document}

\maketitle



\section{Aufgabe 1}
\subsection{a)}
Nein.
\subsection{b)}
Ja, weil dann alle Kreise initial markiert sind.
\subsection{c)}
Ja, sind ein Subset der FC-Netze (Folie.)
\subsection{d)}
Nein.
\subsection{e)}
Ja.
\subsection{f)}
Nein.
\subsection{g)}
ja.


\section{Aufgabe 2}
\begin{tabular}{|c|c|c|c|}
	\hline
	& Zustandsmaschine & Synchronisationsgraph & Free-Choice-Netz \\
	\hline
	N\_\$1\$ & nein & nein & nein (s2 \& s5 passen nicht), (s8 \&6)   \\
	\hline
	N\_\$2\$ & ja & ja & ja \\
	\hline
	N\_\$3\$ & n & ja & ja \\
	\hline
\end{tabular}

\section{Aufgabe 3}
Minimumregionen:=\{ \{0\} , \{1,2,3\} , \{4,5\}  , \{6,7\}  \}\\
E =\{ \{0, \{a,b,c\},  \{1,2,3\}  \}, \{ \{1,2,3\},d, \{4,5\}\} ,\{ \{1,2,3\}, e, \ {6,7}\}\}
\section{Aufgabe 4}
Wie nach Hinweis, habe ich alle Falle und Siphone per Programm ausgetestet.\\
Es ist offentsichtlich, dass der ganze Graph an sich ein Siphon und Falle ist.
\subparagraph{a,b,c)} \ \\ Das sollen Mengen sein, ich war nur zu faul, für die Mengennotation in Latex, ist anstrengend. \\ Mein Code zu finden auf: https://github.com/scary987/Petrinetstuff/blob/master/allcombinations.cpp \\
\begin{tabular}{||c|c|c|c|c|c|c|c|c||} 
	\hline
	Siphone & s8,s9,s11 & s8,s9,s11,s12 & s8,s9,s12 & s8,s10,s11,s12 & s8,s10,s11 & s8,s10,s12 & s10,s12 &    \\
	\hline
	Marking: & s8,s9,s11 & s8,s9,s11 &  - & - & -  & ja & s10,s12&    \\
	\hline
\end{tabular}\\ \ \\
$\rightarrow$ Das Petrinetz ist nicht lebendig, das 3 Siphone, keine initial-markierten Fallen haben.
\begin{tabular}{|c|c|c|c|c|c|c|}
	\hline
	Fallen &8,s9,s11 & s8,s9,s11,s12  & s8.s9,s11 & s9,s10,s12 & s10,s11,s12 & s10,s12 \\
	\hline
\end{tabular}\\ \ \\
Ich habe mir außerdem erlassen, den gesamten Graphen zu testen, da die Identität als Falle und Siphon trivial ist. (s8,s9,s10,s11,s12)\\
\{s8, s9, s10, s12\},
\{s8, s9, s10, s11\} werden von meinem Code nicht ausgegeben, ergeben sich aber durch die Mengenvereinigung der Siphone, und ich habe das debugging noch nicht ganz geschafft. :(\\
\{s9, s10, s11, s12\} als Falle auch. 










\section{Bonus}
Zerlegung von N$_4$:\\
\begin{tabular}{|c|c|c|c|c|c|}
	\hline
	s1,s2,t1,t2 & s3,t3 & s4,t4 & s5,t5 & s6,t6 & s7,s8,t7 \\
	\hline
\end{tabular}
\\
Zerlegung von N$_5$:\\
\begin{tabular}{|c|c|c|}
	\hline
	s1,t2,t3 & s3,t4,t5 & s2,t1 \\
	\hline
\end{tabular}
\end{document}
