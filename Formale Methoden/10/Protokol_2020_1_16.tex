\documentclass[]{article}
\usepackage{amssymb,latexsym,amsmath}
\usepackage[utf8]{inputenc}
\usepackage{hyperref}
\usepackage{ngerman}

\usepackage{scrpage2}
\usepackage{lastpage}
\pagestyle{scrheadings}
\clearscrheadfoot

%opening
\title{Protokoll der Sitzung des Fachschaftsrates Informatik 2020.1.16}
\author{Tobias Reincke}
\date{16.1.2020}
\begin{document}

\maketitle
\tableofcontents

\setlength{\parindent}{0pt} 

\addtolength{\textwidth}{1.0in}
\addtolength{\textheight}{1.00in}
\addtolength{\evensidemargin}{-0.75in}
\addtolength{\oddsidemargin}{-0.75in}
\addtolength{\topmargin}{-.50in}

\cfoot{Protokoll vom 16. Januar 2020 - Seite \pagemark{} von \pageref{LastPage}}
\section{}
\begin{tabular}{l l}
	\textbf{Beginn:} & 17:00 Uhr\\
	\textbf{Ende:} & Uhr\\
	\textbf{Ort:} & KZH R116\\
\end{tabular}\\
\begin{table}[h]
	\centering
	\begin{tabular}{|l|c|c|}
		\hline
		Name & erschienen & gegangen \\
		\hline
		Adrian Lutsch        & 17:00 Uhr &  \\
		Anna-Lena Neufeld    & 17:00 Uhr & \\
		Annika Behrendt      & 17:00 Uhr & 18:33 Uhr \\
		Henning Dirk Richter &17:00 Uhr & \\
		Henrik Bongertmann   & 17:00 Uhr & \\
		Jack Rittelmeyer     & & \\
		Jannik Blanck &  & 18:32 Uhr\\
		Jann-Hannes Polland  & 17:00 Uhr &  \\
		
		
		Julius Richert       & 17:07 Uhr & \\
		Justin Kreikemeyer   & & \\
		Max Kaseler          & 17:00 Uhr & \\
		Nick Kotsakidis      & 17:07 Uhr & \\
		Nicola Drüeke        & 17:00 Uhr & \\
		Richard Dabels       & 17:00 Uhr & \\
		Ruven Kronenberg     & 17:00 Uhr & \\
		Shameel Faraz        & 17:09 Uhr & \\
		Sophie Wallner       & 17:00 Uhr & \\
		Thora Mertz          & 17:00 Uhr & 18:33 Uhr \\
		Tobias Reincke       & 17:10 Uhr & \\
		Vivienne Reuter      & & \\
		Yannik Blank         & & \\
		Yazar Strulik        & & \\
		\hline
	\end{tabular}
\end{table}
\textbf{Gäste: -}
\section{Formalien}
\textbf{Abstimmung Protokoll: Bei einer Abstimmung angenommen.}
\section{Aktuelles}



\section{Kanboard Tickets}
-
\section{Organisation}
\subparagraph{Tshirtbestellung}\ \\ \
Es funktioniert alles normal, $\frac{4}{7}$ der neuen sind eingetragen.
Das genaue Modell kann so nicht direkt bestellt werden bis März, Annalena sucht nach Alternativen.
\subparagraph{Raum für die Zapf}\ \\ \
Alles klar, die Physiker wollen den FSRaum für ihre Zapf benutzen und bekommen Zugriff auf die eine Seite des Lagers und die Zeltgarnitur.
\section{Studium und Lehre}
\subparagraph{Prüfungsfreier Zeitraum}\ \\ \
$\rightarrow$Es wurde berichtet. Unsere Vertretung muss sich im Ton verbessern. \\
Schriftliche Prüfungen im September können nicht gewährleistet werden, da nicht genug Hörsäale.
$\rightarrow$ Prüfungen werden eventuell in den ersten Prüfungszeitraum verschoben.\\ Sehr gefährlich für Studis (Ersties) im zweiten Semester. 
Frage: Was wollen wir? 

Optionen:
\begin{itemize}
\item Weniger Abstand zwischen Studenten? 
\item Mehr Prüfungen in einem kleinerem Zeitraum?
\item Prüfungsfreier Zeitraum verschieben $\rightarrow$ wichtig für die LA's
\item Längere erster Prüfungszeitraum für die Prüfungen ohne LA's
 \end{itemize}
$\rightarrow$ Ruven meldet sich beim Studienbüro.
\subparagraph{Bachelor Informatik}\ \\
Studiumkomission am 17.1.2020 (morgen):
Ruven stellt den neuen aktualisierten Studienplan vor: Mosi und Ki als feste Module anstatt im großen Wahlpflichbereich.\\
Weitere Vorgeschlagene Erklärung: Mathe 1 und Lineare Optimierung tauschen. Berechenbarkeit und Komplexität mit Logik tauschen.

\subparagraph{Programmierausbildung}\ \\
\subparagraph{Berufungsverfahren}\ \\
Nichts.
\subparagraph{Master ITTi}\ \\
Stellungsname zur Änderung im Masterstudiengang ITTI 
Email an Professor Weber:
\begin{itemize}
	\item Was sind die Änderung im Kleinen
	 \item Warum werden wir so spät gefragt?
	 \item Forderung der Erklärung
\end{itemize}
Ruven schlägt dazu ein weitere Sitzung zu tagen bis Mittwoch. 
\subparagraph{Mathe}\ \\
Zentrale Frage: Hiwis zum Einsetzen für die Übungen oder zur Hausaufgabekontrolle? \\
$\rightarrow$ Hausaufgaben sollten weiterhin korrigiert werden. Der FSR würde Geld aus der Wohnsitzprämie dazugeben.
\section{FSR-Veranstaltungen}
\subparagraph{Weihnachtsfeier}\ \\
Geldbestand +- 0 
LTler bekommen noch ein paar Bierkästen zurück.
\subparagraph{Neujahrsgrillen}\ \\
Geldbestand +-0, Bestand leichtes -, aber Einnahmen \\
40-50 Besucher, war ganz gelungen und besucht für das Wetter.\\
Schichtplan war sehr schnell voll. 
\subparagraph{Skatturnier}\ \\
18 Leute waren da. -1 Kasten Bier, 15,49 ct Einahmen
\subparagraph{Smash}\ \\
Es läuft. Anmeldungen sind 22. Preise sind Gutscheine im Wert von 15 und 25 Euro für den Nintendo-Eshop. Turnier und Regeln sind geplant, Räume gebucht. \\
Abstimmungen:
\begin{itemize}
	\item \textbf{Abstimmung der Bereitstellungen Preise: Angenommen mit einer Enthaltung }\\
	\item \textbf{Geld für Plakat: Mit 2 Enthaltungen angenommen.}
\end{itemize}

\section{Berichte aus Gremiem und Arbeitsgruppen }
\subparagraph{AStA}\ \\
Wollen ein neues Auto kaufen für den Ausflug oder Gruppentickets.
\subparagraph{StuRa}\ \\ Es wurde berichtet.
\subparagraph{Fakultätsrat}\ \\ Es wurde berichtet.
\section{Sonstiges Lager und Büro}
1 Jahr seit dem letzten Putz. Papiermüll im Büro muss weg. Thora schlägt die Planung mit einem Doodle vor.

\section{Wohnsitzprämie}
Lautsprecher nicht benutzbar: Anschaffung von eventuell
\begin{itemize}
	\item soundbar
	\item Aktivlautsprecher
	\item normale Lautsprecher mit Verstärker
\end{itemize}
\section{Klausurtagung}
Ist durch.
\section{Sonstiges: Ausflug:}
Mehrere Optionen: Haus zu mieten mit 250 Euro pro Tag, Stralsund als praktische Alternative. Jugendherberge mieten.

\section{Schließung der Sitzung um 18:45}

\end{document}
